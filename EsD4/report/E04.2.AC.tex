\documentclass[10pt, a4paper, italian]{article}
\usepackage[T1]{fontenc}
\usepackage[utf8]{inputenc}
\usepackage{amsmath, amssymb, amsthm, thmtools, amsfonts, mathtools}
\usepackage{nicefrac}
\usepackage{calc}
\usepackage[pdftex, hyperindex, plainpages=false]{hyperref}
\usepackage[nameinlink]{cleveref} %load before classicthesis (clash)
%\usepackage[nochapters,pdfspacing]{classicthesis}
\usepackage{siunitx}
\usepackage[siunitx]{circuitikz}

\usepackage[a4paper]{geometry}
\usepackage{float}
\usepackage{mdframed}
\usepackage{titling}
\usepackage{booktabs}
\usepackage{graphicx}
\usepackage{caption, subcaption}
\usepackage{xcolor}
\usepackage[italian]{babel}
\usepackage{pgfplots}
\usepackage{listings}
%\usepackage{lmodern}
\usepackage{url}
\usepackage{enumitem}
\usepackage{tikz} %loads after classicthesis (xcolor incompat)

% lets graphicx know path where figures to be included are found
\graphicspath{{../figs/}}
\makeatletter
\def\input@path{{../figs/}}
%or: \def\input@path{{/path/to/folder/}{/path/to/other/folder/}}
\makeatother

% tikz pgf plots setup
\usepgfplotslibrary{external}
\pgfplotsset{compat=1.15}
\tikzexternalize

% spaces and significant digits/figures for measurements
\sisetup{free-standing-units, space-before-unit, number-unit-product = \;,
scientific-notation = true, round-mode = figures, round-precision = 2,}

% turns all (hyperlinked) references black [default is blue]
\hypersetup{
	linktoc=all,
	colorlinks=true,
	linkcolor=black
}

% code listings config
\lstset{
language=Python,
basicstyle=\ttfamily,
columns=fullflexible,
keepspaces=true,
}

% mdframed (for boxed text) configuration
\mdfsetup{linewidth=0.6pt}

% Default fixed font does not support bold face
\DeclareFixedFont{\ttb}{T1}{txtt}{bx}{n}{12} % for bold
\DeclareFixedFont{\ttm}{T1}{txtt}{m}{n}{12}  % for normal

% Custom colors
\usepackage{color}
\definecolor{deepblue}{rgb}{0,0,0.5}
\definecolor{deepred}{rgb}{0.6,0,0}
\definecolor{deepgreen}{rgb}{0,0.5,0}

% Commands 
\newcommand{\executeiffilenewer}[3]{%
	\ifnum\pdfstrcmp{\pdffilemoddate{#1}}%
		{\pdffilemoddate{#2}}>0%
	{\immediate\write18{#3}}\fi%
}
% input .svg --> .pdf_tex graphs
\newcommand{\includesvg}[1]{%
	\executeiffilenewer{#1.svg}{#1.pdf}%
	{inkscape -z -D --file=#1.svg %
	--export-pdf=#1.pdf --export-latex}%
	\input{#1.pdf_tex}%
}
% Thanks UniPi's Department of Physics E. Fermi
\newcommand{\thanksdf}{(\thanks{Dipartimento di Fisica E.~Fermi,%
Universit\`a di Pisa - Pisa, Italy.}\;)}

% hyperlink to email address
\newcommand{\mail}[1]{\href{mailto:#1}{\textsf{#1}}}

\input{../../latex/math}
\usepackage{multicol}
\geometry{left=2cm, right=2cm, top=2cm, bottom=2cm}
\usepackage{colortbl}
\usepackage{diagbox}
\usepackage[T1]{fontenc}
\usepackage[utf8]{inputenc}
\usepackage{graphicx}
\usepackage{xcolor}
\usepackage{tkz-graph}
\usepackage{arydshln}
\usetikzlibrary{automata, positioning, arrows}
\newenvironment{FSM}{
\begin{tikzpicture}
\tikzset{
->, % makes the edges directed
>=stealth', % makes the arrow heads bold
node distance=2cm, % specifies the minimum distance between two nodes. Change if necessary.
every state/.style={minimum size = 1cm, thick, fill=gray!10}, % sets the properties for each 'state' node
}
}{
\end{tikzpicture}
}


\lstset{%
  language = Octave,
  backgroundcolor=\color{white},   
  basicstyle=\footnotesize\ttfamily,       
  breakatwhitespace=false,         
  breaklines=true,                 
  captionpos=b,                   
  commentstyle=\color{gray},    
  deletekeywords={...},           
  escapeinside={\%*}{*)},          
  extendedchars=true,              
  frame=single,                    
  keepspaces=true,                 
  keywordstyle=\color{orange},       
  morekeywords={*,...},            
  numbers=left,                    
  numbersep=5pt,                   
  numberstyle=\footnotesize\color{gray}, 
  rulecolor=\color{black},         
  rulesepcolor=\color{blue},
  showspaces=false,                
  showstringspaces=false,          
  showtabs=false,                  
  stepnumber=2,                    
  stringstyle=\color{orange},    
  tabsize=2,                       
  title=\lstname,
  emphstyle=\bfseries\color{blue}%  style for emph={} 
} 

%% language specific settings:
\lstdefinestyle{Arduino}{%
    language = Octave,
    keywords={void, int boolean},%                 define keywords
    morecomment=[l]{//},%             treat // as comments
    morecomment=[s]{/*}{*/},%         define /* ... */ comments
    emph={HIGH, OUTPUT, LOW}%        keywords to emphasize
}

% indexes subsections with letters, sections with numbers (1.a, 1.b, ...)
\renewcommand{\thesubsection}{\thesection.\alph{subsection}}

% lets graphicx know path where figures to be included are found
\graphicspath{{../figs/}}

\newcommand{\dontcare}{X}
\author{Gruppo 1.AC \\ Matteo Rossi, Bernardo Tomelleri}
\title{EsD4 ADC-DAC: Convertitore sigma-delta}
\begin{document}
\date{\today}
\maketitle

\section*{Misura componenti dei circuiti}
Riportiamo per completezza il valore della tensione continua di
alimentazione per i circuiti integrati misurata con il multimetro
\begin{align*}
V_{CC} &= 4.99 \pm 0.03 \si{\V} \\
V_{EE} &= -4.99 \pm 0.03 \si{\V}
\end{align*}
e i valori di capacità dei condensatori di disaccoppiamento che collegano le
linee di alimentazione a massa (sempre misurato con il multimetro)
\begin{align*}
C_{d+} &= 97 \pm 4 \; \si{n\F} \\
C_{d-} &= 107 \pm 4 \; \si{n\F}
\end{align*}

Abbiamo inoltre misurato i valori dei vari componenti del circuito:
\begin{tabular}{cccccc}
\toprule
Resistenze $[\si{\ohm}]$ & $R$ & $\sigma R$ & Capacità $[\si{n\F}]$ & $C$ &
$\sigma C$ \\
\midrule
\midrule
$R_1$	  	& 996 	& 8		& $C_1$ & 94	& 4 \\
$R_2$	  	& 994	& 8		& & & \\
$R_3$	  	& 999	& 8		& & & \\
$R_4$	  	& 994 	& 8	& & & \\
$R_5$	  	& 997	& 8 & & & \\

\bottomrule   
\end{tabular}
\caption{Valori di resistenza e capacità misurate per i componenti del primo
circuito studiato.}
\end{table}
\begin{tabular}{cccccc}
\toprule
Resistenze $[\si{\ohm}]$ & $R$ & $\sigma R$ & Capacità $[\si{n\F}]$ & $C$ &
$\sigma C$ \\
\midrule
\midrule
$R_1$	  	& 995 	& 8		& $C_1$ & 109	& 4 \\
$R_2$	  	& 999	& 8		& & & \\
$R_3$	  	& 998	& 8		& & & \\
$R_4$	  	& 998 	& 8	& & & \\
$R_5$	  	& 996	& 8 & & & \\

\bottomrule   
\end{tabular}
\caption{Valori di resistenza e capacità misurate per i componenti del secondo
circuito studiato.}
\end{table}
\setcounter{section}{0}

%=======================
\section{Analisi e costruzione del circuito}\label{sec: IC}
\begin{figure}[htbp]
    \centering
	\includegraphics[width=\textwidth]{schem}
    \caption{Schematica del circuito utilizzato per il convertitore analogico digitale}
\end{figure}
\subsection{Verifica del funzionamento}
\begin{figure}[htbp]
    \centering
	\includegraphics[width=\textwidth]{MIDDLE.U1.InputVSOutput}
    \caption{Acquisizione del segnale nell'ingresso invertente (canale 1) e dell'uscita (canale 2) del circuito integratore}
\end{figure}
\begin{figure}[htbp]
    \centering
	\includegraphics[width=\textwidth]{MIDDLE.U2.InputVSOutput}
    \caption{Acquisizione del segnale nell'ingresso non invertente (canale 1) e dell'uscita (canale 2) del circuito comparatore semplice}
\end{figure}
\begin{figure}[htbp]
    \centering
	\includegraphics[width=\textwidth]{MIDDLE.U3.InputVSOutput}
    \caption{Acquisizione del segnale in ingresso(canale 1) e in uscita (canale 2) dal Flip Flop}
\end{figure}
\begin{figure}[htbp]
    \centering
	\includegraphics[width=\textwidth]{MIDDLE.U3vU4}
    \caption{Acquisizione del segnale in uscita dal pin Q del Flip Flop(canale 1) e in uscita dal DAC (canale 2)}
\end{figure}

\begin{figure}[htbp]
    \centering
	\includegraphics[width=\textwidth]{BOTTOM}
    \caption{Acquisizione del segnale analogico in ingresso (un'onda sinusoidale di frequenza pari a 10 Hz e ampiezza pari a $2.5 V$) e del segnale logico in uscita dal pin Q del Flip-Flop durante il minimo del segnale}
\end{figure}
\begin{figure}[htbp]
    \centering
	\includegraphics[width=\textwidth]{TOP}
    \caption{Acquisizione del segnale analogico in ingresso (un'onda sinusoidale di frequenza pari a 10 Hz e ampiezza pari a $2.5 V$) e del segnale logico in uscita dal pin Q del Flip-Flop durante il massimo del segnale}
\end{figure}
\begin{figure}[htbp]
    \centering
	\includegraphics[width=\textwidth]{MIDDLE}
    \caption{Acquisizione del segnale analogico in ingresso (un'onda sinusoidale di frequenza pari a 10 Hz e ampiezza pari a $2.5 V$) e del segnale logico in uscita dal pin Q del Flip-Flop durante il punto medio dell'onda}
\end{figure}
\begin{figure}[htbp]
    \centering
	\includegraphics[width=\textwidth]{Conv.Sinusoide.100Hz.2}
    \caption{Acquisizione del segnale analogico in ingresso (un'onda sinusoidale di frequenza pari a 100 Hz e ampiezza pari a $2.5 V$) e del segnale in uscita dal DAC U4}
\end{figure}
%=======================
\section{Descrizione delle misure e acquisizione dati}

\subsection{Campionamento e acquisizione del segnale}

\subsection{Acquisizione con Protocol}

%=======================
\section{Analisi dei dati}

\subsection{Ricostruzione dei segnali in ingresso}

\subsection{Fit sinusoidale}

\subsection{Risposta in frequenza dell'ADC}

\subsection{Stima del fattore di calibrazione del convertitore}


\subsection{Misura del signal/noise ratio (SNR)}


%=======================
\section*{Conclusioni e commenti finali}
Si è riusciti a progettare, costruire e verificare il corretto funzionamento
di circuiti logici combinatori di diversa complessità e svariate applicazioni
(e.g., sistemi di controllo e misura) costruiti con porte NOT, NAND, OR e D-FF.
Inoltre si è riusciti ad apprezzare le diverse modalità di funzionamento delle
macchine a stati finiti implementate secondo i modelli Moore e Mealy, ponendo
particolare attenzione alle loro diverse temporizzazioni nei cambiamenti di
stato, nonostante la bassa risoluzione temporale dell'AD2.

%=======================
\section*{Dichiarazione}
I firmatari di questa relazione dichiarano che il contenuto della relazione \`e
originale, con misure effettuate dai membri del gruppo, e che tutti i firmatari
hanno contribuito alla elaborazione della relazione stessa.

\end{document}
\iffalse
⣿⣿⣿⣿⣿⣿⣿⣿⣿⣿⣿⣿⣿⣿⣿⣿⣿⣿⣿⣿⣿⣿⣿⣿⣿⣿⣿⣿⣿⣿⣿⣿⣿⣿⣿⣿⣿⣿⣿⣿⣿⣿⣿⣿⣿⣿⣿⣿⣿⣿
⣿⣿⣿⣿⣿⣿⣿⣿⣿⣿⣿⣿⣿⣿⣿⣿⣿⣿⣿⣿⣿⣿⣿⣿⣿⣿⣿⣿⣿⣿⣿⣿⣿⣿⣿⣿⣿⣿⣿⣿⣿⣿⣿⣿⣿⣿⣿⣿⣿⣿
⣿⣿⣿⣿⣿⣿⣿⣿⣿⣿⣿⣿⣿⣿⣿⣿⣿⣿⣿⣿⣿⣿⣿⣿⡿⢿⣿⣿⣿⣿⣿⣿⣿⣿⣿⣿⣿⣿⣿⣿⣿⣿⣿⣿⣿⣿⣿⣿⣿⣿
⣿⣿⣿⣿⣿⣿⣿⣿⣿⣿⣿⣿⣿⣿⣿⣿⣿⣿⣿⣿⣿⣿⣿⡟⠀⠀⢻⣿⣿⣿⣿⣿⣿⣿⣿⣿⣿⣿⣿⣿⣿⣿⣿⣿⣿⣿⣿⣿⣿⣿
⣿⣿⣿⣿⣿⣿⣿⣿⣿⣿⣿⣿⣿⣿⣿⣿⣿⣿⣿⣿⣿⣿⠟⠀⠐⠂⠀⠻⣿⣿⣿⣿⣿⣿⣿⣿⣿⣿⣿⣿⣿⣿⣿⣿⣿⣿⣿⣿⣿⣿
⣿⣿⣿⣿⣿⣿⣿⣿⣿⣿⣿⣿⣿⣿⣿⣿⣿⣿⣿⣿⣿⠏⠀⡠⠴⠤⣤⠀⠹⣿⣿⣿⣿⣿⣿⣿⣿⣿⣿⣿⣿⣿⣿⣿⣿⣿⣿⣿⣿⣿
⣿⣿⣿⣿⣿⣿⣿⣿⣿⣿⣿⣿⣿⣿⣿⣿⣿⣿⣿⣿⠏⠀⢀⣀⣀⣀⣀⡀⠀⠹⣿⣿⣿⣿⣿⣿⣿⣿⣿⣿⣿⣿⣿⣿⣿⣿⣿⣿⣿⣿
⣿⣿⣿⣿⣿⣿⣿⣿⣿⣿⣿⣿⣿⣿⣿⣿⣿⣿⡿⠃⠀⠈⠉⢁⣈⣿⣿⣿⣷⡀⠘⢿⣿⣿⣿⣿⣿⣿⣿⣿⣿⣿⣿⣿⣿⣿⣿⣿⣿⣿
⣿⣿⣿⣿⣿⣿⣿⣿⣿⣿⣿⣿⣿⣿⣿⣿⣿⡿⠃⠠⠴⠒⠛⠛⠛⠛⠛⠛⠛⠷⠄⠘⢿⣿⣿⣿⣿⣿⣿⣿⣿⣿⣿⣿⣿⣿⣿⣿⣿⣿
⣿⣿⣿⣿⣿⣿⣿⣿⣿⣿⣿⣿⣿⣿⣿⣿⡟⠁⠀⣀⣤⣴⣶⣶⣶⣶⣶⣶⣤⣄⣀⠀⠈⢻⣿⣿⣿⣿⣿⣿⣿⣿⣿⣿⣿⣿⣿⣿⣿⣿
⣿⣿⣿⣿⣿⣿⣿⣿⣿⣿⣿⣿⣿⣿⣿⠟⠀⣠⣾⣿⠿⠟⠛⠛⠛⠛⠛⠿⠿⣿⣿⣷⣆⠀⠻⣿⣿⣿⣿⣿⣿⣿⣿⣿⣿⣿⣿⣿⣿⣿
⣿⣿⣿⣿⣿⣿⣿⣿⣿⣿⣿⣿⣿⣿⠟⠀⠐⠋⠉⣀⡠⠤⠔⠒⠒⠒⠠⠤⢀⡀⠉⠛⠿⣆⠀⠻⣿⣿⣿⣿⣿⣿⣿⣿⣿⣿⣿⣿⣿⣿
⣿⣿⣿⣿⣿⣿⣿⣿⣿⣿⣿⣿⣿⠋⠀⢀⣠⠔⠊⠁⠀⢠⣤⣤⣶⣤⣤⣤⡀⠈⠑⠠⢄⠈⠁⠀⠙⣿⣿⣿⣿⣿⣿⣿⣿⣿⣿⣿⣿⣿
⣿⣿⣿⣿⣿⣿⣿⣿⣿⣿⣿⣿⠋⠀⢴⠋⠀⠀⠰⣧⠀⢸⣿⣿⣿⣿⣿⣿⡇⠀⣴⣄⠀⠑⠢⣤⡀⠙⣿⣿⣿⣿⣿⣿⣿⣿⣿⣿⣿⣿
⣿⣿⣿⣿⣿⣿⣿⣿⣿⣿⡿⠁⢀⣄⠀⠀⢾⣦⠀⠙⠦⠀⠙⠿⠿⠿⠿⠋⠀⣴⠋⠁⣀⣤⣠⣿⣷⡀⠈⢿⣿⣿⣿⣿⣿⣿⣿⣿⣿⣿
⣿⣿⣿⣿⣿⣿⣿⣿⣿⡟⠁⢀⠀⠙⢷⣶⣾⣿⣷⣤⣄⣰⣦⣄⣀⣀⣠⣴⣾⣿⣷⠾⠿⠀⠈⠉⠛⠓⠀⠈⢻⣿⣿⣿⣿⣿⣿⣿⣿⣿
⣿⣿⣿⣿⣿⣿⣿⣿⡟⠀⢀⠈⠓⢤⣀⠉⠙⠻⠿⠿⣿⣿⣿⣿⣿⣿⣿⣿⣿⣅⣀⣠⣤⣄⡀⠀⠒⠲⠶⠄⠀⢻⣿⣿⣿⣿⣿⣿⣿⣿
⣿⣿⣿⣿⣿⣿⣿⠏⠀⠀⠉⠀⠀⠀⠙⠷⣶⣤⣤⣀⣀⣀⡉⠉⣹⣿⣿⣿⣿⣿⣿⣿⣿⣿⣿⣶⣦⣤⣀⣀⣀⠀⠹⣿⣿⣿⣿⣿⣿⣿
⣿⣿⣿⣿⣿⣿⠏⠀⣰⣿⣷⣄⡀⠙⠢⡀⠈⠻⣿⣿⣿⣿⣿⣿⣿⣿⣿⣿⣿⣿⣿⣿⣿⣿⣿⣿⣿⣿⣿⣿⣿⣆⠀⠹⣿⣿⣿⣿⣿⣿
⣿⣿⣿⣿⣿⣇⣀⣀⣀⣀⣀⣀⣀⣀⣀⣀⣀⣀⣀⣀⣀⣀⣀⣀⣀⣀⣀⣀⣀⣀⣀⣀⣀⣀⣀⣀⣀⣀⣀⣀⣀⣀⣀⣀⣸⣿⣿⣿⣿⣿
⣿⣿⣿⣿⣿⣿⣿⣿⣿⣿⣿⣿⣿⣿⣿⣿⣿⣿⣿⣿⣿⣿⣿⣿⣿⣿⣿⣿⣿⣿⣿⣿⣿⣿⣿⣿⣿⣿⣿⣿⣿⣿⣿⣿⣿⣿⣿⣿⣿⣿
⣿⣿⣿⣿⣿⣿⣿⣿⣿⣿⣿⣿⣿⣿⣿⣿⣿⣿⣿⣿⣿⣿⣿⣿⣿⣿⣿⣿⣿⣿⣿⣿⣿⣿⣿⣿⣿⣿⣿⣿⣿⣿⣿⣿⣿⣿⣿⣿⣿⣿
⣿⣿⣿⣿⣿⣿⣿⣿⣿⣿⣿⣿⣿⣿⣿⣿⣿⣿⣿⣿⣿⣿⣿⣿⣿⣿⣿⣿⣿⣿⣿⣿⣿⣿⣿⣿⣿⣿⣿⣿⣿⣿⣿⣿⣿⣿⣿⣿⣿⣿
⣿⣿⣿⣿⣿⣿⣿⣿⣿⣿⣿⣿⣿⣿⣿⣿⣿⣿⣿⣿⣿⣿⣿⣿⣿⣿⣿⣿⣿⣿⣿⣿⣿⣿⣿⣿⣿⣿⣿⣿⣿⣿⣿⣿⣿⣿⣿⣿⣿⣿
⣿⣿⣿⣿⣿⣿⣿⣿⣿⣿⣿⣿⣿⣿⣿⣿⣿⣿⣿⣿⣿⣿⣿⣿⣿⣿⣿⣿⣿⣿⣿⣿⣿⣿⣿⣿⣿⣿⣿⣿⣿⣿⣿⣿⣿⣿⣿⣿⣿⣿
⣿⣿⣿⣿⣿⣿⠿⢋⣥⣴⣶⣶⣶⣬⣙⠻⠟⣋⣭⣭⣭⣭⡙⠻⣿⣿⣿⣿⣿
⣿⣿⣿⣿⡿⢋⣴⣿⣿⠿⢟⣛⣛⣛⠿⢷⡹⣿⣿⣿⣿⣿⣿⣆⠹⣿⣿⣿⣿
⣿⣿⣿⡿⢁⣾⣿⣿⣴⣿⣿⣿⣿⠿⠿⠷⠥⠱⣶⣶⣶⣶⡶⠮⠤⣌⡙⢿⣿
⣿⡿⢛⡁⣾⣿⣿⣿⡿⢟⡫⢕⣪⡭⠥⢭⣭⣉⡂⣉⡒⣤⡭⡉⠩⣥⣰⠂⠹
⡟⢠⣿⣱⣿⣿⣿⣏⣛⢲⣾⣿⠃⠄⠐⠈⣿⣿⣿⣿⣿⣿⠄⠁⠃⢸⣿⣿⡧
⢠⣿⣿⣿⣿⣿⣿⣿⣿⣇⣊⠙⠳⠤⠤⠾⣟⠛⠍⣹⣛⣛⣢⣀⣠⣛⡯⢉⣰
⣿⣿⣿⣿⣿⣿⣿⣿⣿⣿⣿⣿⣷⡶⠶⢒⣠⣼⣿⣿⣛⠻⠛⢛⣛⠉⣴⣿⣿
⣿⣿⣿⣿⣿⣿⣿⡿⢛⡛⢿⣿⣿⣶⣿⣿⣿⣿⣿⣿⣿⣿⣿⣿⣿⣷⡈⢿⣿
⣿⣿⣿⣿⣿⣿⣿⠸⣿⡻⢷⣍⣛⠻⠿⠿⣿⣿⣿⣿⣿⣿⣿⣿⣿⠿⢇⡘⣿
⣿⣿⣿⣿⣿⣿⣿⣷⣝⠻⠶⣬⣍⣛⣛⠓⠶⠶⠶⠤⠬⠭⠤⠶⠶⠞⠛⣡⣿
⢿⣿⣿⣿⣿⣿⣿⣿⣿⣿⣷⣶⣬⣭⣍⣙⣛⣛⣛⠛⠛⠛⠿⠿⠿⠛⣠⣿⣿
⣦⣈⠉⢛⠻⠿⠿⢿⣿⣿⣿⣿⣿⣿⣿⣿⣿⣿⣿⡿⠿⠛⣁⣴⣾⣿⣿⣿⣿
⣿⣿⣿⣶⣮⣭⣁⣒⣒⣒⠂⠠⠬⠭⠭⠭⢀⣀⣠⣄⡘⠿⣿⣿⣿⣿⣿⣿⣿
⣿⣿⣿⣿⣿⣿⣿⣿⣿⣿⣿⣿⣿⣿⣿⣿⣿⣿⣿⣿⣿⣦⡈⢿⣿⣿⣿⣿⣿
⠀⠀⠀⠀⠀⠀⠀⠀⠀⠀⠀⠀⠀⠀⠀⠀⠀⠀⠀⠀⠀⠀⠀⠀⠀⠀⠀⠀⠀⠀⠀⠀⠀⠀⣀⣀⣀⣀⣀⠀⠀⠀⠀⠀⠀⠀⠀⠀⠀⠀
⠀⠀⠀⠀⠀⠀⠀⠀⠀⠀⠀⠀⠀⠀⠀⠀⠀⠀⠀⠀⠀⠀⠀⠀⠀⠀⠀⠀⠀⠀⠀⠀⢀⣾⠋⠉⠉⠉⠙⠛⠻⢶⣤⡀⠀⠀⠀⠀⠀⠀
⠀⠀⠀⠀⠀⠀⠀⠀⠀⠀⠀⠀⠀⠀⠀⠀⠀⠀⠀⠀⠀⠀⠀⠀⠀⠀⠀⠀⠀⠀⠀⣠⡾⠁⠀⠀⠀⠀⠀⠀⠀⠀⠈⢻⡆⠀⠀⠀⠀⠀
⠀⠀⠀⠀⠀⠀⠀⠀⠀⠀⠀⠀⠀⠀⠀⠀⠀⠀⠀⠀⠀⠀⠀⠀⠀⠀⢀⣀⣀⣠⣴⠟⠀⠀⠀⠀⠀⠀⠀⠀⠀⠀⣠⡟⠁⠀⠀⠀⠀⠀
⠀⠀⠀⠀⠀⠀⠀⠀⠀⠀⠀⠀⠀⠀⠀⠀⠀⠀⠀⠀⠀⢀⣠⣴⠶⠟⠛⠉⠉⠁⠀⠀⠀⠀⠀⠀⠀⠀⠀⠀⠀⢰⡟⠀⠀⠀⠀⠀⠀⠀
⠀⠀⠀⠀⠀⠀⠀⠀⠀⠀⠀⠀⠀⠀⠀⠀⠀⣀⣠⣴⠾⠛⠉⠀⠀⠀⠀⠀⠀⠀⠀⠀⠀⠀⠀⠀⠀⠀⠀⠀⢀⣾⠁⠀⠀⠀⠀⠀⠀⠀
⠀⠀⠀⠀⠀⠀⠀⠀⠀⠀⠀⠀⠀⣠⡶⠛⠛⠋⠉⠀⠀⠀⠀⠀⠀⠀⠀⠀⠀⠀⠀⠀⠀⠀⠀⠀⠀⠀⠀⣴⠟⠁⠀⠀⠀⠀⠀⠀⠀⠀
⠀⠀⠀⠀⠀⠀⠀⠀⠀⠀⠀⠀⣰⡟⠀⠀⠀⠀⠀⠀⠀⠀⠀⠀⠀⠀⠀⠀⠀⠀⠀⠀⠀⠀⠀⠀⠀⢀⣾⠋⠀⠀⠀⠀⠀⠀⠀⠀⠀⠀
⠀⠀⠀⠀⠀⠀⠀⠀⠀⠀⣴⠞⠋⠀⠀⠀⠀⣠⣴⠶⠶⣦⡀⠀⠀⠀⠀⠀⠀⠀⠀⠀⠀⠀⠀⠀⣠⡿⠁⠀⠀⠀⠀⠀⠀⠀⠀⠀⠀⠀
⠀⠀⠀⠀⠀⠀⠀⠀⠀⠀⢿⣄⣀⣀⣤⡴⠟⠋⠀⠀⠀⠘⣷⡀⠀⠀⠀⠀⠀⠀⠀⠀⠀⠀⠀⣰⠟⠀⠀⠀⠀⠀⠀⠀⠀⠀⠀⠀⠀⠀
⠀⠀⠀⠀⠀⠀⠀⠀⠀⠀⢰⣟⠉⠙⢷⣄⠀⠀⠀⠀⠀⠀⠘⣷⠀⠀⠀⠀⠀⠀⠀⠀⠀⠀⣼⠏⠀⠀⠀⠀⠀⠀⠀⠀⠀⠀⠀⠀⠀⠀
⠀⠀⠀⠀⠀⠀⠀⠀⠀⠀⠀⠙⣷⡀⠀⠹⣧⡀⠀⠀⠀⠀⠀⣼⠃⠀⠀⠀⠀⠀⠀⠀⢀⣾⠃⠀⠀⠀⠀⠀⠀⠀⠀⠀⠀⠀⠀⠀⠀⠀
⠀⠀⠀⠀⠀⠀⠀⠀⠀⠀⠀⠀⠘⣷⡀⠀⠈⠛⣷⣤⣴⠶⠟⠃⠀⠀⠀⠀⠀⠀⠀⢀⣾⠃⠀⠀⠀⠀⠀⠀⠀⠀⠀⠀⠀⠀⠀⠀⠀⠀
⠀⠀⠀⠀⠀⠀⠀⠀⠀⠀⠀⠀⠀⠘⣷⡀⠀⠀⠈⠀⠀⠀⠀⠀⠀⠀⠀⠀⠀⠀⢀⣾⠃⠀⠀⠀⠀⠀⠀⠀⠀⠀⠀⠀⠀⠀⠀⠀⠀⠀
⠀⠀⠀⠀⠀⠀⠀⠀⠀⠀⠀⠀⠀⠀⠈⢷⣄⠀⠀⠀⠀⠀⠀⠀⠀⠀⠀⠀⠀⠀⢸⡇⠀⠀⠀⠀⠀⠀⠀⠀⠀⠀⠀⠀⠀⠀⠀⠀⠀⠀
⠀⠀⠀⠀⠀⠀⠀⠀⠀⠀⠀⠀⠀⠀⠀⠀⠙⠻⠟⠛⣻⡿⠀⠀⠀⠀⠀⢀⣤⡾⠛⠀⠀⠀⠀⠀⠀⠀⠀⠀⠀⠀⠀⠀⠀⠀⠀⠀⠀⠀
⠀⠀⠀⠀⠀⠀⠀⠀⠀⠀⠀⠀⠀⠀⠀⠀⠀⠀⣠⡾⠋⠀⠀⠀⠀⢀⣴⠟⣿⠀⠀⠀⠀⠀⠀⠀⠀⠀⠀⠀⠀⠀⠀⠀⠀⠀⠀⠀⠀⠀
⠀⠀⠀⠀⠀⠀⠀⠀⠀⠀⠀⠀⠀⠀⠀⠀⠀⣴⠟⠀⠀⠀⠀⣠⣾⡿⠁⢀⣿⠀⠀⠀⠀⠀⠀⠀⠀⠀⠀⠀⠀⠀⠀⠀⠀⠀⠀⠀⠀⠀
⠀⠀⠀⠀⠀⠀⠀⠀⠀⠀⠀⠀⠀⠀⠀⠀⣼⠏⠀⠀⠀⢀⡾⢫⣿⠁⠀⣼⠇⠀⠀⠀⠀⠀⠀⠀⠀⠀⠀⠀⠀⠀⠀⠀⠀⠀⠀⠀⠀⠀
⠀⠀⠀⠀⠀⠀⠀⠀⠀⠀⠀⠀⠀⠀⢀⡾⠃⠀⠀⢀⣴⠟⢡⡟⣿⠀⠀⣿⠀⠀⠀⠀⠀⠀⠀⠀⠀⠀⠀⠀⠀⠀⠀⠀⠀⠀⠀⠀⠀⠀
⠀⠀⠀⠀⠀⠀⠀⠀⠀⠀⠀⠀⠀⣰⠟⠁⠀⠀⣰⠟⠁⢠⡿⠀⢿⡄⢰⡟⠀⠀⠀⠀⠀⠀⠀⠀⠀⠀⠀⠀⠀⠀⠀⠀⠀⠀⠀⠀⠀⠀
⠀⠀⠀⠀⠀⠀⠀⠀⠀⠀⠀⠀⣼⠏⠀⠀⢠⣾⠋⠀⢀⡿⠁⠀⠈⠛⠿⠃⠀⠀⠀⠀⠀⠀⠀⠀⠀⠀⠀⠀⠀⠀⠀⠀⠀⠀⠀⠀⠀⠀
⠀⠀⠀⠀⠀⠀⠀⠀⠀⠀⠀⢸⡏⠀⠀⣴⢿⡟⠀⠀⣾⠃⠀⠀⠀⠀⠀⠀⠀⠀⠀⠀⠀⠀⠀⠀⠀⠀⠀⠀⠀⠀⠀⠀⠀⠀⠀⠀⠀⠀
⠀⠀⠀⠀⠀⠀⠀⠀⠀⠀⠀⠈⢷⣄⣾⠋⠘⣧⡀⢰⡟⠀⠀⠀⠀⠀⠀⠀⠀⠀⠀⠀⠀⠀⠀⠀⠀⠀⠀⠀⠀⠀⠀⠀⠀⠀⠀⠀⠀⠀
⠀⠀⠀⠀⠀⠀⠀⠀⠀⠀⠀⠀⠀⠉⠁⠀⠀⠈⠛⠋⠀⠀⠀⠀⠀⠀⠀⠀⠀⠀⠀⠀⠀⠀⠀⠀⠀⠀⠀⠀⠀⠀⠀⠀⠀⠀⠀⠀⠀⠀
⠀⠀⠀⠀⠀⠀⠀⠀⠀⠀⠀⠀⠀⠀⠀⠀⠀⠀⠀⠀⠀⠀⢀⡤⠶⢶⣶⣦⣄⡀⠀⠀
⠀⠀⠀⠀⠀⠀⠀⠀⠀⠀⠀⠀⠀⠀⠀⠀⠀⠀⠀⠀⠀⣠⣾⣿⡄⠒⠪⢝⠻⣿⣿⣦⡀
⠀⠀⠀⠀⠀⠀⠀⠀⠀⠀⠀⠀⠀⠀⠀⠀⠀⠀⠀⠀⣴⣿⡿⢉⡀⠀⠈⠐⠄⢿⣿⣿⣷
⠀⠀⠀⠀⠀⠀⠀⠀⠀⠀⠀⠀⠀⠀⠀⠀⠀⠀⠀⣼⣿⡇⠀⠀⠈⡄⠤⢀⠈⣾⣿⣿⣿
⠀⠀⠀⠀⠀⠀⠀⠀⠀⠀⠀⠀⠀⠀⠀⠀⠀⢀⡾⣿⣟⣕⡤⡠⠘⠀⠀⠀⢱⣿⣿⣿⣿
⠀⠀⠀⠀⠀⠀⠀⠀⠀⠀⠀⠀⠀⠀⠀⠀⢠⣾⣾⣿⣞⣄⠮⠔⠈⡢⠄⣠⣾⣿⣿⣿⣿
⠀⠀⠀⠀⠀⠀⠀⠀⠀⠀⠀⠀⠀⠀⠀⠠⢿⣿⢽⡻⣿⣿⣿⣽⣵⣾⡽⣿⣿⣿⣿⣿⡏
⠀⠀⠀⠀⠀⠀⠀⠀⠀⠀⠀⠀⠀⠀⢠⣗⣿⡟⠈⠉⠚⢽⣻⢷⡝⣿⡿⣿⣿⣿⣿⡿⠀
⠀⠀⠀⠀⠀⠀⠀⠀⠀⠀⠀⠀⠀⢠⣿⣿⣿⠇⠀⠀⠀⠀⠀⢩⣯⣭⣾⣿⣿⣿⣿⠁⠀
⠀⠀⠀⠀⠀⠀⠀⠀⠀⠀⠀⠀⣰⣿⣿⣿⠃⠀⠀⠀⠀⠀⣰⣿⣿⣿⣿⣿⣿⣿⠃⠀⠀
⠀⠀⠀⠀⠀⠀⠀⠀⠀⠀⠀⣴⣿⣿⡿⠃⠀⠀⠀⠀⣠⣼⣿⣿⣿⣿⣿⣿⣿⠃⠀⠀⠀
⠀⠀⠀⠀⠀⠀⠀⠀⠀⢀⢞⣿⣿⡿⠁⠀⠀⠀⣠⣾⣿⣿⣿⣿⣿⣿⣿⣿⠃⠀⠀⠀⠀
⠀⠀⠀⠀⠀⠀⠀⠀⣰⣿⣾⣿⠏⠀⠀⠀⠀⣾⣿⣿⣿⣿⣿⣿⡟⣿⣿⠏⠀⠀⠀⠀⠀
⠀⠀⠀⠀⠀⠀⣠⣾⣿⣿⣿⠃⠀⠀⠀⠀⣸⣿⣿⣿⣿⣿⣟⣷⣾⣿⠏⠀⠀⠀⠀⠀⠀
⠀⠀⠀⠀⢀⣾⣿⣿⣿⣿⠃⠀⠀⠀⢀⣴⣿⣿⣿⣿⣿⣿⣿⣿⣿⠋⠀⠀⠀⠀⠀⠀⠀
⠀⠀⠀⡰⢿⣿⣿⣯⡶⠁⠀⠀⢀⣴⣿⣿⣿⣿⣿⣿⣿⣿⣿⡿⠃⠀⠀⠀⠀⠀⠀⠀⠀
⠀⢀⣼⣟⣿⣿⡿⠃⠀⠀⢀⣴⣿⣿⣿⣿⣿⣿⣿⣿⣿⣿⡟⠁⠀⠀⠀⠀⠀⠀⠀⠀⠀
⢀⣾⣿⣿⡯⣿⠀⠀⢠⣴⣿⣿⣿⣿⣿⣿⣿⣿⣿⣿⣿⠏⠀⠀⠀⠀⠀⠀⠀⠀⠀⠀⠀⠀⠀
⣾⣿⣿⣿⣿⣿⣶⣶⣿⣿⣿⣿⣿⣿⣿⣿⣿⣿⣿⡿⠁⠀⠀⠀⠀⠀⠀⠀⠀⠀⠀⠀⠀
⠿⣿⡿⣿⣿⣿⣿⣿⣿⣿⣿⣿⣿⣿⣿⣿⢿⡿⠋⠀⠀⠀⠀⠀⠀⠀⠀⠀⠀⠀⠀⠀⠀
⠀⠛⠀⠈⠀⠻⣿⣿⣿⣿⣿⣟⣛⣿⣿⡭⠋⠀⠀⠀⠀⠀⠀⠀⠀⠀⠀⠀⠀⠀⠀⠀⠀
⠢⡀⠀⠀⠀⠀⠻⣿⣿⣿⣿⣿⣭⠟⠉⠀⠀⠀⠀⠀⠀⠀⠀⠀⠀⠀⠀⠀⠀⠀⠀⠀⠀
⠀⠈⠐⠤⢀⡀⠀⢀⣙⣿⠿⠋⠁⠀⠀⠀⠀⠀⠀⠀⠀⠀⠀⠀⠀⠀⠀⠀⠀⠀⠀⠀⠀
⠀⠀⠀⠀⠀⠀⠀⠀⠀⠀⠀⠀⠀⠀⠀⠀⠀⠀⠀⠀⠀⠀⠀⠀
⣿⣿⣿⣿⣿⣿⣿⣿⣿⣿⣿⣿⣿⣿⣿⡿⠿⠛⠛⠛⠛⠿⣿⣿⣿⣿⣿⣿⣿⣿
⣿⣿⣿⣿⣿⣿⣿⣿⣿⣿⣿⡿⠛⠉⠁⠀⠀⠀⠀⠀⠀⠀⠉⠻⣿⣿⣿⣿⣿⣿
⣿⣿⣿⣿⣿⣿⣿⣿⣿⣿⡟⠁⠀⠀⠀⠀⠀⠀⠀⠀⠀⠀⠀⠀⠘⢿⣿⣿⣿⣿
⣿⣿⣿⣿⣿⣿⣿⣿⣿⡟⠁⠀⠀⠀⠀⠀⠀⠀⠀⠀⠀⠀⠀⠀⠀⣾⣿⣿⣿⣿
⣿⣿⣿⣿⣿⣿⣿⠋⠈⠀⠀⠀⠀⠐⠺⣖⢄⠀⠀⠀⠀⠀⠀⠀⠀⣿⣿⣿⣿⣿
⣿⣿⣿⣿⣿⣿⡏⢀⡆⠀⠀⠀⢋⣭⣽⡚⢮⣲⠆⠀⠀⠀⠀⠀⠀⢹⣿⣿⣿⣿
⣿⣿⣿⣿⣿⣿⡇⡼⠀⠀⠀⠀⠈⠻⣅⣨⠇⠈⠀⠰⣀⣀⣀⡀⠀⢸⣿⣿⣿⣿
⣿⣿⣿⣿⣿⣿⡇⠁⠀⠀⠀⠀⠀⠀⠀⠀⠀⠀⠀⣟⢷⣶⠶⣃⢀⣿⣿⣿⣿⣿
⣿⣿⣿⣿⣿⣿⡅⠀⠀⠀⠀⠀⠀⠀⠀⠀⠀⠀⠀⢿⠀⠈⠓⠚⢸⣿⣿⣿⣿⣿
⣿⣿⣿⣿⣿⣿⡇⠀⠀⠀⠀⢀⡠⠀⡄⣀⠀⠀⠀⢻⠀⠀⠀⣠⣿⣿⣿⣿⣿⣿
⣿⣿⣿⣿⣿⣿⡇⠀⠀⠀⠐⠉⠀⠀⠙⠉⠀⠠⡶⣸⠁⠀⣠⣿⣿⣿⣿⣿⣿⣿
⣿⣿⣿⣿⣿⣿⣿⣦⡆⠀⠐⠒⠢⢤⣀⡰⠁⠇⠈⠘⢶⣿⣿⣿⣿⣿⣿⣿⣿⣿
⣿⣿⣿⣿⣿⣿⣿⣿⡇⠀⠀⠀⠀⠠⣄⣉⣙⡉⠓⢀⣾⣿⣿⣿⣿⣿⣿⣿⣿⣿
⣿⣿⣿⣿⣿⣿⣿⣿⣷⣄⠀⠀⠀⠀⠀⠀⠀⠀⣰⣿⣿⣿⣿⣿⣿⣿⣿⣿⣿⣿
⣿⣿⣿⣿⣿⣿⣿⣿⣿⣿⣷⣤⣀⣀⠀⣀⣠⣾⣿⣿⣿⣿⣿⣿⣿⣿⣿⣿⣿⣿
⣿⣿⣿⣿⣿⣿⣿⣿⣿⣿⣿⣿⣿⣿⣿⣿⣿⣿⣿⣿⣿⣿⣿⣿⣿⣿⣿⣿⣿⣿
⣿⣿⣿⣿⣿⣿⣿⣿⣿⣿⠿⠟⠋⠉⠉⠉⠙⠛⠻⠿⣿⣿⣿⣿⣿⣿⣿⣿⣿⣿
⣿⣿⣿⣿⣿⣿⣿⣿⠟⠁⠀⠀⠀⠀⠀⣠⣤⣦⣤⡀⠈⠻⣿⣿⣿⣿⣿⣿⣿⣿
⣿⣿⣿⣿⣿⣿⣿⠏⠀⠀⠀⠀⣴⣶⣿⣿⣿⣿⣿⣿⣷⡄⠘⣿⣿⣿⣿⣿⣿⣿
⣿⣿⣿⣿⣿⣿⡿⠀⣼⠀⠀⢨⣿⣿⣿⣿⣿⣿⣿⣿⣿⣷⡘⢿⣿⣿⣿⣿⣿⣿
⣿⣿⣿⣿⣿⣿⣿⡏⡄⠄⠀⠘⣿⣿⣿⣿⣿⣿⣿⣿⣿⣿⣿⣝⣿⣿⣿⣿⣿⣿
⣿⣿⣿⣿⣿⡟⠺⢰⣦⡄⠀⠠⠙⢿⣿⣿⣿⣿⣿⣿⣿⣿⣿⣿⡽⣿⣿⣿⣿⣿
⣿⣿⣿⣿⣿⣧⠀⢸⡋⠀⠀⠀⠀⠀⢹⣯⡀⠀⠠⣈⣿⣿⣿⣿⣿⣿⣿⣿⣿⣿
⣿⣿⣿⣿⣿⣿⡇⢺⠇⠀⣬⣶⡇⠀⢸⣿⣿⣿⣿⣿⣿⣿⣿⣿⣿⣿⣿⣿⣿⣿
⣿⣿⣿⣿⣿⣿⣷⡈⠀⠀⠈⢻⠋⠀⢸⣿⣿⣿⣿⣿⣿⣿⣿⣿⣿⣿⣿⣿⣿⣿
⣿⣿⣿⣿⣿⣿⣿⣿⠀⠀⠀⠀⠀⠀⠘⢿⣻⣿⣿⣿⣿⣿⣿⣿⣿⣿⣿⣿⣿⣿
⣿⣿⣿⣿⣿⣿⣿⣿⡀⠀⠀⠀⠀⠀⠴⠸⠿⠿⢿⣿⣿⣿⣿⣿⣿⣿⣿⣿⣿⣿
⣿⣿⣿⣿⣿⣿⣿⣿⢳⡀⠀⠀⠀⠀⣈⣈⣽⣿⣶⣿⣟⣿⣿⣿⣿⣿⣿⣿⣿⣿
⣿⣿⣿⣿⣿⣿⠟⠁⠘⣷⣤⡀⠀⠀⠐⠙⠛⠛⣩⣵⣿⠻⣿⣿⣿⣿⣿⣿⣿⣿
⣿⡿⠿⠛⠉⠁⠀⠀⠀⠈⢿⣿⣶⣄⣤⣤⣐⣾⣿⣿⣿⡆⠈⠻⠿⣿⣿⣿⣿⣿
⠁⠀⠀⠀⠀⠀⠀⠀⠀⠀⠈⢿⣿⣿⣿⣿⣿⠿⠿⣿⣿⣧⠀⠀⠀⠀⠈⠉⠛⠿
⠀⠀⠀⠀⠀⠀⠀⠀⠀⠀⠀⠘⣿⣿⣿⡟⠁⠀⠀⠈⢹⣿⣇⠀⠀⠀⠀⠀⠀⠀
⠀⠀⠀⠀⠀⠀⠀⠀⠀⠀⠀⠀⠘⣿⣿⣷⣦⣀⠀⠀⣿⣿⣿⠀⠀⠀⠀⠀⠀⠀
⠀⠀⠀⠀⠀⠀⠀⠀⠀⠀⠀⠀⠀⢹⣿⣿⣿⡟⠀⠈⢹⣿⣿⡇⠀⠀⠀⠀⠀⠀
⠀⠀⠀⠀⠀⠀⠀⠀⠀⠀⠀⠀⠀⠀⢻⣿⣿⠃⠀⠀⠸⣿⣿⣷⠀⠀⠀⠀⠀⠀
⠀⠀⠀⠀⠀⠀⠀⠀⠀⠀⠀⠀⠀⠀⠈⢿⣿⠀⠀⠀⠀⢻⣿⣿⠀⠀⠀⠀⠀⠀
⠀⠀⠀⠀⠀⠀⠀⠀⠀⠀⠀⠀⠀⠀⠀⠈⣿⠀⠀⠀⠀⠈⢿⣿⡇⠀⠀⠀⠀⠀
⠄⠄⠄⠄⠄⠄⠄⠄⠄⠄⠄⠄⠄⠄⠄⠄⢀⠄⡀⠄⡀⢀⠄⡀⡀⠠⢀⠄⠄⠄⠄⠄⠄⠄⠄⠄⠄⠄⠄⠄⠄⠄⠄⠄⠄⠄
⠄⠄⠄⠄⠄⠄⠄⠄⠄⠄⠄⠄⠄⠄⠄⡀⠠⢀⠄⡅⢔⠰⡨⢢⢡⢑⠔⠅⠕⠄⠅⠄⠄⠄⠄⠄⠄⠄⠄⠄⠄⠄⠄⠄⠄⠄
⠄⠄⠄⠄⠄⠄⠄⠄⠄⠄⠄⠄⠄⡀⠅⢔⠨⢐⠅⡌⢆⢣⠪⡪⡘⡌⡮⡱⡡⣊⢌⢀⢀⠄⠄⠄⠄⠄⠄⠄⠄⠄⠄⠄⠄⠄
⠄⠄⠄⠄⠄⠄⠄⠄⠄⠄⠄⠠⡐⢄⠕⡡⡘⢔⢱⢨⢪⢪⢪⢪⡺⣪⢞⢮⢫⢮⢺⢔⢆⡢⠄⠄⠄⠄⠄⠄⠄⠄⠄⠄⠄⠄
⠄⠄⠄⠄⠄⢀⠐⠈⠄⡠⡊⢌⠢⡑⡌⡆⡇⡇⡇⡇⣇⢗⣝⢮⡺⣪⡳⣹⡪⣞⢵⢝⡵⡝⣕⠡⢀⢂⠄⠄⠄⠄⠄⠄⠄⠄
⠄⠄⠄⠄⢠⢊⢂⠁⡔⡌⡢⢑⠌⡢⡱⡸⡸⡸⡪⡺⡸⣕⢵⡳⣝⢮⡺⣪⢞⣵⣫⡳⣕⢯⡺⡬⠄⠕⡕⡄⠄⠄⠄⠄⠄⠄
⠄⠄⠄⠄⡇⡇⡂⢢⢣⢣⠨⡂⡊⢔⢕⢱⡱⡝⣜⢝⡺⣜⡵⣝⢮⡳⡽⣵⣻⡺⣼⡺⡵⣳⢝⡮⡃⡕⡺⢄⠄⠄⠄⠄⠄⠄
⠄⠄⠄⢸⡱⡱⡨⢪⢪⢢⠡⢂⢊⠢⡱⡱⡱⡝⡮⡳⣝⢞⢮⢗⡯⣯⢯⣗⡯⣟⣮⢯⣟⢮⡳⣝⢇⢇⢎⢵⡀⠄⠄⠄⠄⠄
⠄⠄⠄⣕⢵⢣⢣⢑⢅⢆⢃⠢⠡⡡⡱⡱⡹⡪⡯⣺⢕⡯⣫⢯⢾⢽⢽⣺⢯⣗⡯⣗⣗⢗⣝⢮⡫⣏⢮⢒⢅⠄⠄⠄⠄⠄
⠄⠄⢀⠮⡪⢪⠪⠢⡃⡪⡂⠅⢕⠰⢱⢸⢸⡱⣝⢮⡳⡽⣪⢏⡯⡯⣟⡾⣽⣺⢽⣳⡳⣝⢮⡳⣫⣳⢳⡱⡱⠄⠄⠄⠄⠄
⠄⠄⢀⡃⡫⢔⢨⢕⢕⢔⢌⢌⠢⡑⡱⡘⢜⢜⢮⡳⣝⣞⢵⣫⢿⢽⣺⡽⡾⣾⢽⣳⢯⢮⡳⣝⣕⢗⢵⡑⡕⡀⠄⠄⠄⠄
⠄⠄⠐⠢⡱⡐⢕⢵⣑⡑⢕⢐⠕⠸⠰⡑⠕⢕⢕⠽⡕⣏⢗⢽⢝⢽⠺⠽⠽⢽⢻⢽⢽⢵⢽⡸⣪⢳⢱⢱⡱⡽⡡⡀⠄⠄
⠄⠄⠨⠈⢆⢊⢎⢖⢂⠃⠡⠐⠈⠈⠄⠄⠄⠄⠄⢑⢹⢘⣜⢕⠑⡈⠄⡈⣈⣐⠨⠘⡜⢕⡳⣝⢮⡳⡱⡱⡵⡯⡪⠂⠄⠄
⠄⠄⠄⠅⢂⠑⡕⡕⠄⡢⢀⠄⠠⠄⠁⠁⡌⠠⢠⠄⠸⣸⣺⡪⣐⠅⢢⢈⠄⢬⢍⣆⢧⣳⡽⡮⣺⡪⡺⣘⢼⣝⠆⠄⠄⠄
⠄⠄⠄⠨⡢⡢⡣⡳⢑⢰⢐⠠⡊⡪⣢⡣⡪⡰⠑⡀⠨⣪⢷⣝⡮⡫⡪⣪⢾⢝⣯⢾⣝⣗⡯⣟⢮⡪⡯⡷⣗⡯⠄⠄⠄⠄
⠄⠄⠄⠄⢕⠄⡇⡣⡑⠔⢅⠇⡇⡏⡖⡕⡕⣌⢂⠂⠌⢮⢷⡳⡯⡯⡾⡵⣫⢯⢞⣗⡷⡯⣯⡳⣱⢱⢝⡽⣺⠁⠄⠄⠄⠄
⠄⠄⠄⠄⠡⡡⢱⢐⠨⠨⠐⡡⢃⢇⢇⡏⡞⡔⡐⠠⢑⢽⢽⢽⢽⢽⢽⣝⢗⡯⣟⢾⢽⢝⡞⣜⢜⢜⣽⣺⠎⠄⠄⠄⠄⠄
⠄⠄⠄⠄⠄⠊⠃⠂⡐⠄⠅⡐⡈⡢⡃⢇⠕⡢⠄⠨⢘⢮⢯⢯⣳⡫⡗⣗⢽⢝⡮⡯⡮⡣⡏⣎⢎⠞⠺⠊⠄⠄⠄⠄⠄⠄
⠄⠄⠄⠄⠄⠄⠄⠄⠂⡁⢂⠐⡈⡢⢑⠅⡕⡐⠈⠨⡨⢯⣻⢽⣺⢺⡺⡪⣏⢷⢽⢕⣯⡫⣞⢜⢜⠄⠄⠄⠄⠄⠄⠄⠄⠄
⠄⠄⠄⠄⠄⠄⠄⠄⡁⠄⠄⠂⡂⠌⠄⢕⢐⠌⢄⠄⠨⠨⢘⢥⢅⡵⣝⣝⢮⡳⡽⣝⡮⣺⢪⢎⣗⠄⠄⠄⠄⠄⠄⠄⠄⠄
⠄⠄⠄⠄⠄⠄⠄⠄⠄⢂⢀⠡⠄⠌⠌⢂⢂⠪⠐⡌⢰⢰⢘⢼⢝⣞⢞⣞⡵⡯⣞⢵⣫⢮⡳⡣⡳⠄⠄⠄⠄⠄⠄⠄⠄⠄
⠄⠄⠄⠄⠄⠄⠄⠄⠈⡀⠐⠄⠊⠄⠨⠐⠠⠈⠌⠘⠜⠵⡢⢳⢹⢜⢕⠕⠱⡹⡪⡳⣕⣳⢹⡸⠅⠄⠄⠄⠄⠄⠄⠄⠄⠄
⠄⠄⠄⠄⠄⠄⠄⠄⠄⠄⠈⢄⠡⠁⠠⠄⠄⡀⠄⠐⠔⠔⠔⡢⠦⣒⢎⣎⢦⢢⢩⡫⣎⢮⢣⠃⠄⠄⠄⠄⠄⠄⠄⠄⠄⠄
⠄⠄⠄⠄⠄⠄⠄⠄⢰⣇⠄⠐⠠⠄⠁⠄⠡⢀⢂⢅⢔⢄⣆⣆⢕⢵⡹⣪⡳⣕⢵⢱⢕⢇⡇⠄⠄⠄⠄⠄⠄⠄⠄⠄⠄⠄
⠄⠄⠄⠄⠄⠄⠄⠄⠱⡿⣵⣀⠄⠄⠄⠈⠨⠨⡂⢇⢎⢕⢎⢞⡽⡵⣝⢮⢺⢸⠸⡘⣜⢼⠄⠄⠄⠄⠄⠄⠄⠄⠄⠄⠄⠄
⠄⠄⠄⠄⠄⠄⠄⠄⠄⢹⣺⣳⣧⣄⡀⠄⠄⡀⠂⠌⡂⠕⢌⠊⢎⠪⠪⠪⡊⡆⣕⣕⢧⣓⣧⠄⠄⠄⠄⠄⠄⠄⠄⠄⠄⠄
⠄⠄⠄⠄⠄⠄⠄⠄⠄⠈⢪⢞⣾⡽⣷⣮⡄⠐⠄⡂⠌⢌⠢⢑⢐⢄⢅⢇⢇⣗⣕⢗⡵⣽⣿⠄⠄⠄⠄⠄⠄⠄⠄⠄⠄⠄
⠄⠄⠄⠄⠄⠄⠄⠄⠄⠄⠄⠙⡾⣽⣯⣷⣿⡿⣮⣰⢱⡱⣜⢔⡑⡜⣜⢮⣣⡳⣪⢷⣽⣿⣿⡀⠄⠄⠄⠄⠄⠄⠄⠄⠄⠄
⠄⠄⠄⠄⠄⠄⠄⠄⠄⠄⠄⠄⠘⡷⣿⣽⣿⣿⣿⣿⣿⣾⣮⣗⣽⢵⢕⣕⢮⡺⢝⣵⣿⣿⣿⣷⠄⠄⠄⠄⠄⠄⠄⠄⠄⠄
⠄⠄⠄⠄⠄⠄⠄⠄⠄⠄⠄⠄⠄⠘⣯⡿⣿⣿⣿⣿⣿⣿⣿⣿⣿⣿⣷⣵⣏⣾⣿⣿⣿⣿⣿⣿⣇⠄⠄⠄⠄⠄⠄⠄⠄⠄
⠄⠄⠄⠄⠄⠄⠄⠄⠄⠄⠄⠄⠄⠄⠻⣟⣿⣿⣿⣿⣿⣿⣿⣿⣿⠏⠉⠄⠄⠄⠉⠻⢿⣿⣿⣿⣿⡀⠄⠄⠄⠄⠄⠄⠄⠄
⠄⠄⠄⠄⠄⠄⠄⠄⠄⠄⠄⠄⠄⠄⠄⢻⣿⣿⣿⣿⣿⣿⣿⡟⠁⠄⠄⠁⠠⠄⠄⠄⠈⠽⣿⣿⣿⣷⠄⠄⠄⠄⠄⠄⠄⠄
⠄⠄⠄⠄⠄⠄⠄⠄⠄⠄⠄⠄⠄⠄⠄⠈⢿⣿⣿⣿⣿⣿⡵⣄⠄⠄⠄⠄⠁⠈⠄⠄⢸⣹⡪⣿⣿⣿⡂⠄⠄⠄⠄⠄⠄⠄
⠄⠄⠄⠄⠄⠄⠄⠄⠄⠄⠄⠄⠄⠄⠄⠄⠈⢿⣿⣿⣻⣿⣿⣯⣿⣢⣄⠄⠄⠄⠑⠠⢸⣿⣿⣮⢞⣿⣧⠄⠄⠄⠄⠄⠄⠄
\fi