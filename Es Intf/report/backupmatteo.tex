\documentclass[10pt, a4paper, italian]{article}
\usepackage[T1]{fontenc}
\usepackage[utf8]{inputenc}
\usepackage{amsmath, amssymb, amsthm, thmtools, amsfonts, mathtools}
\usepackage{nicefrac}
\usepackage{calc}
\usepackage[pdftex, hyperindex, plainpages=false]{hyperref}
\usepackage[nameinlink]{cleveref} %load before classicthesis (clash)
%\usepackage[nochapters,pdfspacing]{classicthesis}
\usepackage{siunitx}
\usepackage[siunitx]{circuitikz}

\usepackage[a4paper]{geometry}
\usepackage{float}
\usepackage{mdframed}
\usepackage{titling}
\usepackage{booktabs}
\usepackage{graphicx}
\usepackage{caption, subcaption}
\usepackage{xcolor}
\usepackage[italian]{babel}
\usepackage{pgfplots}
\usepackage{listings}
%\usepackage{lmodern}
\usepackage{url}
\usepackage{enumitem}
\usepackage{tikz} %loads after classicthesis (xcolor incompat)

% lets graphicx know path where figures to be included are found
\graphicspath{{../figs/}}
\makeatletter
\def\input@path{{../figs/}}
%or: \def\input@path{{/path/to/folder/}{/path/to/other/folder/}}
\makeatother

% tikz pgf plots setup
\usepgfplotslibrary{external}
\pgfplotsset{compat=1.15}
\tikzexternalize

% spaces and significant digits/figures for measurements
\sisetup{free-standing-units, space-before-unit, number-unit-product = \;,
scientific-notation = true, round-mode = figures, round-precision = 2,}

% turns all (hyperlinked) references black [default is blue]
\hypersetup{
	linktoc=all,
	colorlinks=true,
	linkcolor=black
}

% code listings config
\lstset{
language=Python,
basicstyle=\ttfamily,
columns=fullflexible,
keepspaces=true,
}

% mdframed (for boxed text) configuration
\mdfsetup{linewidth=0.6pt}

% Default fixed font does not support bold face
\DeclareFixedFont{\ttb}{T1}{txtt}{bx}{n}{12} % for bold
\DeclareFixedFont{\ttm}{T1}{txtt}{m}{n}{12}  % for normal

% Custom colors
\usepackage{color}
\definecolor{deepblue}{rgb}{0,0,0.5}
\definecolor{deepred}{rgb}{0.6,0,0}
\definecolor{deepgreen}{rgb}{0,0.5,0}

% Commands 
\newcommand{\executeiffilenewer}[3]{%
	\ifnum\pdfstrcmp{\pdffilemoddate{#1}}%
		{\pdffilemoddate{#2}}>0%
	{\immediate\write18{#3}}\fi%
}
% input .svg --> .pdf_tex graphs
\newcommand{\includesvg}[1]{%
	\executeiffilenewer{#1.svg}{#1.pdf}%
	{inkscape -z -D --file=#1.svg %
	--export-pdf=#1.pdf --export-latex}%
	\input{#1.pdf_tex}%
}
% Thanks UniPi's Department of Physics E. Fermi
\newcommand{\thanksdf}{(\thanks{Dipartimento di Fisica E.~Fermi,%
Universit\`a di Pisa - Pisa, Italy.}\;)}

% hyperlink to email address
\newcommand{\mail}[1]{\href{mailto:#1}{\textsf{#1}}}

\input{../../latex/math}
\geometry{left=2cm, right=2cm, top=2cm, bottom=2cm}

% indexes subsections with letters, sections with numbers (1.a, 1.b, ...)
\renewcommand{\thesubsection}{\thesection.\alph{subsection}}

% lets graphicx know path where figures to be included are found
\graphicspath{{../figs/}}

\author{Gruppo 1.AC \\ Matteo Rossi, Bernardo Tomelleri}
\title{Es11: Esperimenti di Interferometria per misure di lunghezze d'onda}
\begin{document}
\date{\today}
\maketitle

\setcounter{section}{0}

\section*{Nota sul metodo di fit}
Per determinare i parametri ottimali e le rispettive covarianze si \`e
implementato in \verb+Python+ un algoritmo di fit basato sui minimi quadrati
mediante la funzione \emph{curve\_fit} della libreria \texttt{SciPy}.

%=======================
\section{Misura della lunghezza d'onda di un diodo laser attraverso un pattern di diffrazione}
Dalla teoria sulla natura ondulatoria della luce sappiamo che quando un'onda incide su un reticolo di diffrazione viene diffratta in diversi fasci: il fascio di luce che non subisce deviazioni e che viene trasmesso direttamente viene chiamato ordine 0, che può essere individuato sullo schermo come il punto di massima luminosità. Se poi andiamo a calcolare l'angolo di diffrazione degli altri fasci di luce deviati rispetto a un punto di riferimento, e considerando che i raggi vengono anche riflessi dal nostro reticolo, possiamo stabilire un relazione che collega la posizione dei massimi di rifrazione con la lunghezza d'onda $\lambda$, gli angoli degli ordini di diffrazione $\theta _m$, il passo del reticolo $d$ e l'angolo di incidenza sul reticolo $\theta _i$
\begin{equation}
d(\sin(\theta _i) - \sin(\theta _m))=m \lambda
\label{eq:diff}
\end{equation}
\begin{figure}
\includegraphics[width=\textwidth]{0}
\caption{ \label{schema1} Schema di riferimento dell'apparato sperimentale utilizzato}
\end{figure}
Per determinare gli angoli $\theta _m$ dei vari fasci è sufficiente utilizzare funzioni goniometriche per arrivare alla conclusione che
\begin{equation}
\theta _m=\pi /2 - \arcsin(\frac{h_m}{D})
\end{equation}
dove $h_m$ sono le altezze relative dei massimi di diffrazione rispetto al punto di riferimento (nel nostro caso sarà l'altezza a cui si trova il calibro).
Possiamo infine riscrivere l'equazione \ref{eq:diff} come:
\begin{equation}
Y= -X \frac{\lambda}{d} +Q
\label{eq:fit}
\end{equation}
dove Y,X e Q sono rispettivamente $\sin(\theta _d)$, m e $\sin(\theta _i)$; da cui si riconosce subito l'equazione di una retta, con coefficiente angolare $\frac{\lambda}{d}$.
\subsection{Apparato}
Come reticolo di diffrazione abbiamo utilizzato la superficie riflettente di un calibro (quindi con passo $d=1 \; mm$).
Utilizzando lo schema in figura \ref{schema1} abbiamo cominciato misurando la distanza tra lo schermo e il punto d'incidenza del fascio luminoso sul calibro:
\[
D=2.90 \pm 0.03 m
\]
il valore dell'incertezza deriva dal fatto che il punto preciso in cui il fascio laser incide sul calibro è difficile da stimare, perché invece di un punto si osserva una regione luminosa lunga circa 6 cm (d'altronde $\theta _i$ è molto vicino a $\pi /2$), di conseguenza abbiamo preso come errore la metà della sua lunghezza.
Successivamente abbiamo spostato il calibro in modo che solo una porzione del fascio incidesse sulla scala del calibro, mentre il restante continuasse la sua traiettoria rettilinea senza essere riflesso.
Abbiamo ottenuto così un'immagine da cui si potesse stimare il piano di riferimento da cui far partire le misure di $h_m$ prendendo il punto medio tra il punto dove incideva il fascio riflesso, e il punto in cui incideva il fascio "diretto"; da qui abbiamo stimato il valore di $h_0$:
\[
h_0=5.7 \pm 0.1 cm
\]
per concludere le misure abbiamo registrato le altezze di 25 massimi di diffrazione rispetto al piano di riferimento, ognuno con la relativa incertezza (derivata dallo spessore non trascurabile dello spot luminoso).

%=======================
\section{Misura della lunghezza d'onda di una lampada al mercurio}
\subsection{Interferometro di Michelson}
\subsection{Calibrazione apparato tramite l'uso di un laser He-Ne}
\subsection{Stima della lunghezza d'onda}

%=======================
\section*{Conclusioni e commenti finali}
%TODO scrivi le conclusioni 

%=======================
\section*{Dichiarazione}
I firmatari di questa relazione dichiarano che il contenuto della relazione \`e
originale, con misure effettuate dai membri del gruppo, e che tutti i firmatari
hanno contribuito alla elaborazione della relazione stessa.


\end{document}
