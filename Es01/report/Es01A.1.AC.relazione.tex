\documentclass[10pt,a4paper]{article}
\usepackage[utf8]{inputenc}
\usepackage[italian]{babel}
\usepackage{amsmath}
\usepackage{amsfonts}
\usepackage{amssymb}
\usepackage{graphicx}
\usepackage[left=2cm,right=2cm,top=2cm,bottom=2cm]{geometry}
\newcommand{\rem}[1]{[\emph{#1}]}
\newcommand{\exn}{\phantom{xxx}}

\author{Gruppo 1.Ay \\ Mario Rossi, Anna Bianchi \rem{non dimenticate i nomi}}
\title{Es01A: Uso dello strumento Analog Discovery 2.}
\begin{document}
\date{23 ottembre 2150}
\maketitle

\setcounter{section}{1}

\section{Utilizzo del canale di alimentazione e del multimetro}

\subsection*{2.d Accensione diodo}

La tensione di alimentazione \`e stata variata nell intervallo tra $xx\,\mathrm{V}$ e $yy\,\mathrm{V}$


\vspace{0.5cm}
\framebox(400,30){Inserire commento sulla luminosit\`a del diodo in funzione della tenione e per diversi colori.}

\subsection*{2.e Misura tensione}
Utilizzando il multimetro si misura la tensione ai capi del diodo e si ottiene:

\begin{table}[h]
\centering
\begin{tabular}{|c|c|c|c|c|c|}
\hline 
V+& $\sigma$ V+  & VD & $\sigma$ VD & I(R1)  & $\sigma$ I(R1) \\
\hline 
\exn & \exn & \exn & \exn & \exn &\exn \\
\exn & \exn & \exn & \exn & \exn &\exn \\
\exn & \exn & \exn & \exn & \exn &\exn \\
\exn & \exn & \exn & \exn & \exn &\exn \\
\hline 
\end{tabular} 
\caption{(2.e) Tensione e corrente ai capi del diodo. Tutte le tensioni in V.\label{t:par1}}
\end{table}

%=======================
\section{Uso generatore di forme d'onda}
\exn 
\par
\vspace{0.5cm}
\framebox(400,30){Inserire commento sulle onde generate, ed eventualmente screenshot }

\section{Oscilloscopio}

\subsection*{4.e Uso del trigger}

\exn 
\par
\vspace{0.5cm}
\framebox(400,30){Inserire commento sulle prove effettuate }

\begin{figure}[h]
\centering
%\includegraphics[scale=0.4]{part1.pdf}
\framebox(400,50){ (4.e) Inserire lo screenshot dell'oscilloscopio. }
\caption{(4.e) Relazione tra trigger e segnale}
\end{figure}


\subsection*{4.f Misura tensione massima ai capi del diodo}
\par 
La tensione massima ai capi del diodo misurata con i cursori risulta essere $V_{\mathrm{MAX}}= ( \exn \pm \exn ) \,\mathrm{V}$. La funzione di misura automatica fornisce il valore $V_{\mathrm{AUTO}}= xx \,\mathrm{V}$

\vspace{0.5cm} 
\framebox(400,30){Inserire commento sulla accuratezza della misura.}



\section{Caratteristica del diodo}
\par

\subsection*{5.c Caratteristica del diodo}

\begin{figure}[h]
\centering
%\includegraphics[scale=0.4]{part1.pdf}
\framebox(400,50){ (5.c) Inserire lo screenshot dell'oscilloscopio in XY. }
\caption{(5.c) Caratteristica corrente-tensione del diodo in modalit\`a XY}
\end{figure}

\subsection*{5.d Fit curva del diodo}
\par

\begin{figure}[h]
\centering
%\includegraphics[scale=0.4]{part1.pdf}
\framebox(200,100){(2.b) Inserire il grafico $I_{D}$ vs. $V_{D}$ }
\caption{(2.b) Grafico $I_{D}$ vs. $V_{D}$ e fit all'equazione di Schockley}
\end{figure}



%=====================




\section{Partitore}

\subsection*{6.b Partitore con resistenze da 1k}


Si realizza un partitore con resistenze da $1 \,\mathrm{k}\Omega$. Valori misurati con il multimetro: R1=$\exn \pm \exn \,\mathrm{k}\Omega$, R2=$\exn \pm \exn \,\mathrm{k}\Omega$


\begin{table}[h]
\centering
\begin{tabular}{|c|c|c|c|c|c|}
\hline 
VIN& $\sigma$ VIN  &VOUT	 & $\sigma$ VOUT& VOUT/VIN & $\sigma$ VOUT/VIN \\
\hline 
\exn & \exn & \exn & \exn & \exn &\exn \\
\exn & \exn & \exn & \exn & \exn &\exn \\
\exn & \exn & \exn & \exn & \exn &\exn \\
\exn & \exn & \exn & \exn & \exn &\exn \\
\hline 
\end{tabular} 
\caption{(6.b) Partitore di tensione con resistenze da circa 1k. Tutte le tensioni in V.\label{t:par1}}
\end{table}


\framebox(400,30){Inserire commento sul confronto tra valori misurati ed attesi.}


\subsection*{6.d Partitore con resistenze da circa 1M}
\par
Si realizza un partitore con resistenze da $1 \,\mathrm{M}\Omega$. Valori misurati con il multimetro: R1=$\exn \pm \exn \,\mathrm{M}\Omega$, R2=$\exn \pm \exn \,\mathrm{M}\Omega$


\begin{table}[h]
\centering
\begin{tabular}{|c|c|c|c|c|c|}
\hline 
VIN& $\sigma$ VIN  &VOUT	 & $\sigma$ VOUT& VOUT/VIN & $\sigma$ VOUT/VIN \\
\hline 
\exn & \exn & \exn & \exn & \exn &\exn \\
\exn & \exn & \exn & \exn & \exn &\exn \\
\exn & \exn & \exn & \exn & \exn &\exn \\
\exn & \exn & \exn & \exn & \exn &\exn \\
\hline 
\end{tabular} 
\caption{(6.d) Partitore di tensione con resistenze da circa 1M. Tutte le tensioni in V.\label{t:par2}}
\end{table}


\framebox(400,30){Inserire commento sul confronto tra valori misurati ed attesi.}



\subsection*{6.e Resistenza di ingresso del multimetro}
Usando il modello mostrato nella scheda si ottiene
\[ \frac{R_1}{R_T} =  \frac{V_{IN}}{V_{OUT}} - (1 +  \frac{R_1}{R_2} )
\]

Con i dati con resistenze da 1k si ottiene
\[ R_1/R_{IN} = \exn  \pm  \exn   \rightarrow  R_{IN} > \exn k\Omega
\]


Con i dati con resistenze da 1M si ottiene
\[ R_1/R_{IN} = \exn  \pm  \exn   \rightarrow  R_{IN} = (\exn \pm  \exn)  M\Omega
\]

\framebox(400,30){Inserire commento sulla sensibilit\`a sperimentale della misura.} 




\section{Misure di tempo e frequenza}

\subsection*{7.e Misure di frequenza}
Misure con onda sinusoidale
\begin{table}[h]
\centering
\begin{tabular}{|c|c|c|c|c|c|}
\hline 
Periodo T (s)& $\sigma$ T (s)  &Frequenza f (Hz) & $\sigma$ f (Hz) & Misura oscilloscopio (Hz) & Differenza (Hz)\\
\hline 
\exn & \exn & \exn & \exn & \exn &\exn \\
\exn & \exn & \exn & \exn & \exn &\exn \\
\exn & \exn & \exn & \exn & \exn &\exn \\
\exn & \exn & \exn & \exn & \exn &\exn \\
\hline 
\end{tabular} 
\caption{(7.e) Misura di frequenza di onde sinusoidali  e confronto con misurazione interna dell'oscilloscopio }
\end{table}

\subsection*{7.f Misure di duty cyle}
Misure con onda quadra
\begin{table}[h]
\centering
\begin{tabular}{|c|c|c|c|c|c|}
\hline 
Periodo T (s)& $\sigma$ T (s) & Durata alto $t_H$ (s) & $\sigma$ $t_H$ (s) & Duty cycle D(\%) & $\sigma$ D (\%) \\
\hline 
\exn & \exn & \exn & \exn & \exn &\exn \\
\exn & \exn & \exn & \exn & \exn &\exn \\
\exn & \exn & \exn & \exn & \exn &\exn \\
\exn & \exn & \exn & \exn & \exn &\exn \\
\hline 
\end{tabular} 
\caption{(7.f) Misura di duty cycle per onde quadre }
\end{table}


\subsection*{7.g Tempo di salita e di discesa}
Misure su onda quadra
\[
f = (\exn \pm \exn) \mathrm{MHz}, \quad
t_\mathrm{salita} = (\exn \pm \exn) \mathrm{\mu s},
t_\mathrm{discesa} = (\exn \pm \exn) \mathrm{\mu s},
\]

\framebox(400,30){Inserire commento su altre caratteristiche del segnale ed eventualmente uno screenshot}

\section{Conclusioni e commenti finali}
\framebox(400,30){Inserire eventuali commenti e conclusioni finali}

\section*{Dichiarazione}
I firmatari di questa relazione dichiarano che il contenuto della relazione \`e originale, con misure effettuate dai membri del gruppo, e che tutti i firmatari hanno contribuito alla elaborazione della relazione stessa.

\end{document}