\documentclass[10pt,a4paper]{article}
\usepackage[T1]{fontenc}
\usepackage[utf8]{inputenc}
\usepackage{amsmath, amssymb, amsthm, thmtools, amsfonts, mathtools}
\usepackage{nicefrac}
\usepackage{calc}
\usepackage[pdftex, hyperindex, plainpages=false]{hyperref}
\usepackage[nameinlink]{cleveref} %load before classicthesis (clash)
%\usepackage[nochapters,pdfspacing]{classicthesis}
\usepackage{siunitx}
\usepackage[siunitx]{circuitikz}

\usepackage[a4paper]{geometry}
\usepackage{float}
\usepackage{mdframed}
\usepackage{titling}
\usepackage{booktabs}
\usepackage{graphicx}
\usepackage{caption, subcaption}
\usepackage{xcolor}
\usepackage[italian]{babel}
\usepackage{pgfplots}
\usepackage{listings}
%\usepackage{lmodern}
\usepackage{url}
\usepackage{enumitem}
\usepackage{tikz} %loads after classicthesis (xcolor incompat)

% lets graphicx know path where figures to be included are found
\graphicspath{{../figs/}}
\makeatletter
\def\input@path{{../figs/}}
%or: \def\input@path{{/path/to/folder/}{/path/to/other/folder/}}
\makeatother

% tikz pgf plots setup
\usepgfplotslibrary{external}
\pgfplotsset{compat=1.15}
\tikzexternalize

% spaces and significant digits/figures for measurements
\sisetup{free-standing-units, space-before-unit, number-unit-product = \;,
scientific-notation = true, round-mode = figures, round-precision = 2,}

% turns all (hyperlinked) references black [default is blue]
\hypersetup{
	linktoc=all,
	colorlinks=true,
	linkcolor=black
}

% code listings config
\lstset{
language=Python,
basicstyle=\ttfamily,
columns=fullflexible,
keepspaces=true,
}

% mdframed (for boxed text) configuration
\mdfsetup{linewidth=0.6pt}

% Default fixed font does not support bold face
\DeclareFixedFont{\ttb}{T1}{txtt}{bx}{n}{12} % for bold
\DeclareFixedFont{\ttm}{T1}{txtt}{m}{n}{12}  % for normal

% Custom colors
\usepackage{color}
\definecolor{deepblue}{rgb}{0,0,0.5}
\definecolor{deepred}{rgb}{0.6,0,0}
\definecolor{deepgreen}{rgb}{0,0.5,0}

% Commands 
\newcommand{\executeiffilenewer}[3]{%
	\ifnum\pdfstrcmp{\pdffilemoddate{#1}}%
		{\pdffilemoddate{#2}}>0%
	{\immediate\write18{#3}}\fi%
}
% input .svg --> .pdf_tex graphs
\newcommand{\includesvg}[1]{%
	\executeiffilenewer{#1.svg}{#1.pdf}%
	{inkscape -z -D --file=#1.svg %
	--export-pdf=#1.pdf --export-latex}%
	\input{#1.pdf_tex}%
}
% Thanks UniPi's Department of Physics E. Fermi
\newcommand{\thanksdf}{(\thanks{Dipartimento di Fisica E.~Fermi,%
Universit\`a di Pisa - Pisa, Italy.}\;)}

% hyperlink to email address
\newcommand{\mail}[1]{\href{mailto:#1}{\textsf{#1}}}

\geometry{left=2cm, right=2cm, top=2cm, bottom=2cm}
\newcommand{\rem}[1]{[\emph{#1}]}
\newcommand{\exn}{\phantom{xxx}}

% lets graphicx know path where figures to be included are found
\graphicspath{{../figs/}}

\author{Gruppo 1.AC \\ Matteo Rossi, Bernardo Tomelleri}
\title{Es01A: Uso dello strumento Analog Discovery 2.}
\begin{document}
\date{\today}
\maketitle

\setcounter{section}{1}

\section{Utilizzo del canale di alimentazione e del multimetro}

\subsection*{2.d Accensione diodo}

La tensione di alimentazione \`e stata variata nell'intervallo tra
$0.5\,\mathrm{V}$ e $5\,\mathrm{V}$


Si osserva che la luminosit\`a del diodo è proporzionale alla tensione
erogata dal generatore, una volta superata una tensione di soglia per cui
il LED inizia a emettere luce di intensità osservabile. La tensione di soglia
varia per i diversi colori; in particolare $V_{\mathrm{thr}}$ risulta
proporzionale alla frequenza del colore di luce emessa. Dunque
rosso $<$ giallo $<$ verde $<$ blu.

\subsection*{2.e Misura tensione}
Utilizzando il multimetro si misura la tensione ai capi del diodo e si ottiene:

\begin{table}[h]
\centering
\begin{tabular}{|c|c|c|c|c|c|}
\hline 
V+& $\sigma$ V+  & VD & $\sigma$ VD & I(R1)  & $\sigma$ I(R1) \\
\hline 
\exn & \exn & \exn & \exn & \exn &\exn \\
\exn & \exn & \exn & \exn & \exn &\exn \\
\exn & \exn & \exn & \exn & \exn &\exn \\
\exn & \exn & \exn & \exn & \exn &\exn \\
\hline 
\end{tabular} 
\caption{(2.e) Tensione e corrente ai capi del diodo.
Tutte le tensioni in V.\label{t:par1}}
\end{table}

%=======================
\section{Uso generatore di forme d'onda}
\exn 
\par
\vspace{0.5cm}
\framebox(400,30){Inserire commento sulle onde generate, ed eventualmente
screenshot.}
\begin{figure}[ht]
\centering
\includegraphics[scale=0.3]{sqwdiode}
\caption{(3.b) Onda quadra in ingresso $f \approx 10 \si{\Hz}$ al diodo}
\end{figure}
\section{Oscilloscopio}

\subsection*{4.e Uso del trigger}

\exn 
\par
\vspace{0.5cm}
\framebox(400,30){Inserire commento sulle prove effettuate }

\begin{figure}[ht]
\centering
\includegraphics[scale=0.3]{trgdiode_new}
\caption{(4.e) Relazione tra trigger e segnale}
\end{figure}


\subsection*{4.f Misura tensione massima ai capi del diodo}
\par 
La tensione massima ai capi del diodo misurata con i cursori risulta essere
$V_{\mathrm{MAX}}= ( 1.9 \pm 0.1 ) \,\mathrm{V}$. La funzione di misura
automatica fornisce il valore $V_{\mathrm{AUTO}}= 1.975 \,\mathrm{V}$

\vspace{0.5cm} 
\framebox(400,30){Inserire commento sulla accuratezza della misura.}



\section{Caratteristica del diodo}
\par

\subsection*{5.c Caratteristica del diodo}

\begin{figure}[h]
\centering
\includegraphics[scale=0.4]{shockley_new}
\caption{(5.c) Caratteristica corrente-tensione del diodo in modalit\`a XY}
\end{figure}

\subsection*{5.d Fit curva del diodo}
\par

\begin{figure}[h]
\centering
\includegraphics[scale=0.8]{ivfit}
\caption{(2.b) Grafico $I_{D}$ vs. $V_{D}$ e fit all'equazione di Schockley}
\end{figure}



%=====================




\section{Partitore}

\subsection*{6.b Partitore con resistenze da 1k}


Si realizza un partitore con resistenze da $1 \,\mathrm{k}\Omega$.
Valori misurati con il multimetro: R1=$993 \pm 8 \,\Omega$,
R2=$993 \pm 8 \,\Omega$


\begin{table}[h]
\centering
\begin{tabular}{|c|c|c|c|c|c|}
\hline 
VIN& $\sigma$ VIN  & VOUT	 & $\sigma$ VOUT & VOUT/VIN & $\sigma$ VOUT/VIN \\
\hline 
1.000 & 0.005 & 0.500 & 0.003 & 0.500 & 0.008 \\
2.00 & 0.02 & 1.000 & 0.005 & 0.500 & 0.011 \\
3.00 & 0.02 & 1.500 & 0.008 & 0.500 &0.008 \\
4.00 & 0.03 & 2.00 & 0.02 & 0.500 & 0.012 \\
\hline 
\end{tabular} 
\caption{(6.b) Partitore di tensione con resistenze da circa 1k. Tutte le
tensioni in V.\label{t:par1}}
\end{table}

I valori di attenuazione attesi per il partitore risultano compatibili
con quelli misurati per tutti i valori di tensione compresi nell'intervallo
esplorato ($1 - 4$ V.)

\subsection*{6.d Partitore con resistenze da circa 1M}
\par
Si realizza un partitore con resistenze da $1 \,\mathrm{M}\Omega$.
Valori misurati con il multimetro: R1=$993 \pm 8 \,\mathrm{k}\Omega$,
R2=$996 \pm 0.008 \,\mathrm{k}\Omega$


\begin{table}[h]
\centering
\begin{tabular}{|c|c|c|c|c|c|}
\hline 
VIN& $\sigma$ VIN  &VOUT	 & $\sigma$ VOUT& VOUT/VIN & $\sigma$ VOUT/VIN \\
\hline 
1.000 & 0.005 & 0.481 & 0.003 & 0.481 & 0.008 \\
2.00 & 0.02 & 0.955 & 0.005 & 0.478 & 0.011 \\
3.00 & 0.02 & 1.431 & 0.007 & 0.477 & 0.008 \\
4.00 & 0.03 & 1.906 & 0.009 & 0.477 & 0.009 \\
\hline 
\end{tabular} 
\caption{(6.d) Partitore di tensione con resistenze da circa 1M.
Tutte le tensioni in V.\label{t:par2}}
\end{table}

La tensione in uscita dal partitore $R_1 + R_2$ risulta apprezzabilmente
inferiore rispetto al suo valore atteso. Questo è dovuto al comportamento
non ideale del voltmetro, per cui quando la sua impedenza in ingresso
$10 \si{\Mohm}$ (nom.) è paragonabile a quella della resistenza del partitore
a cui si trova in parallelo durante la misura, ne abbassa la resistenza
effettiva $R_2 \mapsto R_{\mathrm{eff}} = (\frac{1}{R_{\mathrm{in}}}
+ \frac{1}{R2})^{-1}$. Di conseguenza aumenta la corrente che passa per
il partitore, dunque la caduta di tensione ai capi di $R_1$, per cui
diminuiscono la tensione in uscita e quindi il valore di attenuazione, come
osservato.


\subsection*{6.e Resistenza di ingresso del multimetro}
Usando il modello mostrato nella scheda si ottiene
\begin{equation}\label{eq: divider}
\frac{R_1}{R_{IN}} =  \frac{V_{IN}}{V_{OUT}} - (1 +  \frac{R_1}{R_2} )
\end{equation}

Con i dati con resistenze da 1k si ottiene
\begin{equation}\label{eq: estimate}
R_1/R_{IN} = \exn  \pm  \exn   \rightarrow  R_{IN} > \exn k\Omega
\end{equation}


Con i dati con resistenze da 1M si ottiene
\[ R_1/R_{IN} = \exn  \pm  \exn   \rightarrow  R_{IN} = (\exn \pm  \exn)  M\Omega
\]

Quando la resistenza del multimetro $R_{IN} \gg R_2$ come visto al punto
6.b si ha $A \approx A_{\exp}$, per cui dalla \eqref{eq: divider}
\[
\frac{1}{A} - \frac{1}{A_{\exp}} = \frac{V_{IN}}{V_{OUT}} -
(1 +  \frac{R_1}{R_2}) = \frac{R_1}{R_{IN}}
\]
si vede come (a causa dell'incertezza sulla stima di $R_{IN}$ dalla
propagazione dell'errore sulla differenza) non sia possibile dare una misura
soddisfacente del suo valore. Ne possiamo però dare una stima dal basso:
\[
\frac{1}{A} \geq \frac{R_1}{R_{IN}} \implies R_{IN} \geq A R_1
\]
come in \eqref{eq: estimate}.

\section{Misure di tempo e frequenza}

\subsection*{7.e Misure di frequenza}
Misure con onda sinusoidale
\begin{table}[h]
\centering
\begin{tabular}{|c|c|c|c|c|c|}
\hline 
Periodo T ($\mu$ s)& $\sigma$ T ($\mu$ s) &Frequenza f (kHz) & $\sigma$ f (kHz) &
Misura oscilloscopio (kHz) & Differenza (kHz)\\
\hline 
999 & 10 & 0.99 & 0.01 & \exn &\exn \\
99.9 & 1.1 & 10.00 & 0.11 & \exn &\exn \\
9.99 & 0.10 & 100.0 & 1.0 & \exn &\exn \\
0.999 & 0.011 & 1000 & 11 & \exn &\exn \\
\hline 
\end{tabular} 
\caption{(7.e) Misura di frequenza di onde sinusoidali e confronto con
misurazione interna dell'oscilloscopio }
\end{table}

\subsection*{7.f Misure di duty cyle}
Misure con onda quadra
\begin{table}[h]
\centering
\begin{tabular}{|c|c|c|c|c|c|}
\hline 
Periodo T ($\mu$ s)& $\sigma$ T ($\mu$ s) & Durata alto $t_H$ (s) & $\sigma$ $t_H$ (s)
& Duty cycle D(\%) & $\sigma$ D (\%) \\
\hline 
100 & 2 & 9 & 2 & 0.09 & 0.02 \\
100 & 2 & 50 & 2 & 0.50 & 0.02 \\
100 & 2 & 90 & 2 & 0.90 & 0.02 \\
\exn & \exn & \exn & \exn & \exn &\exn \\
\hline 
\end{tabular} 
\caption{(7.f) Misura di duty cycle per onde quadre }
\end{table}


\subsection*{7.g Tempo di salita e di discesa}
Misure su onda quadra
\[
f = (1.000 \pm 0.011) \mathrm{MHz}, \quad
t_\mathrm{salita} = (35 \pm 6) \mathrm{ns},
t_\mathrm{discesa} = (37 \pm 6) \mathrm{ns},
\]

La misura è un po' balorda, visto che il tempo di salita/discesa è dello
stesso ordine di grandezza del periodo di campionamento $\nicefrac{1}{f_s} = \Delta t \approx 10 \si{\ns}$.
\framebox(400,30){Inserire commento su altre caratteristiche del segnale
ed eventualmente uno screenshot}

\section{Conclusioni e commenti finali}
\framebox(400,30){Inserire eventuali commenti e conclusioni finali}

\section*{Dichiarazione}
I firmatari di questa relazione dichiarano che il contenuto della relazione
\`e originale, con misure effettuate dai membri del gruppo, e che tutti i
firmatari hanno contribuito alla elaborazione della relazione stessa.

\end{document}