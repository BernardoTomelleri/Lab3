\documentclass[10pt, a4paper, italian]{article}
\usepackage[T1]{fontenc}
\usepackage[utf8]{inputenc}
\usepackage{amsmath, amssymb, amsthm, thmtools, amsfonts, mathtools}
\usepackage{nicefrac}
\usepackage{calc}
\usepackage[pdftex, hyperindex, plainpages=false]{hyperref}
\usepackage[nameinlink]{cleveref} %load before classicthesis (clash)
%\usepackage[nochapters,pdfspacing]{classicthesis}
\usepackage{siunitx}
\usepackage[siunitx]{circuitikz}

\usepackage[a4paper]{geometry}
\usepackage{float}
\usepackage{mdframed}
\usepackage{titling}
\usepackage{booktabs}
\usepackage{graphicx}
\usepackage{caption, subcaption}
\usepackage{xcolor}
\usepackage[italian]{babel}
\usepackage{pgfplots}
\usepackage{listings}
%\usepackage{lmodern}
\usepackage{url}
\usepackage{enumitem}
\usepackage{tikz} %loads after classicthesis (xcolor incompat)

% lets graphicx know path where figures to be included are found
\graphicspath{{../figs/}}
\makeatletter
\def\input@path{{../figs/}}
%or: \def\input@path{{/path/to/folder/}{/path/to/other/folder/}}
\makeatother

% tikz pgf plots setup
\usepgfplotslibrary{external}
\pgfplotsset{compat=1.15}
\tikzexternalize

% spaces and significant digits/figures for measurements
\sisetup{free-standing-units, space-before-unit, number-unit-product = \;,
scientific-notation = true, round-mode = figures, round-precision = 2,}

% turns all (hyperlinked) references black [default is blue]
\hypersetup{
	linktoc=all,
	colorlinks=true,
	linkcolor=black
}

% code listings config
\lstset{
language=Python,
basicstyle=\ttfamily,
columns=fullflexible,
keepspaces=true,
}

% mdframed (for boxed text) configuration
\mdfsetup{linewidth=0.6pt}

% Default fixed font does not support bold face
\DeclareFixedFont{\ttb}{T1}{txtt}{bx}{n}{12} % for bold
\DeclareFixedFont{\ttm}{T1}{txtt}{m}{n}{12}  % for normal

% Custom colors
\usepackage{color}
\definecolor{deepblue}{rgb}{0,0,0.5}
\definecolor{deepred}{rgb}{0.6,0,0}
\definecolor{deepgreen}{rgb}{0,0.5,0}

% Commands 
\newcommand{\executeiffilenewer}[3]{%
	\ifnum\pdfstrcmp{\pdffilemoddate{#1}}%
		{\pdffilemoddate{#2}}>0%
	{\immediate\write18{#3}}\fi%
}
% input .svg --> .pdf_tex graphs
\newcommand{\includesvg}[1]{%
	\executeiffilenewer{#1.svg}{#1.pdf}%
	{inkscape -z -D --file=#1.svg %
	--export-pdf=#1.pdf --export-latex}%
	\input{#1.pdf_tex}%
}
% Thanks UniPi's Department of Physics E. Fermi
\newcommand{\thanksdf}{(\thanks{Dipartimento di Fisica E.~Fermi,%
Universit\`a di Pisa - Pisa, Italy.}\;)}

% hyperlink to email address
\newcommand{\mail}[1]{\href{mailto:#1}{\textsf{#1}}}

\input{../../latex/math}
\geometry{left=2cm, right=2cm, top=2cm, bottom=2cm}

% makes all hyperlinks the same color as text
\hypersetup{
	linktoc=all,
	colorlinks=false,
	linkcolor=black
	}
% lets graphicx know path where figures to be included are found
\graphicspath{{../figs/}}

\author{Gruppo 1.AC \\ Matteo Rossi, Bernardo Tomelleri}
\title{Es12: Misura del rapporto $h/e$ per effetto fotoelettrico}
\begin{document}
\date{\today}
\maketitle

%=======================
\section{Scopo dell'esperienza}
Scopo dell'esperienza è verificare l'effetto fotoelettrico e la dipendenza
lineare tra energia e frequenza dei fotoni, dunque da questa ricavare una stima
del rapporto tra la costante di Planck e la carica dell'elettrone $h/e$
utilizzando il metodo del potenziale frenante (implementato per la prima volta
da R.A. Millikan, 1914-16).

\begin{figure}[htbp]
    \centering
	\includegraphics[scale=0.7]{schm}
    \caption{Schema dei circuiti di emissione e rilevazione di intensità
    luminosa.
    \label{schm: mesctrl}}
\end{figure}

\section{Metodo di misura}
Si sfrutta l'effetto fotoelettrico, secondo cui un elettrone può essere estratto da un metallo assorbendo un fotone con energia superiore al lavoro di estrazione $ W_{0} $ e venire emesso con energia cinetica data dalla conservazione dell'energia
\[ K = h f - W_{0} \]
dove $ f $ è la frequenza della radiazione elettromagnetica incidente sul metallo. Gli elettroni estratti per effetto fotoelettrico verranno chiamati

Si può calcolare il campo magnetico nella regione vicino al centro di ciascuna
bobina dalla legge di Biot-Savart e, quando queste sono poste ad una distanza
$a = r$ pari al loro raggio -cioè in configurazione di Helmholtz- si può
ricavare un'espressione per il campo totale come sovrapposizione dei due campi
\begin{equation}\label{eq: B-helm}
    B = \frac{\mu_0 N r^2 I\ped{coil}}{\left[r^2 + \left(\dfrac{r}{2}\right)^2
    \right]^\frac{3}{2}} =
    \left(\frac{4}{5}\right)^{\frac{3}{2}} \frac{\mu_{0} N}{r} I\ped{coil}.
\end{equation}
Nel piano parallelo alle spire passante per il punto medio dell'asse
congiungente i centri delle bobine (ovvero il piano della traiettoria degli
elettroni) il campo magnetico è parallelo all'asse $z$ delle spire ed ha
valore massimo della componente lungo lo stesso asse:
\[
Bz\ped{MAX} = 7.40 10^{-4} I\ped{coil}
\]

Un catodo, riscaldato da un filamento incandescente alimentato con una
tensione $V\ped{heat} = \SI{6}{\V}$ emette elettroni per effetto termoionico.
Gli elettroni vengono accelerati da una d.d.p. $V\ped{acc}$ compresa tra 150
e 250 V e, all'uscita dal cannone elettronico urtano gli atomi del gas
rarefatto (He, a pressione di $10^{-1} \; \si{\Pa}$) presente nell'ampolla,
i quali emettono la radiazione che consente di visualizzare il pennello e
misurarne l'orbita.

Una volta liberati dal catodo, nella regione in cui supponiamo assente il
campo elettrico $V\ped{acc}$, per la conservazione dell'energia vale
\begin{equation}\label{eq: T=qV}
    \frac{1}{2} m_{e} v^{2} = e V\ped{acc}
\end{equation}
Per cui, assumendo che il campo magnetico sia statico e uniforme lungo $z$ e
che il fascio di elettroni abbia velocità ortogonale all'asse delle spire,
ci aspettiamo che gli elettroni rimangano in moto circolare uniforme nel
piano ortogonale $x-y$.

Dalla condizione di moto circolare di raggio $R$ dovuto alla forza di Lorentz
abbiamo che
\[
m_{e} \frac{v^2}{R} = e v B \implies v = \frac{e}{m_e} B R
\]
Combinando l'~\cref{eq: T=qV} con la precedente troviamo
\[
v^2 = 2 V\ped{acc} \frac{e}{m_e} \implies \left(\frac{e}{m_e} B R\right)^2 =
2 V\ped{acc} \frac{e}{m_e}
\]
Da cui otteniamo l'equazione tramite cui vogliamo stimare il rapporto
\begin{equation}\label{eq:fit}
\frac{e}{m_{e}} = \frac{2 \Delta V}{(BR)^2}.
\end{equation}

Dal momento che tutte le variabili nel RHS sono direttamente controllabili
configurando le tensioni di alimentazione e possiamo misurare il raggio
della traiettoria $R$ analizzando (come faremo ad esempio con un fit
circolare) le fotografie del moto nel bulbo.

%=======================
\section{Descrizione delle misure}
\subsection{Orientazione delle bobine rispetto al campo magnetico terrestre}

\subsection{Mappatura del campo magnetico lungo l'asse delle bobine}

\subsection{Calibrazione dell'apparato per l'acquisizione delle traiettorie}

\subsection{Misura del raggio della traiettoria}
Sulle foto sopra menzionate si è effettuato un campionamento dei punti sull'arco interno e sull'arco esterno. Le coordinate dei pixel così ricavate sono state interpolate con un \emph{fit} circolare per ottenere una stima del raggio interno ed esterno. Si è poi assunto come valore efficace del raggio dell'orbita la media del raggio del cerchio interno e di quello esterno, e si è attribuito un errore pari alla semi-dispersione degli stessi. I raggi così ottenuti sono stati poi convertiti in unità fisiche come spiegato nella Sezione~\ref{sec:conv}. \\

\section{Analisi dati e stima del rapporto $e/m$}
La stima del rapporto $ e/m_{e} $ è stata poi ottenuta in due modi diversi: come media pesata delle singole stime di tale rapporto ottenute dalla~\eqref{eq:fit} ed effettuando un \emph{fit} lineare di $ 2\Delta V $ al variare di $ (B R)^{2} $ e ottenendo $ e/m_{e} $ dal coefficiente angolare della retta di \emph{best-fit}. \\
Assumendo $ e = \SI{1.602176634e11}{\coulomb} $ e $ m_{e} = \SI{9.10938370e-31}{\kilogram} $ il valore atteso del loro rapporto è
\begin{equation}\label{eq: e-m-exp}
    \left(\frac{e}{m_{e}}\right)\ped{exp} = \SI[]{ 175.882e9 }{ \coulomb/\kilogram }
\end{equation}

%=======================
\section{Valutazione degli effetti sistematici}
\subsection{Spessore del pennello elettronico}

\subsection{Dipendenza della stima di $e/m$ dal raggio dell'orbita $R$}

\subsection{Distorsione del bulbo di vetro}

\subsection{Campo magnetico terrestre}

\subsection{Disuniformità del campo magnetico sulla traiettoria}

%=======================
\section*{Conclusioni e commenti finali}
Si è riusciti a dare una misura ragionevole del rapporto carica/massa
dell'elettrone a partire da un'analisi delle fotografie della sua traiettoria
elicoidale in presenza di un campo magnetico uniforme.

%=======================
\section*{Dichiarazione}
I firmatari di questa relazione dichiarano che il contenuto della relazione \`e
originale, con misure effettuate dai membri del gruppo, e che tutti i firmatari
hanno contribuito alla elaborazione della relazione stessa.

%=======================
\begin{thebibliography}{1}
\bibitem{Coope}{I. D. Coope, Circle fitting by linear and nonlinear least
squares, Department of Mathematics, University of Canterbury, Christchurch,
New Zealand, N.60, May, 1992,
\url{https://ir.canterbury.ac.nz/bitstream/handle/10092/11104/coope_report_no69_1992.pdf?sequence=1&isAllowed=y}}
\end{thebibliography}

\end{document}
