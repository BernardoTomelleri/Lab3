\documentclass[10pt, a4paper, italian]{article}
\usepackage[T1]{fontenc}
\usepackage[utf8]{inputenc}
\usepackage{amsmath, amssymb, amsthm, thmtools, amsfonts, mathtools}
\usepackage{nicefrac}
\usepackage{calc}
\usepackage[pdftex, hyperindex, plainpages=false]{hyperref}
\usepackage[nameinlink]{cleveref} %load before classicthesis (clash)
%\usepackage[nochapters,pdfspacing]{classicthesis}
\usepackage{siunitx}
\usepackage[siunitx]{circuitikz}

\usepackage[a4paper]{geometry}
\usepackage{float}
\usepackage{mdframed}
\usepackage{titling}
\usepackage{booktabs}
\usepackage{graphicx}
\usepackage{caption, subcaption}
\usepackage{xcolor}
\usepackage[italian]{babel}
\usepackage{pgfplots}
\usepackage{listings}
%\usepackage{lmodern}
\usepackage{url}
\usepackage{enumitem}
\usepackage{tikz} %loads after classicthesis (xcolor incompat)

% lets graphicx know path where figures to be included are found
\graphicspath{{../figs/}}
\makeatletter
\def\input@path{{../figs/}}
%or: \def\input@path{{/path/to/folder/}{/path/to/other/folder/}}
\makeatother

% tikz pgf plots setup
\usepgfplotslibrary{external}
\pgfplotsset{compat=1.15}
\tikzexternalize

% spaces and significant digits/figures for measurements
\sisetup{free-standing-units, space-before-unit, number-unit-product = \;,
scientific-notation = true, round-mode = figures, round-precision = 2,}

% turns all (hyperlinked) references black [default is blue]
\hypersetup{
	linktoc=all,
	colorlinks=true,
	linkcolor=black
}

% code listings config
\lstset{
language=Python,
basicstyle=\ttfamily,
columns=fullflexible,
keepspaces=true,
}

% mdframed (for boxed text) configuration
\mdfsetup{linewidth=0.6pt}

% Default fixed font does not support bold face
\DeclareFixedFont{\ttb}{T1}{txtt}{bx}{n}{12} % for bold
\DeclareFixedFont{\ttm}{T1}{txtt}{m}{n}{12}  % for normal

% Custom colors
\usepackage{color}
\definecolor{deepblue}{rgb}{0,0,0.5}
\definecolor{deepred}{rgb}{0.6,0,0}
\definecolor{deepgreen}{rgb}{0,0.5,0}

% Commands 
\newcommand{\executeiffilenewer}[3]{%
	\ifnum\pdfstrcmp{\pdffilemoddate{#1}}%
		{\pdffilemoddate{#2}}>0%
	{\immediate\write18{#3}}\fi%
}
% input .svg --> .pdf_tex graphs
\newcommand{\includesvg}[1]{%
	\executeiffilenewer{#1.svg}{#1.pdf}%
	{inkscape -z -D --file=#1.svg %
	--export-pdf=#1.pdf --export-latex}%
	\input{#1.pdf_tex}%
}
% Thanks UniPi's Department of Physics E. Fermi
\newcommand{\thanksdf}{(\thanks{Dipartimento di Fisica E.~Fermi,%
Universit\`a di Pisa - Pisa, Italy.}\;)}

% hyperlink to email address
\newcommand{\mail}[1]{\href{mailto:#1}{\textsf{#1}}}

\input{../../latex/math}
\geometry{left=2cm, right=2cm, top=2cm, bottom=2cm}

% makes all hyperlinks the same color as text
\hypersetup{
	linktoc=all,
	colorlinks=false,
	linkcolor=black
	}
% lets graphicx know path where figures to be included are found
\graphicspath{{../figs/}}

\author{Gruppo 1.AC \\ Matteo Rossi, Bernardo Tomelleri}
\title{Es12: Misura del rapporto $h/e$ per effetto fotoelettrico}
\begin{document}
\date{\today}
\maketitle

%=======================
\section{Scopo dell'esperienza}
Scopo dell'esperienza è verificare l'effetto fotoelettrico e la dipendenza
lineare tra energia e frequenza dei fotoni, dunque da questa ricavare una stima
del rapporto tra la costante di Planck e la carica dell'elettrone $h/e$
utilizzando il metodo del potenziale frenante (implementato per la prima volta
da R.A. Millikan, 1914-16).

\begin{figure}[htbp]
    \centering
	\includegraphics[scale=0.7]{schm}
    \caption{Schema dei circuiti di emissione e rilevazione di intensità
    luminosa.
    \label{schm: mesctrl}}
\end{figure}

\section{Metodo di misura del potenziale frenante}
L'effetto fotoelettrico prevede che un elettrone può essere estratto da un
metallo assorbendo un fotone con energia superiore al lavoro di estrazione
$W_{0}$ e venire emesso con energia cinetica pari a
\begin{equation}\label{eq: cons}
E = h \nu - W_0
\end{equation}
dove $\nu$ è la frequenza della radiazione elettromagnetica incidente sul
metallo e $h \nu$ l'energia trasportata da ogni fotone di cui è costituita.

Nel nostro apparato sperimentale i \emph{fotoelettroni}, i.e. gli elettroni
estratti per effetto fotoelettrico dal catodo di una cella fotoelettrica
Leybold 5587 determinano una \emph{fotocorrente} verso l'anodo, che
colleghiamo all'ingresso di un picoamperometro (mod. Keithley 595: Quasistatic
C-V Meter) per riuscire a misurarne l'intensità $I_{ph}$.

Questi elettroni vengono `frenati' da un campo elettrico di verso concorde al
loro moto, generato da una tensione regolabile $V\ped{bias}$ applicata tra
anodo e catodo, al fine di trovare la differenza di potenziale critica $V_0$
per cui l'intensità di corrente $I_{ph}(V\ped{bias})$ si annulla.
La barriera di potenziale $eV_0$ corrispondente alla tensione che arresta il
flusso di fotoelettroni ci fornisce quindi una stima della massima
energia cinetica da essi raggiunta, che possiamo mettere in relazione alla
frequenza della radiazione incidente (per conservazione dell'energia
\cref{eq: cons})
\begin{equation}\label{eq: V0}
V_0 = \frac{h}{e} \nu - \frac{W_0}{e}
\end{equation}

Questa è proprio la relazione lineare tra energia e frequenza che intendiamo
verificare, tramite cui possiamo stimare il valore del rapporto $h/e$ a
partire dal coefficiente angolare della retta di best-fit.

%=======================
\section{Descrizione delle misure}
Dal momento che tutte le variabili nel RHS sono direttamente controllabili
configurando le tensioni di alimentazione e possiamo misurare il raggio
della traiettoria $R$ analizzando (come faremo ad esempio con un fit
circolare) le fotografie del moto nel bulbo.


\subsection{Stima dell'esponente $\alpha$}
Si è notato che la stima dei valori del potenziale frenante critico e
dell'esponente $\alpha$ esibivano una apprezzabile dipendenza dall'intervallo
di tensioni di bias considerate nel fit della fotocorrente.

Considerando valori di $V_{bias} \sim 0$, oltre
a quelle vicine al ginocchio in $V_{0}$, si è visto che il valore
dell'esponente $\alpha$ assume un andamento monotono crescente in funzione
della lunghezza d'onda in maniera analoga a quanto osservato finora per il
parametro $a$.

Si è scelto quindi di fissare il valore di $\alpha$ come media pesata dei
valori ottimali stimati dai fit prendendo in considerazione l'intero range di
tensioni esplorato, dunque di eseguire un nuovo fit ai 4 dataset lasciando
liberi solamente i restanti parametri dipendenti dalla frequenza della luce
incidente: $V_{0}$ e $a$.

In particolare, dal fit con alpha libero si trova
np.mean(alphas)
Out[27]: 2.4806408028824034

np.mean(dalphas)
Out[28]: 0.010212701821352213

1/pars[0]
Out[29]: 3.568317911720477
Quindi un valore del rapporto h/e sottostimato di circa il $15 \%$ rispetto al
valore atteso.

Mentre dai valori di $V_{0}$ ricavati fissando $\alpha$ proponiamo un fit
lineare alla relazione inversa
\begin{equation}
\nu = \frac{e}{h} V_{0} - W_{0}/h
\end{equation}
da cui otteniamo come valore del rapporto $h/e = 4.19 \pm 0.13 \; \si{\V \s}$,
che è compatibile entro l'incertezza con il valore tabulato di 4.14.

\section{Analisi dati e stima del rapporto $e/m$}
La stima del rapporto $ e/m_{e} $ è stata poi ottenuta in due modi diversi: come media pesata delle singole stime di tale rapporto ottenute dalla~\eqref{eq:fit} ed effettuando un \emph{fit} lineare di $ 2\Delta V $ al variare di $ (B R)^{2} $ e ottenendo $ e/m_{e} $ dal coefficiente angolare della retta di \emph{best-fit}. \\
Assumendo $ e = \SI{1.602176634e11}{\coulomb} $ e $ m_{e} = \SI{9.10938370e-31}{\kilogram} $ il valore atteso del loro rapporto è
\begin{equation}\label{eq: e-m-exp}
    \left(\frac{e}{m_{e}}\right)\ped{exp} = \SI[]{ 175.882e9 }{ \coulomb/\kilogram }
\end{equation}

%=======================
\section{Valutazione degli effetti sistematici}
In realtà il nostro apparato è racchiuso da una scatola metallica per
schermarlo dalla luce e dal rumore elettronico ambientale, ma la misura
di $h/e$ è affetta da diverse fonti di errore sistematico.

\subsection{Effetto fotoelettrico sull'anodo}
Dal datasheet della cella fotoelettrica sappiamo che questa è composta da un
anodo in lega di platino e rodio, che è una spira di $SI{3}{c\meter}$ e da un
catodo in potassio rivestito di ossido d'argento, costituito da un arco
delle dimensioni di $\SI{4}{c\meter}$.

Per minimizzare il contributo alla corrente inversa dato dall'emissione di
fotoelettroni dall'anodo quando viene investito dagli aloni del fascio di
luce incidente viene ridotta l'apertura del diaframma così che la sua
immagine, messa a fuoco al centro del catodo, abbia dimensioni molto
inferiori al raggio dell'anodo.

Non solo, il potassio, in quanto metallo alcalino è caratterizzato da un
lavoro di estrazione $W\ped{0, K} \approx 2.15 \; \si{\V}$ più basso di quello
del platino $W\ped{0, Pt} \approx 5.29 \; \si{\V}$

Il raggio di luce monocromatica viene focalizzato dal diaframma a iride così da incidere sul catodo della fotocellula. I fotoelettroni emessi dal catodo si spostano poi verso l'anodo dando luogo a una una corrente.

Tuttavia parte della luce incidente potrebbe incidere sull'anodo dando luogo a una corrente inversa: purché minoritaria, questa diventa più rilevante per valori prossimi a $V\ped{0}$ in quanto i fotoelettroni emessi dall'anodo vengono accelerati dalla differenza di potenziale. Per $ V > V_{0} $ la corrente prodotta dai fotoelettroni emessi dal catodo si annulla e la corrente inversa dell'anodo è misurabile.

Questo effetto è limitabile modificando l'apertura del diaframma. Inoltre cambiando filtro e dunque cambiando lunghezza d'onda, aree differenti del catodo potrebbero essere illuminate: questo effetto è da evitare il più possibile essendo il lavoro di estrazione $W\ped{0}$ dipendente dalle caratteristiche di omogeneità della superficie del materiale.

\subsection{Effetto termoionico e connessione ohmica tra gli elettrodi}
A differenza dell'effetto precedentemente citato ci sono due effetti che non possono essere soppressi.
\begin{itemize}
    \item L'effetto termoionico si ha sia sul catodo che sull'anodo e diventa pià significativo all'aumentare della tensione di \emph{bias}. La corrente $ I\ped{inv} $ misurata per $ V > V\ped{0}$ è principalmente attribuibile a questo contributo e dovrebbe essere osservabile (e quindi quantificabile in modo più preciso) anche in assenza di illuminazione;

    \item I cavi coassiali uscenti dalla scatola metallica in cui è contenuto l'apparato sono connessi al catodo e all'anodo e poi esternamente collegati rispettivamente in parallelo al voltmetro e in serie al picoamperometro. La scatola metallica scherma dal rumore elettronico, ma le guaine di rivestimento dei cavi non sono perfettamente isolanti e questo può comportare una ulteriore corrente inversa tra gli elettrodi, che per via dell'ordine di grandezza delle correnti in gioco, diventa non trascurabile.
\end{itemize}

\subsection{Considerazioni sul metodo \texttt{A}}
Nel \hyperref[sec:metodoA]{metodo \texttt{A}} si determina il valore della corrente inversa interpolando le misure nella regione $I \leq 0 $ con un modello lineare. I parametri $ b $ e $ I_{0} $ non dovrebbero dipendere dalla lunghezza d'onda della radiazione incidente: considerando un modello del tipo equazione di Shockley per un diodo a giunzione per la fotocellula, il modello lineare dovrebbe essere uno sviluppo al primo ordine dell'equazione caratteristica nel range considerato, con $ b $ parametro nell'esponenziale e $ I_{0} $ valore asintotico della corrente.

Tuttavia dalla~\eqref{fig:fit-oscuro} è possibile notare come $ I\ped{inv} $ abbia in realtà andamenti diversi al variare della lunghezza d'onda: per le lunghezze d'onda minori, quindi per fotoni più energetici, il potenziale di frenamento $ V_{0} $ è più elevato e dunque la corrente tende più lentamente al valore asintotico. Questo comporta una deviazione significativa dall'andamento lineare e dunque i parametri del \emph{fit} manifestano necessariamente una dipendenza dalla lunghezza d'onda impiegata.

\subsection{Considerazioni sul metodo \texttt{B}}
Nel \hyperref[sec:metodoB]{metodo \texttt{B}} si adopera l'equazione~\eqref{eq:modello-magico} e in tale caso i parametri $ b $ e $ I_{0} $ risultano compatibili. Nella zona di transizione $ V\sim V_{0} $, il parametro $ a $ tiene conto della dipendenza dalla lunghezza d'onda: dai parametri di \emph{best-fit} si nota infatti che $ a $ cambia significativamente al variare della lunghezza d'onda ed è monotona crescente nella lunghezza d'onda.

L'origine fisica del parametro $\alpha$ nella legge di potenza può essere compresa descrivendo, in prima approssimazione, il gas di elettroni nella banda di conduzione del metallo alcalino di cui è costituito il catodo come un gas degenere di Fermi: ovviamente questo modello considera gli elettroni nel metallo come liberi e non tiene conto delle successive interazioni dei fotoelettroni con il bulbo della fotocella. Si ottengono infatti valori maggiori di $ 3/2 $, gli $ \alpha $ di \emph{best-fit} non risultano compatibili tra loro e risultano monotoni crescenti nella lunghezza d'onda.



%=======================
\section*{Conclusioni e commenti finali}
Si è riusciti a dare una misura ragionevole del rapporto carica/massa
dell'elettrone a partire da un'analisi delle fotografie della sua traiettoria
elicoidale in presenza di un campo magnetico uniforme.

%=======================
\section*{Dichiarazione}
I firmatari di questa relazione dichiarano che il contenuto della relazione \`e
originale, con misure effettuate dai membri del gruppo, e che tutti i firmatari
hanno contribuito alla elaborazione della relazione stessa.

%=======================
\begin{thebibliography}{1}
\bibitem{Coope}{I. D. Coope, Circle fitting by linear and nonlinear least
squares, Department of Mathematics, University of Canterbury, Christchurch,
New Zealand, N.60, May, 1992,
\url{https://ir.canterbury.ac.nz/bitstream/handle/10092/11104/coope_report_no69_1992.pdf?sequence=1&isAllowed=y}}
\end{thebibliography}

\end{document}
