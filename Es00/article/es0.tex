\documentclass[a4paper, 12pt, italian]{article}
\usepackage[T1]{fontenc}
\usepackage[utf8]{inputenc}
\usepackage{amsmath, amssymb, amsthm, thmtools, amsfonts, mathtools}
\usepackage{nicefrac}
\usepackage{calc}
\usepackage[pdftex, hyperindex, plainpages=false]{hyperref}
\usepackage[nameinlink]{cleveref} %load before classicthesis (clash)
%\usepackage[nochapters,pdfspacing]{classicthesis}
\usepackage{siunitx}
\usepackage[siunitx]{circuitikz}

\usepackage[a4paper]{geometry}
\usepackage{float}
\usepackage{mdframed}
\usepackage{titling}
\usepackage{booktabs}
\usepackage{graphicx}
\usepackage{caption, subcaption}
\usepackage{xcolor}
\usepackage[italian]{babel}
\usepackage{pgfplots}
\usepackage{listings}
%\usepackage{lmodern}
\usepackage{url}
\usepackage{enumitem}
\usepackage{tikz} %loads after classicthesis (xcolor incompat)

% lets graphicx know path where figures to be included are found
\graphicspath{{../figs/}}
\makeatletter
\def\input@path{{../figs/}}
%or: \def\input@path{{/path/to/folder/}{/path/to/other/folder/}}
\makeatother

% tikz pgf plots setup
\usepgfplotslibrary{external}
\pgfplotsset{compat=1.15}
\tikzexternalize

% spaces and significant digits/figures for measurements
\sisetup{free-standing-units, space-before-unit, number-unit-product = \;,
scientific-notation = true, round-mode = figures, round-precision = 2,}

% turns all (hyperlinked) references black [default is blue]
\hypersetup{
	linktoc=all,
	colorlinks=true,
	linkcolor=black
}

% code listings config
\lstset{
language=Python,
basicstyle=\ttfamily,
columns=fullflexible,
keepspaces=true,
}

% mdframed (for boxed text) configuration
\mdfsetup{linewidth=0.6pt}

% Default fixed font does not support bold face
\DeclareFixedFont{\ttb}{T1}{txtt}{bx}{n}{12} % for bold
\DeclareFixedFont{\ttm}{T1}{txtt}{m}{n}{12}  % for normal

% Custom colors
\usepackage{color}
\definecolor{deepblue}{rgb}{0,0,0.5}
\definecolor{deepred}{rgb}{0.6,0,0}
\definecolor{deepgreen}{rgb}{0,0.5,0}

% Commands 
\newcommand{\executeiffilenewer}[3]{%
	\ifnum\pdfstrcmp{\pdffilemoddate{#1}}%
		{\pdffilemoddate{#2}}>0%
	{\immediate\write18{#3}}\fi%
}
% input .svg --> .pdf_tex graphs
\newcommand{\includesvg}[1]{%
	\executeiffilenewer{#1.svg}{#1.pdf}%
	{inkscape -z -D --file=#1.svg %
	--export-pdf=#1.pdf --export-latex}%
	\input{#1.pdf_tex}%
}
% Thanks UniPi's Department of Physics E. Fermi
\newcommand{\thanksdf}{(\thanks{Dipartimento di Fisica E.~Fermi,%
Universit\`a di Pisa - Pisa, Italy.}\;)}

% hyperlink to email address
\newcommand{\mail}[1]{\href{mailto:#1}{\textsf{#1}}}

\input{../../latex/math}

% adjustable page margins, currently scientific article standards
\geometry{left=25mm, right=25mm, top=25mm, bottom=25mm}

\title{Esercitazione 0 di Laboratorio 3}
\author{B.~Tomelleri\thanksdf \and M.~Rossi(\protect\footnotemark[1] )}
\date{\today}

\begin{document}
\maketitle

%================================================================
%                         Introduzione
%================================================================
\section{Introduzione}
Scopo dell’esercitazione è di acquisire familiarità con le funzionalità
di Waveforms come oscilloscopio e generatore di funzioni.

%================================================================
%                         Cenni teorici
%================================================================
\section{Cenni Teorici}

%================================================================
%                Metodo e apparato sperimentale
%================================================================
\section{Metodo e apparato sperimentale}
Digilent AD2 e applicazione dedicata Waveforms

%================================================================
%                   Analisi dati e Risultati
%================================================================
\section{Analisi dati e Risultati}

Da un \emph{fit} con un modello:
\begin{lstlisting}
def dosc(t, A, frq, phi, ofs, tau):
    return A*np.exp(-t/tau)*np.cos(2*np.pi*frq*t + phi) + B
\end{lstlisting}

Lasciando liberi tutti i parametri $A$, $\omega$, $\tau$ $\phi$ e $B$
(e propagando gli errori sulla variabile indipendente) si ottengono i valori:
\begin{align*}
A &= 1.2045 \pm 0.0007  \; \si{\volt} \qquad &B &= -73.27 \pm 0.02 \;
\si{m\volt} \\
f &= 5722 \pm 1 \; \si{\hertz} \qquad &\phi &= 311 \pm 2 \; \si{m\radian} \\  
\tau &= 88.41 \pm 0.02 \; \si{\us} \qquad &Q_f &= 1.6 \pm 0.2 \\
\rm{norm\_cov_{(A, f)}} &= 0.71 \qquad &\rm{norm\_cov_{(A, B)}} &= -0.03 \\
\rm{norm\_cov_{(A, \tau)}} &= -0.73 \qquad &\rm{norm\_cov_{(A, \phi)}} &= -0.85 \\ 
\rm{norm\_cov_{(f, \phi)}} &= -0.84 \qquad &\rm{norm\_cov_{(f, \tau)}} &= -0.59 \\
\rm{norm\_cov_{(f, B)}} &= -0.02 \qquad &\rm{norm\_cov_{(\phi, B)}} &= 0.02 \\
\rm{norm\_cov_{(\phi, \tau)}} &= 0.53 \qquad &\rm{norm\_cov_{(B, \tau)}} &= 0.05 \\
\chi^2 &= 215.4/245 \qquad &\rm{abs\_sigma} &= \rm False
\end{align*}

Ed il grafico \ref{plt:rlcfit}\\
\begin{figure}[!htb]
	\centering 
 		\includegraphics[scale=0.9]{rlcfit}
    \caption{Fit del segnale $V(t)$ rispetto al tempo \label{plt:rlcfit}}
\end{figure}
% Alternatively
%\begin{figure}[!htb]
%	\centering 
% 		\includegraphics[scale=0.9]{./figs/dist.pdf}
% 	\caption{Distribuzione delle • per $• = • \pm • \rm •$ \label{dist:1}}
%\end{figure}
In entrambi casi impostando $\rm{abs\_sigma} = \rm False$ in quanto l'errore
non statistico è predominante sulle misure.

% Alternatively
%\lstinputlisting[language=Python, firstline=38, lastline=39]{../foo.py}
%or even
%\begin{align*}
%\rm def\; &foo(•_{•}, •_{•}, •_{•}):\\
%    &\rm return\; •_{•}/•_{•} + •_{•}
%\end{align*}

\subsection{Nota sul metodo di fit}
Per determinare i parametri ottimali e le rispettive covarianze si \`e
implementato in \verb+Python+ un algoritmo di fit basato sui minimi quadrati
mediante la funzione \emph{curve\_fit} della libreria 
\texttt{SciPy}\cite{scipy}.
%================================================================
%                          Conclusioni
%================================================================
\section{Conclusioni}
La forma d'onda oscillante ha si spegne dopo qualche periodo, per cui
il fattore di merito trovato è ragionevole, essendo maggiore del valore
$Q_c = \nicefrac{1}{2}$ corrispondente a smorzamento critico del sistema,
ma comunque dello stesso ordine di grandezza.
\bibliography{../../latex/refs}
\end{document}
