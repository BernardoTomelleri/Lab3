\documentclass[10pt,a4paper]{article}
\usepackage[T1]{fontenc}
\usepackage[utf8]{inputenc}
\usepackage{amsmath, amssymb, amsthm, thmtools, amsfonts, mathtools}
\usepackage{nicefrac}
\usepackage{calc}
\usepackage[pdftex, hyperindex, plainpages=false]{hyperref}
\usepackage[nameinlink]{cleveref} %load before classicthesis (clash)
%\usepackage[nochapters,pdfspacing]{classicthesis}
\usepackage{siunitx}
\usepackage[siunitx]{circuitikz}

\usepackage[a4paper]{geometry}
\usepackage{float}
\usepackage{mdframed}
\usepackage{titling}
\usepackage{booktabs}
\usepackage{graphicx}
\usepackage{caption, subcaption}
\usepackage{xcolor}
\usepackage[italian]{babel}
\usepackage{pgfplots}
\usepackage{listings}
%\usepackage{lmodern}
\usepackage{url}
\usepackage{enumitem}
\usepackage{tikz} %loads after classicthesis (xcolor incompat)

% lets graphicx know path where figures to be included are found
\graphicspath{{../figs/}}
\makeatletter
\def\input@path{{../figs/}}
%or: \def\input@path{{/path/to/folder/}{/path/to/other/folder/}}
\makeatother

% tikz pgf plots setup
\usepgfplotslibrary{external}
\pgfplotsset{compat=1.15}
%\tikzexternalize

% spaces and significant digits/figures for measurements
\sisetup{free-standing-units, space-before-unit, number-unit-product = \;,
scientific-notation = false, round-mode = figures, round-precision = 1,}

% turns all (hyperlinked) references black [default is blue]
\hypersetup{
	linktoc=all,
	colorlinks=true,
	linkcolor=black
}

% code listings config
%\lstset{
%language=Python,
%basicstyle=\ttfamily,
%columns=fullflexible,
%keepspaces=true,
%}

% mdframed (for boxed text) configuration
\mdfsetup{linewidth=0.6pt}

% Default fixed font does not support bold face
\DeclareFixedFont{\ttb}{T1}{txtt}{bx}{n}{12} % for bold
\DeclareFixedFont{\ttm}{T1}{txtt}{m}{n}{12}  % for normal

% Custom colors
\usepackage{color}
\definecolor{deepblue}{rgb}{0,0,0.5}
\definecolor{deepred}{rgb}{0.6,0,0}
\definecolor{deepgreen}{rgb}{0,0.5,0}

% Commands 
\newcommand{\executeiffilenewer}[3]{%
	\ifnum\pdfstrcmp{\pdffilemoddate{#1}}%
		{\pdffilemoddate{#2}}>0%
	{\immediate\write18{#3}}\fi%
}
% input .svg --> .pdf_tex graphs
%\newcommand{\includesvg}[1]{%
%	\executeiffilenewer{#1.svg}{#1.pdf}%
%	{inkscape -z -D --file=#1.svg %
%	--export-pdf=#1.pdf --export-latex}%
%	\input{#1.pdf_tex}%
%}
% Thanks UniPi's Department of Physics E. Fermi
\newcommand{\thanksdf}{(\thanks{Dipartimento di Fisica E.~Fermi,%
Universit\`a di Pisa - Pisa, Italy.}\;)}

% hyperlink to email address
\newcommand{\mail}[1]{\href{mailto:#1}{\textsf{#1}}}

\geometry{left=2cm, right=2cm, top=2cm, bottom=2cm}
\newcommand{\rem}[1]{[\emph{#1}]}
\newcommand{\exn}{\phantom{xxx}}

% lets graphicx know path where figures to be included are found
\graphicspath{{../figs/}}

\author{Gruppo 1.AC \\ Matteo Rossi, Bernardo Tomelleri}
\title{Es02A: Circuito RC -- Filtri passivi}
\begin{document}
\date{\today}
\maketitle

\setcounter{section}{1}

\section*{Filtro passa-basso}

\subsection*{1.b Scelta della frequenza di taglio}

La frequenza nominale di taglio \`e stata fissata a $f_1 = \ldots\;\; \Rightarrow |A_v(3\,\mathrm{kHz})| = \ldots\;\; |A_v(30\,\mathrm{kHz})| = \ldots\;\; $  

\vspace{0.5cm}
\framebox(400,30){Motivare la scelta di $f_1$}

\subsection*{1.c,1.d Scelta dei componenti}

I valori nominali scelti sono $R_1 = \dots\;\;C_1 = \ldots$.  

\vspace{0.5cm}
\framebox(400,30){Motivare la scelta dei componenti (\`e sufficiente  anche solo indicare le formule di riferimento)}

\subsection*{1.e Misura di $C_1$}
\[
C_1 = \ldots \pm \ldots 
\]

\subsection*{1.f Calcolo della frequenza di taglio e delle attenuazioni attese}
\[
\begin{array}{rcl}
f_1 &=& \ldots \pm \ldots\\
|A_v(2\,\mathrm{kHz})| &=& \ldots \pm \ldots\\
|A_v(20\,\mathrm{kHz})| &=& \ldots \pm \ldots
\end{array}
\]

\subsection*{3 Misura $A_v$}
Dalla misura delle ampiezze dei segnali di ingresso/uscita e del loro sfasamento si ottiene:
\begin{table}[h]
\centering
\begin{tabular}{|c|c|c|c|c|}
\hline 
$f$ & $V_s \pm \sigma(V_s)$  & $V_{out} \pm \sigma(V_{out})$ & $A_v \pm \sigma(A_V)$ & $\phi \pm \sigma(\phi)$ \\
\hline 
\exn & \exn & \exn & \exn & \exn  \\
\exn & \exn & \exn & \exn & \exn  \\
\exn & \exn & \exn & \exn & \exn  \\
\hline 
\end{tabular} 
\caption{(3) Amplficazione e sfasamento del filtro passa-basso a bassa ed alta frequenza ed alla frequenza nominale di taglio.
\label{tab:par1}}
\end{table}

\subsection*{4 Risposta in frequenza}
\begin{figure}[h]
\centering
%\includegraphics[scale=0.4]{part1.pdf}
\framebox(400,50){ (4) Salvare ed inserire l'~immagine del Network analyzer}
\caption{(4) Plot di Bode per il filtro passa-basso.}
\end{figure}

\subsection*{5.a Stima della frequenza di taglio (metodo a)}
La nostra stima della frequenza per cui $A_v$(dB) = -3 dB \`e
\[
f_{1A} = \ldots\pm \ldots
\]

\subsection*{5.b Misura della frequenza di taglio (metodo b)}
Dal fit a bassa frequenza ($f\ll f_1$) otteniamo
\[
A_1(dB) = \ldots\pm \ldots\;\;\; \chi^2 = \ldots\;\;\; d.o.f. = \ldots
\]

Ad alta frequenza ($f \gg f_1$) la retta di best-fit al plot di Bode in ampiezza ha i seguenti parametri:
\[
\mathrm{intercetta} = \ldots\pm \ldots\;\;\;\mathrm{pendenza}= \ldots\pm \ldots\;\;\;\mathrm{covarianza} 
= \ldots\;\;\;\chi^2 = \ldots\;\;\; d.o.f. = \ldots
\]
Dall'~intersezione delle due rette stimiamo per la frequenza di taglio il valore
\[
f_{1B} = \ldots\pm \ldots
\]

\subsection*{5.c Misura della frequenza di taglio (metodo c)}
Dal fit complessivo del modulo della funzione di trasferimento  
otteniamo per l'~amplificazione di centro-banda e per la frequenza di taglio i seguenti valori:
\[
A_1(dB) = \ldots\pm \ldots \;\;\;f_{1B} = \ldots\pm \ldots\;\;\;\chi^2 = \ldots\;\;\; d.o.f.= \ldots
\]

\subsection*{5.d Confronto misure-predizione}
\vspace{0.5cm}
\framebox(400,30){Commentare l'~accordo tra le varie stime di $f_1$ ed il valore atteso.}

\subsection*{6 Risposta del filtro ad un gradino}
Il fronte del segnale di uscita ha un tempo di salita, misurato con i cursori, di 
\[
t_r = \ldots\pm \ldots
\]
da cui 
\[
f_1 \simeq \frac{2.2}{2\pi t_r} = \ldots\pm \ldots
\]
\begin{figure}[h]
\centering
%\includegraphics[scale=0.4]{part1.pdf}
\framebox(400,50){ (6) Salvare ed inserire uno screenshot dell'~oscillografo.}
\caption{(6) Risposta del filtro passa-basso ad un gradino di tensione.}
\end{figure}

\subsection*{7.a Impedenze di ingresso/uscita}
(Qui \`e sufficiente scrivere le espressioni in termini della frequenza e dei componenti)
\[
Z_{in} = \ldots
\]
\[
Z_{out} = \ldots
\]

\subsection*{7.b Effetti dovuti all'~accoppiamento con un carico}
(Qui \`e richiesto che valutiate l'~amplificazione di centro-banda e la frequenza di taglio nel 
caso in cui il carico sia rispettivamente 100 e 10 k$\Omega$)
\[
\begin{array}{rl}
R_L=100 \,k\Omega & \Rightarrow A_1 = \ldots\;\;\; f_1 = \ldots\\
R_L=10 \,k\Omega & \Rightarrow A_1 = \ldots\;\;\; f_1 = \ldots\\
\end{array}
\]

%=======================
\section*{Filtro passa-banda}

\subsection*{8.a Misura dei componenti}
\[ 
C_1 = \ldots\pm \ldots
\]

\subsection*{8.b Filtro passa-basso, stima della frequenza di taglio}
Dalla risposta in frequenza risulta
\[
A_1(dB) = \ldots \pm \ldots,\;\;f_1 = \ldots\pm \ldots
\]

\subsection*{9.a Misura dei componenti}
\[ 
C_2 = \ldots\pm \ldots
\]

\subsection*{9.b Filtro passa-alto, stima della frequenza di taglio}
Dalla risposta in frequenza risulta
\[
A_2(dB) = \ldots \pm \ldots,\;\;f_2 = \ldots\pm \ldots
\]

\subsection*{10.a Filtro passa-banda, risposta in frequenza}
\begin{figure}[h]
\centering
%\includegraphics[scale=0.4]{part1.pdf}
\framebox(400,50){ (10.a) Salvare ed inserire l'~immagine del Network analyzer per il passa-banda}
\caption{(4) Plot di Bode per il filtro passa-banda.}
\end{figure}
La nostra stima dell'~amplificazione di centro-banda e delle frequenze di taglio (per cui il guadagno si riduce di 3 dB rispetto a centro-banda) \`e
\[
A(dB) = \ldots\pm \ldots \;\;\;f_{L} = \ldots\pm \ldots\;\;\;f_{H} = \ldots\pm \ldots
\]

\subsection*{10.b Interpolazione del plot di Bode}
Dal fit del plot di Bode in ampiezza si ha
\[
A(dB) = \ldots\pm \ldots \;\;\;f_{L} = \ldots\pm \ldots\;\;\;f_{H} = \ldots\pm \ldots\;\;\;\chi^2 = \ldots\;\;\; d.o.f.= \ldots
\]
\subsection*{10.c Differenze}
\vspace{0.5cm}
\framebox(400,30){Motivare la differenza rispetto ai filtri standalone}

\subsection*{10.d Dipendenza dai valori delle resistenze}
\vspace{0.5cm}
\framebox(400,30){Commentare la dipendenza dalle resistenze, come da guida}

\subsection*{10.e Andamento della fase}
\vspace{0.5cm}
\framebox(400,30){Commentare la dipendenza della fase dalla frequenza}

\section*{Conclusioni e commenti finali}
\framebox(400,30){Inserire eventuali commenti e conclusioni finali}

\section*{Dichiarazione}
I firmatari di questa relazione dichiarano che il contenuto della relazione \`e originale, con misure effettuate dai membri del gruppo, e che tutti i firmatari hanno contribuito alla elaborazione della relazione stessa.

\end{document}