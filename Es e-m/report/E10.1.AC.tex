\documentclass[10pt, a4paper, italian]{article}
\usepackage[T1]{fontenc}
\usepackage[utf8]{inputenc}
\usepackage{amsmath, amssymb, amsthm, thmtools, amsfonts, mathtools}
\usepackage{nicefrac}
\usepackage{calc}
\usepackage[pdftex, hyperindex, plainpages=false]{hyperref}
\usepackage[nameinlink]{cleveref} %load before classicthesis (clash)
%\usepackage[nochapters,pdfspacing]{classicthesis}
\usepackage{siunitx}
\usepackage[siunitx]{circuitikz}

\usepackage[a4paper]{geometry}
\usepackage{float}
\usepackage{mdframed}
\usepackage{titling}
\usepackage{booktabs}
\usepackage{graphicx}
\usepackage{caption, subcaption}
\usepackage{xcolor}
\usepackage[italian]{babel}
\usepackage{pgfplots}
\usepackage{listings}
%\usepackage{lmodern}
\usepackage{url}
\usepackage{enumitem}
\usepackage{tikz} %loads after classicthesis (xcolor incompat)

% lets graphicx know path where figures to be included are found
\graphicspath{{../figs/}}
\makeatletter
\def\input@path{{../figs/}}
%or: \def\input@path{{/path/to/folder/}{/path/to/other/folder/}}
\makeatother

% tikz pgf plots setup
\usepgfplotslibrary{external}
\pgfplotsset{compat=1.15}
\tikzexternalize

% spaces and significant digits/figures for measurements
\sisetup{free-standing-units, space-before-unit, number-unit-product = \;,
scientific-notation = true, round-mode = figures, round-precision = 2,}

% turns all (hyperlinked) references black [default is blue]
\hypersetup{
	linktoc=all,
	colorlinks=true,
	linkcolor=black
}

% code listings config
\lstset{
language=Python,
basicstyle=\ttfamily,
columns=fullflexible,
keepspaces=true,
}

% mdframed (for boxed text) configuration
\mdfsetup{linewidth=0.6pt}

% Default fixed font does not support bold face
\DeclareFixedFont{\ttb}{T1}{txtt}{bx}{n}{12} % for bold
\DeclareFixedFont{\ttm}{T1}{txtt}{m}{n}{12}  % for normal

% Custom colors
\usepackage{color}
\definecolor{deepblue}{rgb}{0,0,0.5}
\definecolor{deepred}{rgb}{0.6,0,0}
\definecolor{deepgreen}{rgb}{0,0.5,0}

% Commands 
\newcommand{\executeiffilenewer}[3]{%
	\ifnum\pdfstrcmp{\pdffilemoddate{#1}}%
		{\pdffilemoddate{#2}}>0%
	{\immediate\write18{#3}}\fi%
}
% input .svg --> .pdf_tex graphs
\newcommand{\includesvg}[1]{%
	\executeiffilenewer{#1.svg}{#1.pdf}%
	{inkscape -z -D --file=#1.svg %
	--export-pdf=#1.pdf --export-latex}%
	\input{#1.pdf_tex}%
}
% Thanks UniPi's Department of Physics E. Fermi
\newcommand{\thanksdf}{(\thanks{Dipartimento di Fisica E.~Fermi,%
Universit\`a di Pisa - Pisa, Italy.}\;)}

% hyperlink to email address
\newcommand{\mail}[1]{\href{mailto:#1}{\textsf{#1}}}

\input{../../latex/math}
\geometry{left=2cm, right=2cm, top=2cm, bottom=2cm}

% makes all hyperlinks the same color as text
\hypersetup{
	linktoc=all,
	colorlinks=false,
	linkcolor=black
	}
% lets graphicx know path where figures to be included are found
\graphicspath{{../figs/}}

\author{Gruppo 1.AC \\ Matteo Rossi, Bernardo Tomelleri}
\title{Es10: Misura del rapporto carica-massa dell'elettrone $e/m_e$}
\begin{document}
\date{\today}
\maketitle

%=======================
\section{Scopo dell'esperienza}
Si vuole misurare il rapporto $e/m$ attraverso la misura del raggio di
curvatura della traiettoria circolare di un fascio di elettroni immersi in un
campo magnetico uniforme (generato da bobine in configurazione di Helmholtz)
accelerati da una differenza di potenziale nota.

\section{Metodo di misura}
Consideriamo il campo magnetico prodotto da due bobine coassiali di raggio
medio $r = 15.8 \; \si{c\m}$, costituite da $N = 130$ spire collegate in
serie e percorse da una stessa corrente di intensità $I\ped{coil}$ da noi
controllabile.

Si può calcolare il campo magnetico nella regione vicino al centro di ciascuna
bobina dalla legge di Biot-Savart e, quando queste sono poste ad una distanza
$a = r$ pari al loro raggio -cioè in configurazione di Helmholtz- si può
ricavare un'espressione per il campo totale come sovrapposizione dei due campi
\begin{equation}\label{eq: B-helm}
    B = \frac{\mu_0 N r^2 I\ped{coil}}{\left[r^2 + \left(\dfrac{r}{2}\right)^2
    \right]^\frac{3}{2}} =
    \left(\frac{4}{5}\right)^{\frac{3}{2}} \frac{\mu_{0} N}{r} I\ped{coil}.
\end{equation}
Nel piano parallelo alle spire passante per il punto medio dell'asse
congiungente i centri delle bobine (ovvero il piano della traiettoria degli
elettroni) il campo magnetico è parallelo all'asse $z$ delle spire ed ha
valore massimo della componente lungo lo stesso asse:
\begin{equation}\label{eq: Bmax}
Bz\ped{MAX} = 7.40 10^{-4} \left[\frac{T}{A}\right] I\ped{coil}
\end{equation}


Un catodo, riscaldato da un filamento incandescente alimentato con una
tensione $V\ped{heat} = \SI{6}{\V}$ emette elettroni per effetto termoionico.
Gli elettroni vengono accelerati da una d.d.p. $V\ped{acc}$ compresa tra 150
e 250 V e, all'uscita dal cannone elettronico urtano gli atomi del gas
rarefatto (He, a pressione di $10^{-1} \; \si{\Pa}$) presente nell'ampolla,
i quali emettono la radiazione che consente di visualizzare il pennello elettronico e
misurarne l'orbita.

Una volta liberati dal catodo, nella regione in cui supponiamo assente il
campo elettrico $V\ped{acc}$, per la conservazione dell'energia vale
\begin{equation}\label{eq: T=qV}
    \frac{1}{2} m_{e} v^{2} = e V\ped{acc}
\end{equation}
Per cui, assumendo che il campo magnetico sia statico e uniforme lungo $z$ e
che il fascio di elettroni abbia velocità ortogonale all'asse delle spire,
ci aspettiamo che gli elettroni rimangano in moto circolare uniforme nel
piano ortogonale $x-y$.

Dalla condizione di moto circolare di raggio $R$ dovuto alla forza di Lorentz
abbiamo che
\[
m_{e} \frac{v^2}{R} = e v B \implies v = \frac{e}{m_e} B R
\]
Combinando l'~\cref{eq: T=qV} con la precedente troviamo
\[
v^2 = 2 V\ped{acc} \frac{e}{m_e} \implies \left(\frac{e}{m_e} B R\right)^2 =
2 V\ped{acc} \frac{e}{m_e}
\]
Da cui otteniamo l'equazione tramite cui vogliamo stimare il rapporto
\begin{equation}\label{eq: fit}
\frac{e}{m_{e}} = \frac{2 \Delta V}{(BR)^2}.
\end{equation}

Dal momento che tutte le variabili nel RHS sono direttamente controllabili
configurando le tensioni di alimentazione e possiamo misurare il raggio
della traiettoria $R$ analizzando (come faremo ad esempio con un fit
circolare) le fotografie del moto nel bulbo.

%=======================
\section{Descrizione delle misure}
\subsection{Orientazione delle bobine rispetto al campo magnetico terrestre}
Usando le due bussole in dotazione e la bussola di un cellulare abbiamo per
prima cosa orientato l'apparato in modo che il campo magnetico generato dalle
bobine fosse nella stessa direzione del campo magnetico generato dalle bobine: in questo modo non dovremo preoccuparci di componenti del campo magnetico non perpendicolari alla direzione degli elettroni, che avrebbero contribuito rendendo il moto elicoidale.
\subsection{Mappatura del campo magnetico lungo l'asse delle bobine}
Si vuole a questo punto misurare la variazione d'intensità del campo magnetico lungo l'asse delle bobine: per farlo ci serviamo di una sonda ad effetto hall con fattore di conversione di $5.0 \pm 0.1 \frac{mV}{G} = 50 \pm 1 \frac{V}{T}$.
A questo punto abbiamo alimentato le bobine, facendo attenzione a non utilizzare più dei valori massimi consigliati dal costruttore (2 A di corrente); abbiamo quindi misurato il valore della corrente di alimentazione:
\[
I_{coil}=2.00 \pm 0.03 A
\]
Facendo un veloce calcolo con l'equazione \ref{eq: Bmax} e con il valore misurato di $I_{coil}$ ci si aspetta che l'intensità del campo magnetico massimo misurato dovrà essere $1.48 \pm 0.02 \si{mT}$; se ne deduce che quindi il sensore Hall da solo non è sufficiente per misurare approfonditamente le variazioni del campo magnetico (il sensore produrrebbe una d.d.p di circa 70 mV al massimo).
Si è quindi inserito tra la sonda e il multimetro un amplificatore-sottrattore (da calibrare) con un coefficiente di amplificazione pari a $11.1 \pm 0.1$.
Dato che l'errore sul valore letto dal multimetro è circa $0.7 \percent + 1 digit$ contro al $2 \percent + 1 \percent$ derivante dal circuito sonda + Amplificatore, lo abbiamo trascurato, prendendo in considerazione solo quelli del circuito sopracitato.\\
Il primo passo è stato quello di calibrare il circuito amplificatore-sottrattore, aggiustando l'offset, in modo che il multimetro misurasse lo stesso valore ma di segno invertito ruotando la sonda di $180 \degree$; da questa procedura ci accorgiamo che la sonda misura un campo magnetico residuo, infatti dal multimetro leggiamo una tensione di circa 19 mV, che convertendoli ci danno come risultato un campo magnetico di circa $34.2 \pm 0.7 \mu T$, compatibile con l'intensità del campo magnetico terrestre. Ci accorgiamo che questo contributo, che non è quindi trascurabile, andrà ad aumentare il campo magnetico misurato di circa il $2 \percent$, rendendo la nuova aspettativa di $1.51 \pm 0.02$.\\
Successivamente è stato acceso il generatore che alimenta le bobine e si è inserita la sonda nell'apposita guida rettilinea situata nel punto medio tra le bobine, tagliandole lungo il diametro. Dunque abbiamo misurato il valore in uscita dal multimetro all'avanzare a ogni centimetro della sonda.
Considerando che il campo magnetico massimo misurato, trovato in corrispondenza del centro delle spire, è pari a $1.47 \pm 0.03 \;mT$ possiamo affermare che il valore è compatibile con le aspettative entro le 2 barre di errore.
\begin{figure}
\includegraphics[width=\textwidth]{sounasegaiochiamalocomecazzotiparetantocifounasegaio}
\caption{Andamento misurato per l'intensità del campo magnetico nel piano centrale tra le spire in funzione della distanza dal centro}
\end{figure}
\begin{figure}
\includegraphics[width=\textwidth]{riferimentoB}
\caption{Andamento previsto per l'intensità del campo magnetico nel piano centrale tra le spire in funzione della distanza dal centro}
\end{figure}
La mappatura delle intensità del campo magnetico risulta quindi qualitativamente compatibile con quanto aspettato.
\subsection{Calibrazione dell'apparato per l'acquisizione delle traiettorie}
\label{sec: conv}
A questo punto si è potuto procedere con l'acquisizione delle misure del raggio.
Per farlo abbiamo fatto uso di una fotocamera la quale è stata posizionata in modo tale che il centro delle lenti corrispondesse al centro delle bobine e del bulbo di vetro: in questo modo le distorsioni dovute al bulbo saranno minimizzate, di conseguenza sarà più facile stimare il raggio effettivo; inoltre se prendiamo come traiettoria quella di raggio tale che il suo centro che passa per il centro del bulbo, le distorsioni saranno ancora più minimizzate, e dalla fotocamera vedremo una circonferenza meno distorta.
Come impostazioni per la fotocamera abbiamo utilizzato un ISO di 400 e un'esposizione di 10 secondi, in modo da vedere al meglio i limiti interni ed esterni del pennello elettronico e stimare con maggiore precisione il raggio della traiettoria.\\
Per cominciare abbiamo scattato una foto in cui si vedono 2 righelli (risoluzione 1 mm) posizionati ognuno in una bobina, questo per valutare gli effetti della geometria proiettiva (e ottenere una scala di conversione pixel-cm) e per valutare successivamente gli effetti di distorsione del bulbo.
\begin{figure}
\includegraphics[width=\textwidth]{cal1}
\caption{Foto di calibrazione in assenza di bulbo, da questa ricaveremo una scala per convertire successivamente i dati da pixel a cm}
\end{figure}
Abbiamo quindi alimentato il dispositivo di emissione di elettroni con una tensione di 6 Volt e abbiamo aspettato che il filamento si scaldasse il più possibile, dopodiché abbiamo iniziato a variare la tensione di accelerazione: inizialmente, nonostante si variasse $V_{acc}$, il pennello elettronico non era visibile in quanto non avevamo aspettato ancora abbastanza e il filamento doveva ancora raggiungere la temperatura ottimale. Dopo circa 10 minuti (in totale) il pennello elettronico aveva raggiunto lo stato di massima luminosità, quindi si sono spente le luci e abbiamo scattato diverse fotografie a diverse tensioni di accelerazioni e correnti $I_{coil}$; come ci si aspettava aumentando la tensione di accelerazione aumenta il raggio di curvatura, diminuendo $I_{coil}$ aumenta il raggio.
\subsection{Misura del raggio della traiettoria}
Utilizzando le foto sopra menzionate si è effettuato un campionamento dei punti
sull'arco interno e sull'arco esterno delle traiettorie (dato che il pennello elettronico ha uno spessore non trascurabile). Le coordinate dei pixel così ricavate
sono state interpolate con un \emph{fit} circolare per ottenere una stima del
raggio interno ed esterno. Si è poi assunto come valore efficace del raggio
dell'orbita la media del raggio del cerchio interno e di quello esterno,
e si è attribuito un errore pari alla semi-dispersione degli stessi.
I raggi così ottenuti sono stati poi convertiti in unità fisiche come spiegato
nella~\cref{sec: conv}. \\

\section{Analisi dati e stima del rapporto $e/m$}
La stima del rapporto $ e/m_{e} $ è stata poi ottenuta in due modi diversi:
come media pesata delle singole misure ottenute dalla~\eqref{eq: fit} ed
effettuando un \emph{fit} lineare di $ 2\Delta V $ al variare di
$ (B R)^{2} $ e ottenendo $ e/m_{e} $ dal coefficiente angolare della retta
di \emph{best-fit}.

Prendendo come valori esatti $e = \SI{1.602176634e11}{\coulomb}$ e
$m_{e} = \SI{9.10938370e-31}{\kilogram}$ il valore atteso per il loro
rapporto è
\begin{equation}\label{eq: e-m-exp}
\left(\frac{e}{m_{e}}\right)\ped{exp} = \SI{175.882e9}{\coulomb/\kilogram}
\end{equation}

%=======================
\section{Valutazione degli effetti sistematici}

\subsection{Spessore del pennello elettronico}

\subsection{Dipendenza della stima di $e/m$ dal raggio dell'orbita $R$}

\subsection{Distorsione del bulbo di vetro}

\subsection{Campo magnetico terrestre}

\subsection{Disuniformità del campo magnetico sulla traiettoria}

%=======================
\section*{Conclusioni e commenti finali}
Si è riusciti a dare una misura ragionevole del rapporto carica/massa
dell'elettrone a partire da un'analisi delle fotografie della sua traiettoria
elicoidale in presenza di un campo magnetico uniforme.

%=======================
\section*{Dichiarazione}
I firmatari di questa relazione dichiarano che il contenuto della relazione \`e
originale, con misure effettuate dai membri del gruppo, e che tutti i firmatari
hanno contribuito alla elaborazione della relazione stessa.

%=======================
\begin{thebibliography}{1}
\bibitem{Coope}{I. D. Coope, Circle fitting by linear and nonlinear least
squares, Department of Mathematics, University of Canterbury, Christchurch,
New Zealand, N.60, May, 1992,
\url{https://ir.canterbury.ac.nz/bitstream/handle/10092/11104/coope_report_no69_1992.pdf?sequence=1&isAllowed=y}}
\end{thebibliography}

\end{document}
