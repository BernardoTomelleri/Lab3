\documentclass[10pt, a4paper, italian]{article}
\usepackage[T1]{fontenc}
\usepackage[utf8]{inputenc}
\usepackage{amsmath, amssymb, amsthm, thmtools, amsfonts, mathtools}
\usepackage{nicefrac}
\usepackage{calc}
\usepackage[pdftex, hyperindex, plainpages=false]{hyperref}
\usepackage[nameinlink]{cleveref} %load before classicthesis (clash)
%\usepackage[nochapters,pdfspacing]{classicthesis}
\usepackage{siunitx}
\usepackage[siunitx]{circuitikz}

\usepackage[a4paper]{geometry}
\usepackage{float}
\usepackage{mdframed}
\usepackage{titling}
\usepackage{booktabs}
\usepackage{graphicx}
\usepackage{caption, subcaption}
\usepackage{xcolor}
\usepackage[italian]{babel}
\usepackage{pgfplots}
\usepackage{listings}
%\usepackage{lmodern}
\usepackage{url}
\usepackage{enumitem}
\usepackage{tikz} %loads after classicthesis (xcolor incompat)

% lets graphicx know path where figures to be included are found
\graphicspath{{../figs/}}
\makeatletter
\def\input@path{{../figs/}}
%or: \def\input@path{{/path/to/folder/}{/path/to/other/folder/}}
\makeatother

% tikz pgf plots setup
\usepgfplotslibrary{external}
\pgfplotsset{compat=1.15}
\tikzexternalize

% spaces and significant digits/figures for measurements
\sisetup{free-standing-units, space-before-unit, number-unit-product = \;,
scientific-notation = true, round-mode = figures, round-precision = 2,}

% turns all (hyperlinked) references black [default is blue]
\hypersetup{
	linktoc=all,
	colorlinks=true,
	linkcolor=black
}

% code listings config
\lstset{
language=Python,
basicstyle=\ttfamily,
columns=fullflexible,
keepspaces=true,
}

% mdframed (for boxed text) configuration
\mdfsetup{linewidth=0.6pt}

% Default fixed font does not support bold face
\DeclareFixedFont{\ttb}{T1}{txtt}{bx}{n}{12} % for bold
\DeclareFixedFont{\ttm}{T1}{txtt}{m}{n}{12}  % for normal

% Custom colors
\usepackage{color}
\definecolor{deepblue}{rgb}{0,0,0.5}
\definecolor{deepred}{rgb}{0.6,0,0}
\definecolor{deepgreen}{rgb}{0,0.5,0}

% Commands 
\newcommand{\executeiffilenewer}[3]{%
	\ifnum\pdfstrcmp{\pdffilemoddate{#1}}%
		{\pdffilemoddate{#2}}>0%
	{\immediate\write18{#3}}\fi%
}
% input .svg --> .pdf_tex graphs
\newcommand{\includesvg}[1]{%
	\executeiffilenewer{#1.svg}{#1.pdf}%
	{inkscape -z -D --file=#1.svg %
	--export-pdf=#1.pdf --export-latex}%
	\input{#1.pdf_tex}%
}
% Thanks UniPi's Department of Physics E. Fermi
\newcommand{\thanksdf}{(\thanks{Dipartimento di Fisica E.~Fermi,%
Universit\`a di Pisa - Pisa, Italy.}\;)}

% hyperlink to email address
\newcommand{\mail}[1]{\href{mailto:#1}{\textsf{#1}}}

\input{../../latex/math}
\geometry{left=2cm, right=2cm, top=2cm, bottom=2cm}

% indexes subsections with letters, sections with numbers (1.a, 1.b, ...)
\renewcommand{\thesubsection}{\thesection.\alph{subsection}}

% lets graphicx know path where figures to be included are found
\graphicspath{{../figs/}}

\author{Gruppo 1.AC \\ Matteo Rossi, Bernardo Tomelleri}
\title{Es08A: Amplificatore JFET}
\begin{document}
\date{\today}
\maketitle

\setcounter{section}{0}

\section*{Misura componenti dei circuiti}
\begin{table}[htbp]
\centering
\begin{tabular}{cc}
\midrule
$R_1$	  	& $100 \pm 1 \si{\ohm}$\\
$C_{in}$	  	& $99 \pm 4 \si{\F}$	\\
$C_{out}$	& $9.6 \pm 0.4 \si{n\F}$	\\
$C_E$	  	& $88 \pm 5 \si{\micro \F}$\\
$R_s$	  	& $217 \pm 3 \si{\ohm}$	\\
$R_d$	  	& $996 \pm 8 \si{\ohm}$\\
$R_g$	  	& $1.02 \pm 0.1 \si{M\ohm}$\\
$R_S$	  	& $99.6 \pm 0.8 \si{k\ohm}$\\

\bottomrule     
\end{tabular}
\caption{Valori di resistenza e capacità misurate per i componenti dei
circuiti studiati. \label{tab: rcmes_B}}
\end{table}

\begin{table}[H]
\centering
\begin{tabular}{cc}
\midrule
$R_1$	  	& $100 \pm 1 \si{\ohm}$\\
$C_{in}$	  	& $99 \pm 4 \si{\F}$	\\
$C_{out}$	& $9.6 \pm 0.4 \si{n\F}$	\\
$C_E$	  	& $88 \pm 5 \si{\micro \F}$\\
$R_s$	  	& $217 \pm 3 \si{\ohm}$	\\
$R_d$	  	& $996 \pm 8 \si{\ohm}$\\
$R_g$	  	& $1.02 \pm 0.1 \si{M\ohm}$\\
$R_S$	  	& $99.6 \pm 0.8 \si{k\ohm}$\\

\bottomrule     
\end{tabular}
\caption{Valori di resistenza e capacità misurate per i componenti dei
circuiti studiati. \label{tab: rcmes_B}}
\end{table}

\subsection*{Nota sul metodo di fit}
Per determinare i parametri ottimali e le rispettive covarianze si \`e
implementato in \verb+Python+ un algoritmo di fit basato sui minimi quadrati
mediante la funzione \emph{curve\_fit} della libreria \texttt{SciPy}.

%=======================
\section{Studio del funzionamento}
Come primo passo abbiamo verificato il corretto funzionamento del JFET utilizzando lo schema seguente:
\begin{figure}[H]
    \centering
	\includegraphics[scale=0.5]{Draft1}
    \caption{Schema circuitale per la verifica di funzionamento del JFET}
\end{figure}
Vista la struttura del jfet, sappiamo che aumentando $V_{GS}$ le zone della giunzione vengono svuotate dai portatori di carica, fino a che non si raggiunge un potenziale di pinch-off $V_p$ in cui il canale risulta completamente svuotato e la corrente di drain tende a 0. Al contrario invece, quando $V_{GS}=0$ il canale risulta "aperto", perciò misureremo in questa situazione il massimo di corrente; quando invece $V_{GS}>V_p$ la corrente sarà pressoché nulla.
Abbiamo quindi inviato a $V_{SS}$ una tensione continua di -4.75 V, a WG1 una rampa a scalini di 250 mV partendo da -4.75V fino a 0 V, e a WG2 che per ogni gradino fatto da WG1 genera una rampa che parte dal valore corrente di WG1 e arriva fino a 5V.
Di seguito quello che otteniamo dall'oscilloscopio:
\begin{figure}[H]
    \centering
	\includegraphics[scale=0.4]{time}
    \caption{Grafici di CH1, CH2 e math1 (definito come CH2/R1) in funzione del tempo a sinistra, grafico di math1 in funzione di CH1 a destra}
\end{figure}
Come detto prima, il momento in cui la corrente è maggiore (nel grafico a sinistra è sufficiente soffermarsi a vedere l'andamento di CH2) si ottiene quando $V_{GS}$ è pari a 0, che nello stesso grafico è quando la rampa di WG2 misurata da CH1 è più alta (perchè in quel caso $V_S$ è pari a $V_G$ ovvero $V_{SS}$.
Inoltre si può vedere che oltre un certo punto l'andamento di CH2 risulta approssimativamente costante: questo si ottiene quando viene superata la tensione di pinch-off, che siamo andati a misurare tramite cursori:
\[
V_p=2.62 \pm 0.03 V
\]
Infine, sempre utilizzando i cursori, abbiamo misurato la corrente nella traccia in cui $V_{GS}=0$, nel grafico di sinistra è la curva più alta; da cui abbiamo ricavato:
\[
I_{dss}=9.4 \pm 0.1 mA
\]
Confrontando col datasheet risulta che entrambi i valori risultano compatibili con gli intervalli dichiarati dai costruttori (dato che noi abbiamo utilizzato valori di $V_{DS}$ minori di quelli utilizzati nel datasheet)
\subsection{Amplificatore e punto di lavoro}
A questo punto abbiamo costruito il circuito per l'amplificatore di tensione:
\begin{figure}[htbp]
    \centering
	\includegraphics[scale=0.7]{Draft2}
    \caption{Schema circuitale per l'amplificazione di segnale tramite JFET}
\end{figure}
Quindi si è collegato $V_{CC}$ a 5V e $V_{EE}$ a -5V tenendo scollegato $V_{in}$ per verificare il punto di lavoro del JFET.
Misurando la Caduta di potenziale ai capi della resistenza $R_D$ abbiamo calcolato la corrente di quiescenza con la legge di ohm, da cui si ricava $I_{ds}=4.09 \pm 0.04 mA$ che risulta essere circa la metà della $I_{dss}$.
Si è quindi proseguito con la misura di $V_{GS}$ e di $V_DS$ per verificare quello che abbiamo appena misurato.
Sappiamo infatti che dato $V_{GS}$ e $V_p$ e se il JFET è in regime di saturazione:
\begin{equation}
I_{ds}=\frac{I_{dss}}{V_p ^2}(V_{GS}-V_p)^2
\end{equation}

%=======================
\section{Amplificatore del Noise rispetto al Set}


%=======================
\section{Controllo integrale}

%=======================
\section{Controllo proporzionale}

%=======================
\section*{Conclusioni e commenti finali}


%=======================
\section*{Dichiarazione}
I firmatari di questa relazione dichiarano che il contenuto della relazione \`e
originale, con misure effettuate dai membri del gruppo, e che tutti i firmatari
hanno contribuito alla elaborazione della relazione stessa.


\end{document}
