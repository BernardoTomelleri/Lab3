\documentclass[10pt, a4paper, italian]{article}
\usepackage[T1]{fontenc}
\usepackage[utf8]{inputenc}
\usepackage{amsmath, amssymb, amsthm, thmtools, amsfonts, mathtools}
\usepackage{nicefrac}
\usepackage{calc}
\usepackage[pdftex, hyperindex, plainpages=false]{hyperref}
\usepackage[nameinlink]{cleveref} %load before classicthesis (clash)
%\usepackage[nochapters,pdfspacing]{classicthesis}
\usepackage{siunitx}
\usepackage[siunitx]{circuitikz}

\usepackage[a4paper]{geometry}
\usepackage{float}
\usepackage{mdframed}
\usepackage{titling}
\usepackage{booktabs}
\usepackage{graphicx}
\usepackage{caption, subcaption}
\usepackage{xcolor}
\usepackage[italian]{babel}
\usepackage{pgfplots}
\usepackage{listings}
%\usepackage{lmodern}
\usepackage{url}
\usepackage{enumitem}
\usepackage{tikz} %loads after classicthesis (xcolor incompat)

% lets graphicx know path where figures to be included are found
\graphicspath{{../figs/}}
\makeatletter
\def\input@path{{../figs/}}
%or: \def\input@path{{/path/to/folder/}{/path/to/other/folder/}}
\makeatother

% tikz pgf plots setup
\usepgfplotslibrary{external}
\pgfplotsset{compat=1.15}
\tikzexternalize

% spaces and significant digits/figures for measurements
\sisetup{free-standing-units, space-before-unit, number-unit-product = \;,
scientific-notation = true, round-mode = figures, round-precision = 2,}

% turns all (hyperlinked) references black [default is blue]
\hypersetup{
	linktoc=all,
	colorlinks=true,
	linkcolor=black
}

% code listings config
\lstset{
language=Python,
basicstyle=\ttfamily,
columns=fullflexible,
keepspaces=true,
}

% mdframed (for boxed text) configuration
\mdfsetup{linewidth=0.6pt}

% Default fixed font does not support bold face
\DeclareFixedFont{\ttb}{T1}{txtt}{bx}{n}{12} % for bold
\DeclareFixedFont{\ttm}{T1}{txtt}{m}{n}{12}  % for normal

% Custom colors
\usepackage{color}
\definecolor{deepblue}{rgb}{0,0,0.5}
\definecolor{deepred}{rgb}{0.6,0,0}
\definecolor{deepgreen}{rgb}{0,0.5,0}

% Commands 
\newcommand{\executeiffilenewer}[3]{%
	\ifnum\pdfstrcmp{\pdffilemoddate{#1}}%
		{\pdffilemoddate{#2}}>0%
	{\immediate\write18{#3}}\fi%
}
% input .svg --> .pdf_tex graphs
\newcommand{\includesvg}[1]{%
	\executeiffilenewer{#1.svg}{#1.pdf}%
	{inkscape -z -D --file=#1.svg %
	--export-pdf=#1.pdf --export-latex}%
	\input{#1.pdf_tex}%
}
% Thanks UniPi's Department of Physics E. Fermi
\newcommand{\thanksdf}{(\thanks{Dipartimento di Fisica E.~Fermi,%
Universit\`a di Pisa - Pisa, Italy.}\;)}

% hyperlink to email address
\newcommand{\mail}[1]{\href{mailto:#1}{\textsf{#1}}}

\input{../../latex/math}
\geometry{left=2cm, right=2cm, top=2cm, bottom=2cm}

% indexes subsections with letters, sections with numbers (1.a, 1.b, ...)
\renewcommand{\thesubsection}{\thesection.\alph{subsection}}

% lets graphicx know path where figures to be included are found
\graphicspath{{../figs/}}

\author{Gruppo 1.AC \\ Matteo Rossi, Bernardo Tomelleri}
\title{Es08A: Amplificatore di tensione con JFET a emettitore comune}
\begin{document}
\date{\today}
\maketitle

\setcounter{section}{0}

\section{Misura componenti dei circuiti}
\begin{table}[htbp]
\centering
\begin{tabular}{cccccc}
\toprule
Resistenze $[\si{\ohm}]$ & $R$ & $\sigma R$ & Capacità $[\si{\F}]$ & $C$ &
$\sigma C$ \\
\midrule
\midrule
$R_1$	  & 100 	& 1 	 & $C\ped{in}$ & 95	n	 & 4 n \\
$R_S$	  & 219	& 3 	 & $C\ped{out}$ & 9.6 n	 & 0.4 n \\
$R_D$	  & 997		& 8	 	 & $C_S$ 		& 95  µ  & 5 µ \\
$R_G$	  & 1.02 M	& 0.1 M	 & & & \\
$R_s$	  & 99.6 k	& 0.8 k	 & & & \\
\bottomrule     
\end{tabular}
\caption{Valori di resistenza e capacità misurate per i componenti dei
circuiti studiati. \label{tab: rcmes_M}}
\end{table}

\begin{table}[htbp]
\centering
\begin{tabular}{cccccc}
\toprule
Resistenze $[\si{\ohm}]$ & $R$ & $\sigma R$ & Capacità $[\si{\F}]$ & $C$ &
$\sigma C$ \\
\midrule
\midrule
$R_1$	  & 100.2 	& 0.9 	 & $C\ped{in}$ & 99	n	 & 4 n \\
$R_s$	  & 217	& 3 	 & $C\ped{out}$ & 10.4 n	 & 0.4 n \\
$R_d$	  & 993		& 8	 	 & $C_S$ 		& 96  µ  & 4 µ \\
$R_g$	  & 994 k	& 8	 & & & \\
$R_s$	  & 996 k	& 8	 & & & \\
\bottomrule     
\end{tabular}
\caption{Valori di resistenza e capacità misurate per i componenti dei
circuiti studiati. \label{tab: rcmes_B}}
\end{table}

\subsection*{Nota sul metodo di fit}
Per determinare i parametri ottimali e le rispettive covarianze si \`e
implementato in \verb+Python+ un algoritmo di fit basato sui minimi quadrati
mediante la funzione \emph{curve\_fit} della libreria \texttt{SciPy}.

%=======================
\section{Studio del funzionamento}
Come primo passo abbiamo verificato il corretto funzionamento del JFET a canale
N 2N3819 studiandone le caratteristiche dalla configurazione circuitale
riportata in \ref{schm: tracer}
\begin{figure}[htbp]
    \centering
	\includegraphics[scale=0.5]{Draft1}
    \caption{Schema circuitale per la verifica di funzionamento del JFET
    \label{schm: tracer}}
\end{figure}

Vista la struttura del JFET, sappiamo che aumentando $V_{GS}$ le zone della giunzione vengono svuotate dai portatori di carica, fino a che non si raggiunge un potenziale di pinch-off $V_p$ in cui il canale risulta completamente svuotato e la corrente di drain $I_{DS}$ tende a 0. Al contrario invece, quando $V_{GS} = \SI{0}{\V}$ il canale risulta "aperto", perciò misureremo in questa situazione il massimo di corrente; quando invece $V_{GS} < V_p$ la corrente sarà pressoché nulla.

\subsection{Curve tracer}
Abbiamo quindi inviato a $V_{SS}$ una tensione continua di -5 V, a WG1 una rampa a scalini di 250 mV partendo da -5V fino a 0 V, e a WG2 che per ogni gradino fatto da WG1 genera una rampa che parte dal valore corrente di WG1 e arriva fino a 5V.
Di seguito quello che otteniamo dall'oscilloscopio:
\begin{figure}[htbp]
    \centering
	\includegraphics[scale=0.4]{timey}
    \caption{Acquisizione all'oscilloscopio di $V_{DS}$ (CH1), $V_{R_1}$ (CH2)
    e $I_{DS} = V_{R_1}/R_1$ in funzione del tempo}
\end{figure}
Notiamo esplicitamente come $V_{DS}$ risulta sempre positivo, mentre $V_{GS}$
è sempre negativo, in accordo con le condizioni $V_{DS} > 0$ e $V_{GS} < 0$
da verificare.

\subsection{Curve caratteristiche ottenute}
\begin{figure}[htbp]
    \centering
	\includegraphics[scale=0.4]{xy}
    \caption{Curve caratteristiche del jfet per $I_{DS}$ su $V_{DS}$ col primo JFET}
\end{figure}
\begin{figure}[htbp]
    \centering
	\includegraphics[scale=0.4]{vgs}
    \caption{Curve caratteristiche del jfet per $I_{DS}$ su $V_{DS}$ col secondo JFET}
\end{figure}
\subsection{Confronto con datasheet}
Come detto prima, il momento in cui la corrente è maggiore (nel grafico a sinistra è sufficiente soffermarsi a vedere l'andamento di CH2) si ottiene quando $V_{GS}$ è pari a 0, che nello stesso grafico è quando la rampa di WG2 misurata da CH1 è più alta (perchè in quel caso $V_S$ è pari a $V_G$ ovvero $V_{SS}$.
Inoltre si può vedere che oltre un certo punto l'andamento di CH2 risulta approssimativamente costante: questo si ottiene quando viene superata la tensione di pinch-off, che siamo andati a misurare tramite cursori:
\[
V_p = -3.0 \pm 0.2 \; \si{\V}
\]
\[
V_p = -4.0 \pm 0.2 \; \si{\V}
\]
in cui l'incertezza associata sarà uguale al passo dei nostri scalini.
Infine, sempre utilizzando i cursori, abbiamo misurato la corrente nella traccia in cui $V_{GS} = 0$, nel grafico di sinistra è la curva più alta; da cui abbiamo ricavato:
\[
I_{DSS} = 8.62 \pm 0.07 \; \si{m\A}
\]
\[
I_{DSS} = 12.6 \pm 0.2 \; \si{m\A}
\]
Confrontando col datasheet risulta che entrambi i valori risultano compatibili con gli intervalli dichiarati dai costruttori (dato che noi abbiamo utilizzato valori di $V_{DS}$ minori di quelli utilizzati nel datasheet)

\section{Amplificatore e punto di lavoro}
A questo punto abbiamo costruito il circuito per l'amplificatore di tensione:
\begin{figure}[htbp]
    \centering
	\includegraphics[scale=0.7]{Draft2}
    \caption{Schema circuitale per l'amplificazione di segnale tramite JFET}
\end{figure}
Quindi si è collegato $V_{CC}$ a 5V e $V_{SS}$ a -5V tenendo scollegato $V\ped{in}$ per verificare il punto di lavoro del JFET.
\subsection{Corrente di quiescenza}
Misurando la Caduta di potenziale ai capi della resistenza $R_D$ abbiamo calcolato la corrente di quiescenza con la legge di ohm, da cui si ricava $I_{DS}=4.06 \pm 0.05 \; \si{m\A}$ e $I_{DS}=6.42 \pm 0.08 \; \si{m\A}$ che risultano essere circa la metà delle rispettive $I_{DSS}$.
Si è quindi proseguito con la misura di $V_{GS}$ e di $V_{DS}$ per verificare quello che abbiamo appena misurato.
Sappiamo infatti che dato $V_{GS}$ e $V_p$ e se il JFET è in regime di saturazione:
\begin{equation}
I_{DS}=\frac{I_{DSS}}{V_p ^2}(V_{GS}-V_p)^2
\end{equation}
Per misurare $V_{GS}$ abbiamo si è calcolata la differenza tra le misure di $V_G$ e $V_S$:
\[
V_{GS}=-886 \pm 7 \; \si{m\V}
\]
\[
V_{DS}=2.35 \pm 0.01 \; \si{\V}
\]
mentre per il secondo JFET risulta:
\[
V_{GS}=-994 \pm 8 \; \si{m\V}
\]
\[
V_{DS}=2.10 \pm 0.02 \; \si{\V}
\]
Dato che $V_{DS} > V_{GS} - V_p$ si ricava che siamo in zona di saturazione nel caso del primo jfet, mentre per il secondo siamo ancora in zona ohmica. A questo punto basta applicare l'equazione solo per il primo; si ottiene quindi 
\[
I_{DS}(V_{GS},V_p,I_{DSS})=4.3\pm 0.4 \; \si{m\A}
\]

Che risulta essere compatibile con quanto misurato.
Dato che il secondo JFET non sembra essere entrato in saturazione applichiamo la seguente formula per stimare $I_{DS}$ nella regione ohmica:
\begin{equation}
I_{DS}=\frac{I_{DSS}}{V_P^2}[2(V_{GS}-V_P) - V_{DS}]V_{DS}
\end{equation}
da cui si ottiene $I_{DS}= 6.47 \pm 0.06 \; mA$, che risulta perfettamente compatibile con la misura.
Stimiamo infine la transconduttanza tramite la formula
\begin{equation}
g_m = \frac{2 I_{DSS}}{|V_p|}\sqrt{\frac{I_{d}}{I_{DSS}}}
\end{equation}
da cui ricaviamo $g_m = 3.94 \si{m\mho}$ e $g_m = 4.51 \si{m\mho}$.
A questo punto abbiamo invece provato a misurarla dal grafico, facendo il rapporto tra la differenza nella corrente $I_{DS} $ e la differenza di potenziale di $V_{GS}$ tra le 2 curve associate più vicine al punto di lavoro (vista la "risoluzione" di 250 mV abbiamo preso come punti per effettuare la misura $V_{GS}=-0.75 V$ e $V_{GS}= -1 V$) e a parità di $V_{DS}$ da cui abbiamo ricavato
\[
g_m=3.92\pm 0.05 \; \si{m\mho}
\]
\[
g_m=4.49\pm 0.05 \; \si{m\mho}
\]
Successivamente siamo andati a ricercare nel datasheet il valore fornito da costruttore, troviamo un grafico di $g_m$ in funzione della frequenza a cui opera il JFET, in particolare notiamo che per frequenze minori di circa 500 Mhz la transconduttanza deve essere compresa approssimativamente tra 4 e 5, similarmente anche la transammettenza deve essere compresa tra 3 e 6.5 $\si{m\mho}$; i valori misurati sono quindi compatibili con quelli nel datasheet e con quelli ricavati usando l'equazione (2).
%=======================
\section{Amplificatore di segnale}
A questo punto si è collegato l'ingresso $V\ped{in}$ all'entrata del primo amplificatore (dato che il secondo nonostante i nostri tentativi non entrasse in conduzione), e lo abbiamo pilotato con un'onda sinusoidale di 1 kHz con un'ampiezza compresa tra 100 mV e 2.8 V a passi di 100 mV. Dall'ampiezza della risposta in uscita $V\ped{out}$ abbiamo misurato il guadagno $A_v = \frac{V\ped{in}}{V\ped{out}}$
\begin{table}[H]
\centering
\begin{tabular}{cccccc}
\toprule
$V\ped{in} [V]$ & $\sigma V\ped{in}[V]$ & $V\ped{out} [V]$ & $\sigma V\ped{out} [V]$ & $|A_v|$ & $\sigma A_v$ \\
\midrule
\midrule
100 m & 1 m  & 206 m & 2 m   & 2.06 & 0.02 \\
200 m & 2 m  & 411 m & 4 m   & 2.05 & 0.02 \\
300 m & 3 m  & 616 m & 7 m  & 2.05 & 0.03 \\
400 m & 4 m  & 821 m & 7 m  & 2.05 & 0.02 \\
501 m & 4 m  & 1.03  & 8 m  & 2.05 & 0.02 \\
601 m & 7 m  & 1.22  & 18 m & 2.04 & 0.04 \\
701 m & 7 m  & 1.42  & 19 m & 2.03 & 0.03 \\
801 m & 7 m  & 1.62  & 20 m & 2.02 & 0.03 \\
901 m & 7 m  & 1.80  & 0.02 & 2.00 & 0.03 \\
1.00  & 8 m  & 1.99  & 0.02 & 1.99 & 0.03 \\
1.10  & 8 m  & 2.16  & 0.02 & 1.96 & 0.02 \\
1.20  & 8 m  & 2.31  & 0.02 & 1.93 & 0.02 \\
1.30  & 9 m  & 2.46  & 0.02 & 1.89 & 0.02 \\
1.40  & 9 m  & 2.59  & 0.03 & 1.86 & 0.02 \\
1.50  & 0.02 & 2.71  & 0.04 & 1.82 & 0.03 \\
1.60  & 0.02 & 2.84  & 0.04 & 1.78 & 0.03 \\
1.70  & 0.02 & 2.93  & 0.04 & 1.72 & 0.03 \\
1.80  & 0.02 & 3.05  & 0.04 & 1.69 & 0.03 \\
1.90  & 0.02 & 3.15  & 0.04 & 1.65 & 0.03 \\
2.00  & 0.02 & 3.25  & 0.04 & 1.62 & 0.03 \\
2.10  & 0.02 & 3.34  & 0.04 & 1.59 & 0.02 \\
2.20  & 0.02 & 3.48  & 0.04 & 1.56 & 0.02 \\
2.30  & 0.02 & 3.53  & 0.04 & 1.53 & 0.02 \\
2.40  & 0.02 & 3.59  & 0.04 & 1.49 & 0.02 \\
2.50  & 0.02 & 3.61  & 0.04 & 1.44 & 0.02 \\
2.60  & 0.03 & 3.63  & 0.04 & 1.39 & 0.02 \\
2.71  & 0.03 & 3.63  & 0.04 & 1.34 & 0.02 \\
2.81  & 0.04 & 3.64  & 0.04 & 1.29 & 0.02 \\
\bottomrule
\end{tabular}
\end{table}
Data la caratteristica del circuito di essere in configurazione common source, possiamo modellare il guadagno come
\begin{equation}
A_v = -\frac{g_m R_d}{1+g_mR_s}
\end{equation}
Utilizzando i dati ottenuti prima possiamo ricavare il valore atteso:
\[
A_v = -2.10 \pm 0.03
\]
Ci accorgiamo subito che questo risultato risulta essere perfettamente compatibile con quanto misurato (a piccole ampiezze di segnale).
Essendo il guadagno negativo se ne deduce che l'amplificatore sarà di tipo invertente, cosa che si vede immediatamente dai grafici di $V\ped{in}$ e $V\ped{out}$.
\begin{figure}[htbp]
    \centering
	\includegraphics[scale=0.4]{amp.200}
    \caption{Grafici di $V\ped{in}$ e $V\ped{out}$ in funzione del tempo con ampiezza $V\ped{in}=200 \; \si{m\V}$}
\end{figure}
\begin{figure}[htbp]
    \centering
	\includegraphics[scale=0.4]{amp.1000}
    \caption{Grafici di $V\ped{in}$ e $V\ped{out}$ in funzione del tempo con ampiezza $V\ped{in}=1000 \; \si{m\V}$; in questa situazione si nota una distorsione nel segnale in uscita, in particolare la parte inferiore dell'onda risulta schiacciata verso lo 0}
\end{figure}
\begin{figure}[htbp]
    \centering
	\includegraphics[scale=0.4]{amp.2000}
    \caption{Grafici di $V\ped{in}$ e $V\ped{out}$ in funzione del tempo con ampiezza $V\ped{in}=2000 \; \si{m\V}$; qua la distorsione della parte inferiore dell'onda è molto più pronunciata}
\end{figure}
\begin{figure}[htbp]
    \centering
	\includegraphics[scale=0.4]{amp.2500}
    \caption{Grafici di $V\ped{in}$ e $V\ped{out}$ in funzione del tempo con ampiezza $V\ped{in}=2500 \; \si{m\V}$; si inizia a intravedere un taglio nella parte superiore dell'onda, mentre la parte inferiore risulta ancora distorta}
\end{figure}
\begin{figure}[htbp]
    \centering
	\includegraphics[scale=0.4]{amp.3000}
    \caption{Grafici di $V\ped{in}$ e $V\ped{out}$ in funzione del tempo con ampiezza $V\ped{in}=3000 \; \si{m\V}$; il taglio della parte alta dell'onda risulta ora più evidente}
\end{figure}

%=======================
\section{Risposta in frequenza}
Analizzando il circuito ci si accorge che funzioni anche come filtro passa banda, in cui la frequenza di taglio del passa alto sarà definita come $\frac{1}{2 \pi R_G C_{in}} = 1.64 \pm 0.07 Hz$, perché la combinazione di $R_G$ e $C_{in}$ in quella determinata configurazione costituiscono un filtro passa alto. Inoltre  l'effetto Miller amplifica la capacità parassita tra l'input e l'output del Jfet, diminuendo così l'amplificazione ad alte frequenze
Utilizzando lo strumento Network analyzer siamo passati a misurare la risposta in frequenza del circuito tra i 5Hz e i 10 Mhz utilizzando una $V\ped{in}=200 \; \si{m\V}$\\
\begin{figure}[htbp]
    \centering
	\includegraphics[scale=0.4]{net}
    \caption{Risposta in frequenza del circuito amplificatore a common source}
\end{figure}
Si è quindi misurato il guadagno di centro banda, che risulta essere pari a $6.31 \pm 0.07 dB$ che convertendolo in $A_v$ si ottiene $2.07 \pm 0.02$, che risulta compatibile con quanto misurato al punto precedente.
A partire da ciò abbiamo ricavato le frequenze di taglio (basso e alto) tramite cursori, ricercando quando la curva del guadagno vale 3.31 dB:
\[
F_{TA}=3.54 \pm 0.02Mhz
\]
\[
F_{TB}= 16.0 \pm 0.1Hz
\]
La misura della frequenza di passa alto non è quindi compatibile con quanto misurato.
%=======================
\section{Aumento del guadagno}
Abbiamo ora inserito un condensatore elettrolitico in parallelo a $R_S$, chiamato $C_E$.
\begin{figure}[htbp]
    \centering
	\includegraphics[scale=0.4]{amp.200.cap}
    \caption{Grafici di $V\ped{in}$ e $V\ped{out}$ in funzione del tempo con ampiezza $V\ped{in}=200 \; \si{m\V}$ con condensatore elettrolitico $C_E$ in parallelo a $R_s$}
\end{figure}
Aggiungendo il condensatore $C_E$ viene a modificarsi l'impedenza in parallelo con $R_s$, che diminuisce, aumentando così il guadagno secondo:
\begin{equation}
A_v = -\frac{g_m R_d}{1+g_m (R_{eq})}
\end{equation}
dove $R_{eq}$ è l'impedenza in parallelo tra $R_s$ e $C_E$ che vale:
\begin{equation}
R_{eq} = \frac{R_s}{j\omega C R_s +1}
\end{equation}
che è un numero complesso.
Ricavandone il modulo si ottiene un valore di resistenza parallelo $C_E \parallel R_S$ di $1.68 \pm 0.09 \ohm$ per una frequenza di 1 kHz.
A questo punto si può ricavare il nuovo guadagno utilizzando la (4):
$A_v =3.88 \pm 0.06$.
A questo punto siamo passati a prendere delle misure per il guadagno a 1 kHz con un'ampiezza in ingresso di 100 mV, da cui si ricava $A_v=3.89 \pm 0.05$, risultato compatibile entro una barra di errore con l'aspettativa. Per effettuare al meglio questa misura è stato necessario agire più volte sul circuito, smontandolo e rimontandolo in seguito, e spostando i componenti (lasciando invariata la schematica), perché lo stesso era estremamente instabile e il fattore di amplificazione cambiava notevolmente anche al più piccolo urto sul tavolo, rendendo da quel momento in poi impossibile effettuare la misura.
%=======================
\section{Impedenza in ingresso}
Si è infine provata a misurare l'impedenza in ingresso al circuito mettendo in serie a $V\ped{in}$ un resistenza $R_S$ dello stesso ordine di grandezza dell'impedenza del circuito attesa, ed utilizzando lo strumento impedance analyzer di waveform in configurazione “W1-C1-R-C2-DUT-GND”; si è così valutata l'impedenza $Z_{IN}$ e la reattanza in parallelo $X_P$ in funzione della frequenza, compresa tra 1 e 10 kHz.
\begin{figure}[htbp]
    \centering
	\includegraphics[scale=0.4]{amp}
    \caption{Grafici di impedenza in ingresso e reattanza parallelo in funzione della frequenza per $R_S=1 \si{M\ohm}$}
\end{figure}

%=======================
\section*{Conclusioni e commenti finali}
Si è riusciti a costruire e caratterizzare un amplificatore di tensione
invertente con un JFET in configurazione a emettitore comune. In particolare
si è riusciti ad apprezzare il differente comportamento (anche non lineare)
del circuito in vari regimi, dare una stima di guadagno, impedenza di
ingresso e uscita e frequenze caratteristiche della sua risposta in frequenza.

%=======================
\section*{Dichiarazione}
I firmatari di questa relazione dichiarano che il contenuto della relazione \`e
originale, con misure effettuate dai membri del gruppo, e che tutti i firmatari
hanno contribuito alla elaborazione della relazione stessa.


\end{document}
