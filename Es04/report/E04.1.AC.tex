\documentclass[10pt,a4paper]{article}
\usepackage[T1]{fontenc}
\usepackage[utf8]{inputenc}
\usepackage{amsmath, amssymb, amsthm, thmtools, amsfonts, mathtools}
\usepackage{nicefrac}
\usepackage{calc}
\usepackage[pdftex, hyperindex, plainpages=false]{hyperref}
\usepackage[nameinlink]{cleveref} %load before classicthesis (clash)
%\usepackage[nochapters,pdfspacing]{classicthesis}
\usepackage{siunitx}
\usepackage[siunitx]{circuitikz}

\usepackage[a4paper]{geometry}
\usepackage{float}
\usepackage{mdframed}
\usepackage{titling}
\usepackage{booktabs}
\usepackage{graphicx}
\usepackage{caption, subcaption}
\usepackage{xcolor}
\usepackage[italian]{babel}
\usepackage{pgfplots}
\usepackage{listings}
%\usepackage{lmodern}
\usepackage{url}
\usepackage{enumitem}
\usepackage{tikz} %loads after classicthesis (xcolor incompat)

% lets graphicx know path where figures to be included are found
\graphicspath{{../figs/}}
\makeatletter
\def\input@path{{../figs/}}
%or: \def\input@path{{/path/to/folder/}{/path/to/other/folder/}}
\makeatother

% tikz pgf plots setup
\usepgfplotslibrary{external}
\pgfplotsset{compat=1.15}
\tikzexternalize

% spaces and significant digits/figures for measurements
\sisetup{free-standing-units, space-before-unit, number-unit-product = \;,
scientific-notation = true, round-mode = figures, round-precision = 2,}

% turns all (hyperlinked) references black [default is blue]
\hypersetup{
	linktoc=all,
	colorlinks=true,
	linkcolor=black
}

% code listings config
\lstset{
language=Python,
basicstyle=\ttfamily,
columns=fullflexible,
keepspaces=true,
}

% mdframed (for boxed text) configuration
\mdfsetup{linewidth=0.6pt}

% Default fixed font does not support bold face
\DeclareFixedFont{\ttb}{T1}{txtt}{bx}{n}{12} % for bold
\DeclareFixedFont{\ttm}{T1}{txtt}{m}{n}{12}  % for normal

% Custom colors
\usepackage{color}
\definecolor{deepblue}{rgb}{0,0,0.5}
\definecolor{deepred}{rgb}{0.6,0,0}
\definecolor{deepgreen}{rgb}{0,0.5,0}

% Commands 
\newcommand{\executeiffilenewer}[3]{%
	\ifnum\pdfstrcmp{\pdffilemoddate{#1}}%
		{\pdffilemoddate{#2}}>0%
	{\immediate\write18{#3}}\fi%
}
% input .svg --> .pdf_tex graphs
\newcommand{\includesvg}[1]{%
	\executeiffilenewer{#1.svg}{#1.pdf}%
	{inkscape -z -D --file=#1.svg %
	--export-pdf=#1.pdf --export-latex}%
	\input{#1.pdf_tex}%
}
% Thanks UniPi's Department of Physics E. Fermi
\newcommand{\thanksdf}{(\thanks{Dipartimento di Fisica E.~Fermi,%
Universit\`a di Pisa - Pisa, Italy.}\;)}

% hyperlink to email address
\newcommand{\mail}[1]{\href{mailto:#1}{\textsf{#1}}}

\input{../../latex/math}
\geometry{left=2cm, right=2cm, top=2cm, bottom=2cm}

% indexes subsections with letters, sections with numbers (1.a, 1.b, ...)
\renewcommand{\thesubsection}{\thesection.\alph{subsection}}

% lets graphicx know path where figures to be included are found
\graphicspath{{../figs/}}

\author{Gruppo 1.AC \\ Matteo Rossi, Bernardo Tomelleri}
\title{Es04A: Amplificatori operazionali e filtri attivi}
\begin{document}
\date{\today}
\maketitle

\setcounter{section}{0}

\section*{Misura componenti del circuito}
\begin{table}[ht]
\centering
\begin{tabular}{cccccc}
\toprule
Resistenze $[\si{\ohm}]$ & $R$ & $\sigma R$ & Capacità $[\si{n\F}]$ & $C$ &
$\sigma C$ \\
\midrule
\midrule
$R_1$	  & 998 	& 8 	 & $C$ & 50			 & 2 \\
$R_2^a$	  & 7.04		& 0.06 	 & $C'$ & 49		 & 2	\\
$R_2^f$	  & 9.85 k		& 0.08 k & & 90 & 5	\\
$R_3$	  & 998		& 8		 &				&				 &		\\
\bottomrule     
\end{tabular}
\caption{Valori di resistenza e capacità misurate per i componenti dei
circuiti studiati. \label{tab: rcmes}}
\end{table}
Riportiamo per completezza anche il valore calcolato della resistenza di base
\[
R_B = R_1 || R_2 = 8.70 \pm 0.07 \; \si{k\ohm}
\]
e i valori delle tensioni di alimentazione continue misurate con il multimetro
\begin{align*}
V_{CC} &= 4.99 \pm 0.03 \si{\V} \\
V_{EE} &= -4.99 \pm 0.03 \si{\V}
\end{align*}

\section{Circuito amplificatore invertente}
\subsection{Progettazione del circuito}
Scegliamo di costruire un amplificatore invertente a partire da un op-amp
TL081CP con impedenza in ingresso maggiore o uguale a $\SI{1}{k\ohm}$ e guadagno
$A\ped{v, atteso} = - \dfrac{R_2}{R_1}$ compreso (in valore assoluto) tra $5$
e $10$ come quello in figura \ref{fig: ampschm}

\begin{figure}[ht]
    \centering
    \begin{circuitikz}
        \draw
		(3,0) node[op amp] (opamp) {}
		(opamp.+) to[short] (1.8, -2) node[eground]{}
		(opamp.-) to[R=$ R_1 $, *-] (-0.8,0.5)
		(opamp.out) to (5,0) node[ocirc, label=$ v\ped{out} $]{}
		(opamp.up) --++(0,0.5) node[vcc]{$V_{CC} =$ 5\,\textnormal{V}}
		(opamp.down) --++(0,-0.5) node[vee]{$V_{EE} =$ -5\,\textnormal{V}};
		\draw
		(1.8, 0.5) to [short] (1.8, 2.7)
		to [R=$ R_2 $] (4.5, 2.7)
		to [short, -*] (4.5, 0);
		\draw
		(-0.8, -2) node[eground] {}
		to[sV, v=$ v\ped{in} $] (-0.8, 0.5);
		\draw
		(5, -1) --++ (0, 0.45) node[vcc] {}
		--++ (0, -1.4) node[vee] {};
    \end{circuitikz}
    \caption{Schema di massima dell'amplificatore invertente costruito.
    \label{fig: ampschm}}
\end{figure}

In condizione di op-amp ideale gli ingressi $+, -$ sono dei circuiti aperti,
per cui la stessa corrente scorre attraverso $R_1$ ed $R_2$:
$V_+ = V_- \approx 0 \implies R\ped{in} \approx R_1$, allora per
soddisfare la richiesta $5 \leq A_v \leq 10$ basta imporre
$5 R_1 \leq R_2 \leq 10 R_1$.

Dunque una volta fissata $R_1 = 1 \pm 1\% \; \si{k\ohm}$, dobbiamo avere
$\SI{5}{k\ohm} \leq R_2 \leq \SI{10}{k\ohm}$, di conseguenza scegliamo
$R_2 = 5.1 \pm 1\% \; \si{k\ohm}$, che corrisponde ad un guadagno
di centro banda $A\ped{v, atteso} = 5.1 \pm 2\%$ 

Con il multimetro digitale abbiamo misurato
\begin{align*}
V_{BE}^Q &= 630 \pm 4 \; \si{m\V} \\
V_{CE}^Q &= 3.67 \pm 0.03 \; \si{\V} \\
I_C^Q &= \frac{\Delta V_{R_C}}{R_C} = 1.134 \pm 0.011 \; \si{m\A} \\
\end{align*}

Prendendo come riferimento (arbitrario) il valore per la tensione di soglia
della giunzione BE $V_\gamma = 0.6 \pm 0.1 \; \si{\V}$ e come valore atteso
per la tensione al terminale di base del transistor
$\ds V\ped{B, exp} = \frac{V_{CC} - V_{EE}}{1 + R_1/R_2}$, ci aspettiamo di
trovare
\begin{align*}
V\ped{BE, exp}^Q &\approx V_\gamma = 0.6 \pm 0.1 \; \si{\V} \\
I\ped{C, exp}^Q &= \frac{V_B - V_{BE}^Q}{R_E + R_B/h_{FE}} =
1.09 \pm 0.05 \si{\m\A} \\
V\ped{CE, exp}^Q &= (V_{CC} - V_{EE}) - I_C^Q(R_C + R_E) = 3.9 \pm 0.2 \;
\si{\V}
\end{align*}

Se consideriamo l'equazione della retta di carico del BJT, indicando con
$V_0 = V_{CC} - V_{EE}$ la tensione di alimentazione,
\begin{equation*}
V_{CE}^Q = V_0 + R_C I_C^Q
\end{equation*}
per assicurarci che il transistor sia in mezzo alla zona attiva possiamo
richiedere che le componenti quiescenti della curva caratteristica siano a
metà tra i propri valori minimi $(0)$ e massimi, che corrispondono alle
intercette con gli assi della caratteristica di collettore $V_0$ e $V_0/R_C$.
In breve
\begin{align*}
V^Q\ped{CE, ideale} &= \frac{V0}{2} \approx 5 \; \si{\V} \\
I^Q\ped{C, ideale} &= \frac{V_0}{2R_C} \approx 1 \; \si{\mA}
\end{align*}

Per cui vediamo che l'intensità di corrente di collettore $I_C^Q$ misurata è
compatibile entro il $10 \%$ con il valore ideale, mentre la tensione ai capi
della giunzione $V_{CE}^Q$ non si trova altrettanto vicina alla metà attesa.
Questo comporterà un'asimmetria nella risposta a segnali alternanti di
ampiezza $v\ped{in}$ abbastanza grande da portare l'amplificatore in regime
non lineare. Infatti, al crescere di $v\ped{in}$, ci aspettiamo di incontrare
prima il regime di saturazione del transistor $V_{CE} < V_\gamma \; \si{\V}$,
$V_{BE} \sim V_\gamma$ di quello di interdizione.

\subsection{Amplificazione di piccoli segnali}
Si è inviato all'ingresso dell'amplificatore un segnale sinusoidale di
ampiezza $v\ped{in} = \SI{200}{m\V}$ e frequenza $\SI{5}{k\Hz}$.

Dunque abbiamo misurato l'ampiezza del segnale in uscita
$v\ped{out} = \pm \; \si{\V}$, ottenendo così una stima del guadagno del
circuito amplificatore $A_v = \dfrac{v\ped{out}}{v\ped{in}} = \pm $.

Come valori attesi otteniamo
\begin{align*}
V\ped{E, exp} &= R_E I_E \approx R_E I\ped{C, exp}^Q = 0.54 \pm 0.2 \; \si{\V} \\
V\ped{B, exp} &= \frac{V_{CC} - V_{EE}}{1 + R_1/R_2} = 1.24 \pm 0.13 \; \si{\V}
\\
V\ped{C, exp} &= R_C I\ped{C, exp}^Q = 5.5 \pm 0.2 \; \si{V} \\
\end{align*}

\subsection{Misure di guadagno al variare di $v\ped{in}$}
Misurando con l'oscilloscopio l'ampiezza dei segnali in ingresso $v\ped{in}$
e in uscita $v\ped{out}$ dall'amplificatore possiamo ricavare una misura del
guadagno del circuito dal rapporto $A_v = \dfrac{v\ped{out}}{v\ped{in}}$.
\begin{table}[htb]
\centering
\begin{tabular}{cccc}
\toprule
$v\ped{in}(\si{m\V})$ (nom.) & $v\ped{in} \pm \sigma(v\ped{in})$ [mV] & $v\ped{out} \pm \sigma(v\ped{out})$ [V] & $A_v \pm \sigma(A_v)$ \\
\midrule
\midrule
20 & $19.8 \pm 0.5$ & $192 \pm 2 \; \si{m}$ & $9.7 \pm 0.3$ \\
40 & $39.9 \pm 0.8$ & $383 \pm 3 \; \si{m}$ & $9.6 \pm 0.2$ \\
60 & $59.8 \pm 1.0$ & $576 \pm 4 \; \si{m}$ & $9.63 \pm 0.17$ \\
80 & $80.0 \pm 1.1$ & $768 \pm 5 \; \si{m}$ & $9.60 \pm 0.15$ \\
100 & $100.0 \pm 1.2$ & $960 \pm 6 \; \si{m}$ & $9.60 \pm 0.13$ \\
125 & $125.0 \pm 1.5$ & $1200 \pm 7$ & $9.60 \pm 0.13$ \\
150 & $150.0 \pm 1.6$ & $1439 \pm 8$ & $9.60 \pm 0.12$ \\
175 & $175.2 \pm 1.8$ & $1.678 \pm 0.010$ & $9.58 \pm 0.11$ \\
200 & $200 \pm 2$ & $1.916 \pm 0.011$ & $9.58 \pm 0.11$ \\
225 & $225 \pm 2$ & $2.155 \pm 0.012$ & $9.58 \pm 0.11$ \\
250 & $250 \pm 2$ & $2.438 \pm 0.014$ & $9.75 \pm 0.11$ \\
275 & $275 \pm 2$ & $2.679 \pm 0.015$ & $9.74 \pm 0.11$ \\
300 & $300 \pm 3$ & $2.922 \pm 0.016$ & $9.74 \pm 0.11$ \\
325 & $325 \pm 3$ & $3.17 \pm 0.02$ & $9.75\pm 0.10$ \\
\bottomrule
\end{tabular} 
\caption{Misure di guadagno al variare della tensione in ingresso
all'amplificatore \label{tab: bjtmes}}
\end{table}

Con un fit lineare possiamo stimare il guadagno dell'amplificatore a partire
dal grafico di $v\ped{out} = A_v v\ped{in}$ al variare di $v\ped{in}$.
\begin{figure}[htb]
\centering
%\includegraphics[scale=0.6]{gain}
\caption{Fit lineare per l'andamento dell'uscita rispetto al segnale in
ingresso. \label{fig: gainfit}}
\end{figure}
Da cui troviamo i seguenti parametri per la retta di best-fit
\begin{align*}
\mathrm{intercetta} = -2 \pm 3 \;\;\;\mathrm{pendenza} = 9.66 \pm 0.03 \;\;\;\mathrm{correlazione} 
= -0.72 \;\;\; \chi^2 = 3 \;\;\; d.o.f. = 11 \\
\text{coefficiente angolare/senza intercetta} = 9.65 \pm 0.02 \;\;\;
\chi^2 = 3 \;\;\; d.o.f. = 12
\end{align*}

Il valore atteso per il guadagno dal valore dei componenti in questa
configurazione del circuito è pari a
\[
A\ped{v, exp} = -\frac{R_2}{R_1} = - 5.1 \pm 0.1
\]
Questo è compatibile con quanto trovato sperimentalmente, specialmente
tenendo conto della notevole indeterminazione sul valore dei parametri di
costruzione dell'opamp.

\subsection{Impedenza in ingresso}
Inserendo in serie al generatore una resistenza $R_S$ dello stesso ordine di
$R\ped{in}$ attesa e misurando la tensione in uscita con o senza $R_S$ è
possibile dare un stima della resistenza in ingresso del circuito.
Detta $V_1$ la tensione $V\ped{out}$ misurata senza $R_S$ e $V_2$ la tensione
misurata con $R_S$ inserita, vale l'equazione:
\begin{equation}\label{eq: Zin}
\frac{R_S}{R\ped{in}} = \frac{V_1}{V_2} - 1
\end{equation}
Abbiamo preso come $R_S$ la serie di tre resistenze $R_S =
R_{S_s} + R_{S_t} + R_{S_u} =
\SI{5.1}{k\ohm} + \SI{2}{k\ohm} + \SI{330}{\ohm} =
\SI{7.4}{k\ohm}$
Per cui, avendo misurato $V_1 = 1916 \pm 11 \; \si{m\V}$ e
$V_2 = 1011 \pm 7 \; \si{m\V}$, troviamo come resistenza in ingresso:
\begin{align*}
{R\ped{in}} &= \frac{R_S}{V_1/V_2 - 1} = 8.26 \pm 0.17
\end{align*}
Che è compatibile (entro le incertezze con cui sono noti i parametri del
transistor $\sim 10 \%$) con il valore atteso dalla \eqref{eq: Zin}.

\section{Risposta in frequenza e slew rate}
\subsection{Network analyzer}

\begin{figure}[htb]
\centering
%\includegraphics[scale=0.35]{1-10Mnet}
\caption{Plot di Bode ottenuto dallo scan con Network tra $\SI{100}{\Hz}$ e
$\SI{5}{M\Hz}$ con un segnale sinusoidale in ingresso all'amplificatore
invertente di ampiezza costante $v\ped{in} = \SI{200}{m\V}$. \label{fig: invbode}}
\end{figure}

Se i due segnali sono in opposizione di fase il passaggio per 0 con la stessa
pendenza/slope devono distare un semi-periodo dall'altro; come si vede bene
dalla figura \ref{fig: Alin} in cui ai massimi del segnale in ingresso (la
traccia gialla) corrispondono i minimi del segnale in uscita (la traccia blu)
\begin{figure}[htb]
\centering
%\includegraphics[scale=0.335]{Alin200mV}
\caption{Risposta del circuito ad un segnale sinusoidale di ampiezza
$\SI{200}{m\V}$ e $f = \SI{1}{k\Hz}$ in ingresso. Quando l'amplificatore
è in pieno regime attivo. \label{fig: Alin}}
\end{figure}

Da una misura con i cursori troviamo
\begin{align*}
\Delta t &= 50.2 \pm 1.0 \; \si{n\s}\\
\Delta \phi &= 2\pi f \Delta t = 3.14 \pm 0.06 \; \si{\radian}
\end{align*}

mentre con la funzione di misura automatica definita con uno script di Wavegen
risulta:
\[
\phi = 179.63 \pm 0.10 \; \si{\degree}
\]

che sono entrambi compatibili con il valore atteso di $\Delta \phi\ped{exp}
= \pi \; \si{\radian}$ per la natura invertente dell'amplificatore.

\subsection{Misura dello slew rate}
Misurando con l'oscilloscopio l'ampiezza dei segnali in ingresso $v\ped{in}$
e in uscita $v\ped{out}$ dall'amplificatore possiamo ricavare una misura del
guadagno del circuito dal rapporto $A_v = \dfrac{v\ped{out}}{v\ped{in}}$.
\begin{table}[htb]
\centering
\begin{tabular}{cccc}
\toprule
$v\ped{in}(\si{m\V})$ (nom.) & $v\ped{in} \pm \sigma(v\ped{in})$ [mV] & $v\ped{out} \pm \sigma(v\ped{out})$ [V] & $A_v \pm \sigma(A_v)$ \\
\midrule
\midrule
20 & $19.8 \pm 0.5$ & $192 \pm 2 \; \si{m}$ & $9.7 \pm 0.3$ \\
40 & $39.9 \pm 0.8$ & $383 \pm 3 \; \si{m}$ & $9.6 \pm 0.2$ \\
60 & $59.8 \pm 1.0$ & $576 \pm 4 \; \si{m}$ & $9.63 \pm 0.17$ \\
80 & $80.0 \pm 1.1$ & $768 \pm 5 \; \si{m}$ & $9.60 \pm 0.15$ \\
100 & $100.0 \pm 1.2$ & $960 \pm 6 \; \si{m}$ & $9.60 \pm 0.13$ \\
125 & $125.0 \pm 1.5$ & $1200 \pm 7$ & $9.60 \pm 0.13$ \\
150 & $150.0 \pm 1.6$ & $1439 \pm 8$ & $9.60 \pm 0.12$ \\
175 & $175.2 \pm 1.8$ & $1.678 \pm 0.010$ & $9.58 \pm 0.11$ \\
200 & $200 \pm 2$ & $1.916 \pm 0.011$ & $9.58 \pm 0.11$ \\
225 & $225 \pm 2$ & $2.155 \pm 0.012$ & $9.58 \pm 0.11$ \\
250 & $250 \pm 2$ & $2.438 \pm 0.014$ & $9.75 \pm 0.11$ \\
275 & $275 \pm 2$ & $2.679 \pm 0.015$ & $9.74 \pm 0.11$ \\
300 & $300 \pm 3$ & $2.922 \pm 0.016$ & $9.74 \pm 0.11$ \\
325 & $325 \pm 3$ & $3.17 \pm 0.02$ & $9.75\pm 0.10$ \\
\bottomrule
\end{tabular} 
\caption{Misure di guadagno al variare della tensione in ingresso
all'amplificatore \label{tab: bjtmes}}
\end{table}

Con un fit lineare possiamo stimare il guadagno dell'amplificatore a partire
dal grafico di $v\ped{out} = A_v v\ped{in}$ al variare di $v\ped{in}$.
\begin{figure}[htb]
\centering
%\includegraphics[scale=0.6]{gain}
\caption{Fit lineare per l'andamento dell'uscita rispetto al segnale in
ingresso. \label{fig: gainfit}}
\end{figure}
Da cui troviamo i seguenti parametri per la retta di best-fit
\begin{align*}
\mathrm{intercetta} = -2 \pm 3 \;\;\;\mathrm{pendenza} = 9.66 \pm 0.03 \;\;\;\mathrm{correlazione} 
= -0.72 \;\;\; \chi^2 = 3 \;\;\; d.o.f. = 11 \\
\text{coefficiente angolare/senza intercetta} = 9.65 \pm 0.02 \;\;\;
\chi^2 = 3 \;\;\; d.o.f. = 12
\end{align*}

Il valore atteso per il guadagno dal valore dei componenti in questa
configurazione del circuito è pari a
\[
A\ped{v, exp} = -\frac{R_C}{R_E + h_{ie}/h_{fe}} = -9.44 \pm 0.12
\]
Questo è compatibile con quanto trovato sperimentalmente, specialmente
tenendo conto della notevole indeterminazione sul valore dei parametri di
costruzione del transistor.

\section{Circuito derivatore attivo}
\subsection{Risposta in frequenza}
\begin{figure}[htb]
\centering
%\includegraphics[scale=0.35]{1-10Mnet}
\caption{Plot di Bode ottenuto dallo scan con Network tra $\SI{100}{\Hz}$ e
$\SI{5}{M\Hz}$ con un segnale sinusoidale in ingresso al derivatore RC attivo
di ampiezza costante $v\ped{in} = \SI{200}{m\V}$.
\label{fig: derbode}}
\end{figure}

Come valore atteso per l'impedenza in ingresso al circuito abbiamo:
\[
Z\ped{in}(\omega) =
h_{ie} + h_{fe}Z_E(\omega) \; \lvert\rvert \; R_B =
\left(\frac{1}{h_{ie} + h_{fe}Z_E} + \frac{1}{R_1} + \frac{1}{R_2}\right)^{-1}
= 7.5 \pm 10\% \; \si{k\ohm} 
\]

dove abbiamo indicato con $Z_E(\omega)$ l'impedenza del ramo di emettitore,
che nel nostro circuito vale $Z_E = R_E$; meno che nel punto 5, dove in
parallelo a $R_E$ si aggiunge un passa alto costruito con $C_E + R\ped{es}$,
per cui vale $Z_E(\omega) = R_E || \left(R\ped{es} + \frac{1}{j \omega C_E}
\right)$.

\subsection{Risposta ad un'onda triangolare}
Si è inviato all'ingresso del filtro passa-alto un segnale triangolare di
ampiezza $v\ped{in} = \SI{200}{m\V}$ e frequenza $\SI{5}{k\Hz}$.

Inserendo tra l'uscita e la massa una resistenza di carico $R_L$ dello stesso
ordine di $R\ped{out}$ e misurando la tensione di uscita con o senza
resistenza è possibile dare una stima della resistenza in uscita
dell'amplificatore.
Detta $V_1$ la tensione misurata senza $R_L$ e $V_2$ la tensione misurata
con $R_L$, vale la formula:
\begin{equation}\label{eq: Zout}
\frac{R\ped{out}}{R_L} = \frac{V_1}{V_2} -1
\end{equation}
Per cui, una volta misurate $V_1 = 1725 \pm 8 \; \si{m\V}$,
$V_2 = 866 \pm 4\; \si{m\V}$ e $R_L = 5.08 \pm 0.05 \si{k\ohm}$ abbiamo ottenuto come impedenza d'uscita:
\[
R\ped{out} = R_L \left(\frac{V_1}{V_2} - 1\right)
\]

Risulta $R\ped{out} = 5.0 \pm 0.1\; \si{k\ohm}$ che è compatibile con la stima iniziale dell'impedenza.

\subsection{Confronto con i valori attesi}

%=======================
\section{Circuito integratore attivo}
\subsection{Risposta in frequenza}
\begin{figure}[htb]
\centering
%\includegraphics[scale=0.35]{1-10Mnet}
\caption{Plot di Bode ottenuto dallo scan con Network tra $\SI{10}{\Hz}$ e
$\SI{5}{M\Hz}$ con un segnale sinusoidale in ingresso all'integratore RC
attivo di ampiezza costante $v\ped{in} = \SI{200}{m\V}$.
\label{fig: derbode}}
\end{figure}

\subsection{Risposta ad un'onda quadra @ 10 kHz}
Si è inviato all'ingresso del filtro passa-basso un'onda quadra di
ampiezza $v\ped{in} = \SI{200}{m\V}$ e frequenza $\SI{5}{k\Hz}$.

Partendo da una misura con i cursori del guadagno a centro banda,
$A_V = 19.65 \pm 0.05 \; \si{dB} = 9.65 \pm 0.08$, possiamo ottenere una stima del valore
delle frequenze di taglio a bassa $f_L$ e ad alta frequenza $f_H$ dai punti
in cui il guadagno diminuisce di un fattore $1/\sqrt{2}$, cioè di circa
$-3.01 \; \si{dB}$ rispetto ad $A_V$.
\begin{align*}
f_L &= 80.77 \pm 0.12 \; \si{\Hz}\\
f_H &= 646.1 \pm 0.5 \; \si{k\Hz}
\end{align*}

Trascurando le capacità delle giunzioni nel transistor ci aspettiamo che
la frequenza di taglio ``bassa'' corrisponda a quella di un filtro passa~alto
costituito dalla serie $C\ped{in} + R_B$
\begin{equation}
f\ped{L, exp} = \frac{1}{2\pi R_B C\ped{in}} = 83 \pm 4 \; \si{\Hz}
\end{equation} 
che è in accordo con il valore misurato.

Mentre per la frequenza di taglio ``alta'' la resistenza in uscita è data
da $R_C$, per cui la capacità in serie dev'essere dell'ordine delle centinaia
di pF per avere ordine di grandezza compatibile con il valore misurato.
Ma nel datasheet risulta al massimo $C\ped{ibo} \approx \SI{25}{p\F}$, per cui
è difficile stabilire un valore di riferimento per la frequenza $f_H$ attesa.

\section{Circuito amplificatore non invertente}
\begin{figure}[htb]
\centering
%\includegraphics[scale=0.35]{1-10Mnet}
\caption{Plot di Bode ottenuto dallo scan con Network tra $\SI{100}{\Hz}$ e
$\SI{5}{M\Hz}$ con un segnale sinusoidale in ingresso all'amplificatore non
invertente di ampiezza costante $v\ped{in} = \SI{200}{m\V}$.
\label{fig: derbode}}
\end{figure}

Per mitigare la diminuzione del guadagno dovuta alla resistenza tra emettitore
e $V_{EE}$ si inserisce in parallelo a questa la serie $R\ped{es} + C_E$,
in modo tale che $R_E$ sia vista ``per intero'' solamente in condizioni
stazionarie (cioè dalle tensioni e correnti continue di alimentazione).
Al contrario, per frequenze abbastanza alte il condensatore si comporterà come
un corto circuito, per cui la resistenza del parallelo tenderà al valore più
piccolo tra le due resistenze, cioè $R\ped{es} \ll R_E$.
Quindi in breve l'impedenza all'emettitore si comporterà grossolanamente
come un filtro passa alto.

\subsection{Risposta in frequenza}

\begin{figure}[htb]
\centering
%\includegraphics[scale=0.35]{1-10Mnet}
\caption{Plot di Bode ottenuto dallo scan con Network tra $\SI{100}{\Hz}$ e
$\SI{5}{M\Hz}$ con un segnale sinusoidale in ingresso all'amplificatore
non-invertente di ampiezza costante $v\ped{in} = \SI{200}{m\V}$.
\label{fig: ampbode}}
\end{figure}

\subsection{Misure di guadagno e frequenza di taglio}
Partendo da una misura con i cursori del guadagno a centro banda,
$A_V = 19.65 \pm 0.05 \; \si{dB} = 9.65 \pm 0.08$, possiamo ottenere una stima del valore
delle frequenze di taglio a bassa $f_L$ e ad alta frequenza $f_H$ dai punti
in cui il guadagno diminuisce di un fattore $1/\sqrt{2}$, cioè di circa
$-3.01 \; \si{dB}$ rispetto ad $A_V$.
\begin{align*}
f_L &= 80.77 \pm 0.12 \; \si{\Hz}\\
f_H &= 646.1 \pm 0.5 \; \si{k\Hz}
\end{align*}

Una volta inserito il ramo in parallelo a $R_E$, dalla formula per il guadagno
atteso otteniamo
\[
A_v = -\frac{R_C}{\abs{Z_E}} =
- \frac{R_C}{R_E \; \lvert \rvert \; \left(R\ped{es} + \abs{1/j\omega C_E}
\right)} =
- R_C \abs{\frac{1}{R_E} + \frac{1}{R\ped{es} + 1/\omega C_E}}
\]

Visto che abbiamo scelto $C_E \gg C\ped{in} \sim C\ped{out}$, alla frequenza
di lavoro $f = \SI{10}{k\Hz}$ possiamo considerare trascurabile l'impedenza
del condensatore $\abs{Z_{C_E}} = \dfrac{1}{2\pi f C_E} \approx 0.1 \si{\ohm}
\ll R\ped{es}$, per cui in buona approssimazione ci aspettiamo
\[
\abs{A_v} \approx
\frac{R_C}{R_E \; \lvert \rvert \; R\ped{es}} =
R_C \abs{\frac{1}{R_E} + \frac{1}{R\ped{es}}} \approx
\frac{R_C}{R\ped{es}} = 110 \pm 1
\]

Questo però assumendo che l'impedenza del transistor sia trascurabile rispetto
a $Z_E$, o meglio $\abs{Z_E} \gg \dfrac{h_{ie}}{h_{fe}}$
\begin{align*}
\abs{Z_E} &= \abs{\frac{1}{R_E} + \frac{1}{R\ped{es} + 1/\omega C_E}}^{-1} =
45 \pm 2 \; \si{\ohm} \\
\frac{h_{ie}}{h_{fe}} &\approx \SI{40}{\ohm}
\end{align*}
che non risulta affatto verificata.

Considerando nel modello anche l'impedenza in ingresso del transistor in
serie a quella del ramo $Z_E$ avremo come valore atteso per il guadagno
\begin{equation}
A_v = \frac{R_C}{\abs{Z_E} + h_{ie}/h_{fe}} \approx 60
\end{equation}

Che è in buon accordo con il valore misurato per il guadagno sempre
entro le grandi incertezze relative sui parametri di costruzione del
transistor.

\begin{figure}[htb]
\centering
%\includegraphics[scale=0.335]{Alin200mV}
\caption{Sovrapposizione dei plot di Bode ottenuti per l'amplificatore
non-invertente. \label{fig: prdbode}}
\end{figure}

\section*{Conclusioni e commenti finali}
Si è riusciti a costruire e studiare alcuni dei circuiti più comuni che si
possono realizzare con un amplificatore operazionale, tra cui: due filtri
attivi, passa-basso e passa-alto, un amplificatore di tensione invertente
(e uno non).
In particolare siamo riusciti ad apprezzare il differente comportamento dei
circuiti (anche in regime non lineare) dare una stima di guadagno, impedenza di
ingresso e frequenze caratteristiche della loro risposta in frequenza.

\section*{Dichiarazione}
I firmatari di questa relazione dichiarano che il contenuto della relazione \`e
originale, con misure effettuate dai membri del gruppo, e che tutti i firmatari
hanno contribuito alla elaborazione della relazione stessa.

\end{document}
